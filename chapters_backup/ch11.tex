\chapter{كيف تراقب التقدم؟}
\label{ch:11}

\section*{من الميدان}

في إحدى زياراتي، فتحت الابنة دفتراً صغيراً على الطاولة. صفحات مليئة بتواريخ، أسهم، وملصقات صغيرة. خلال دقائق عرفت أن والدها: - تمرّن أربعة أيام هذا الأسبوع، - تعثر مرتين في الممر ليلاً، - أصبح يحتاج "لمسة خفيفة" فقط للقيام من الكرسي.

هذا الدفتر البسيط وفّر عليّ نصف الأسئلة، وساعدني أعدّل الخطة بسرعة. ليس ملفاً طبياً ولا وثيقة رسمية، لكنه \textbf{مرآة صادقة} لما يحدث بين الزيارات.

\section*{لماذا أصر على وجود دفتر منزلي؟}

ليس لأنني أحب "الأوراق" أو زيادة مسؤولياتكم، بل لأن الدفتر يغيّر شكل المتابعة بالكامل. عندما تكتب الأسرة يومياً متى تمرّن كبير السن، وعدد المرات تقريباً، وأي سقوط أو تعثر، تتحول الزيارة من أسئلة عامة إلى نقاش دقيق: - نرى معاً إن كان الالتزام بالتمارين ثابتاً أم متقطعاً. - نربط بين أوقات السقوط وأماكنها لنكتشف "نقاط الخطر" في البيت. - نلاحظ التحسن البسيط الذي قد لا يُلاحظ في الزحام اليومي (مثل: "اليوم قام من الكرسي بلمسة خفيفة فقط"). - ونستخدم الدفتر كمصدر تحفيز نفسي، عندما يرى المريض نفسه أنه "تقدم خطوة" مقارنة بالأسبوع الماضي.

لذلك أقول دائماً: الدفتر الصغير في البيت أحياناً أهم من التقرير الطويل في المستشفى.

\section*{ماذا ندوّن يومياً؟ (قالب جاهز للنسخ)}

 \textbf{نصيحة سريعة:} خصصوا صفحة لكل أسبوع. اكتبوا بقلم عادي. لا يحتاج الدفتر لتطبيق أو طابعة فاخرة.

\begin{longtable}[]{@{}
  >{\raggedright\arraybackslash}p{(\columnwidth - 10\tabcolsep) * \real{0.1515}}
  >{\raggedright\arraybackslash}p{(\columnwidth - 10\tabcolsep) * \real{0.1919}}
  >{\raggedright\arraybackslash}p{(\columnwidth - 10\tabcolsep) * \real{0.1717}}
  >{\raggedright\arraybackslash}p{(\columnwidth - 10\tabcolsep) * \real{0.1818}}
  >{\raggedright\arraybackslash}p{(\columnwidth - 10\tabcolsep) * \real{0.1414}}
  >{\raggedright\arraybackslash}p{(\columnwidth - 10\tabcolsep) * \real{0.1616}}@{}}
\toprule\noalign{}
\begin{minipage}[b]{\linewidth}\raggedright
اليوم/التاريخ
\end{minipage} & \begin{minipage}[b]{\linewidth}\raggedright
التمرين أو النشاط
\end{minipage} & \begin{minipage}[b]{\linewidth}\raggedright
المدة/التكرارات
\end{minipage} & \begin{minipage}[b]{\linewidth}\raggedright
شدة الجهد (0-10)
\end{minipage} & \begin{minipage}[b]{\linewidth}\raggedright
الألم (0-10)
\end{minipage} & \begin{minipage}[b]{\linewidth}\raggedright
ملاحظات الأسرة
\end{minipage} \\
\begin{minipage}[b]{\linewidth}\raggedright
الأحد
\end{minipage} & \begin{minipage}[b]{\linewidth}\raggedright
المشي في الممر
\end{minipage} & \begin{minipage}[b]{\linewidth}\raggedright
8 دقائق
\end{minipage} & \begin{minipage}[b]{\linewidth}\raggedright
4
\end{minipage} & \begin{minipage}[b]{\linewidth}\raggedright
1
\end{minipage} & \begin{minipage}[b]{\linewidth}\raggedright
تعب خفيف بعد الدقيقة 6
\end{minipage} \\
\begin{minipage}[b]{\linewidth}\raggedright
الاثنين
\end{minipage} & \begin{minipage}[b]{\linewidth}\raggedright
القيام من الكرسي
\end{minipage} & \begin{minipage}[b]{\linewidth}\raggedright
3 × 6 تكرار
\end{minipage} & \begin{minipage}[b]{\linewidth}\raggedright
5
\end{minipage} & \begin{minipage}[b]{\linewidth}\raggedright
2
\end{minipage} & \begin{minipage}[b]{\linewidth}\raggedright
احتاج لمسة خفيفة للمساعدة
\end{minipage} \\
\begin{minipage}[b]{\linewidth}\raggedright
\ldots }
\end{minipage} & \begin{minipage}[b]{\linewidth}\raggedright
\ldots }
\end{minipage} & \begin{minipage}[b]{\linewidth}\raggedright
\ldots }
\end{minipage} & \begin{minipage}[b]{\linewidth}\raggedright
\ldots }
\end{minipage} & \begin{minipage}[b]{\linewidth}\raggedright
\ldots }
\end{minipage} & \begin{minipage}[b]{\linewidth}\raggedright
\ldots }
\end{minipage} \\
\midrule\noalign{}
\endhead
\bottomrule\noalign{}
\endlastfoot
\end{longtable}

\subsection*{للذين لا يحبون الجداول}

بعض الأسر لا ترتاح للجداول والخانات. لا مشكلة. يمكن أن يكون الدفتر بكل بساطة: - سطر أو سطران لكل يوم، مثل: "الاثنين: مشى في الممر ٣ مرات ذهاباً وإياباً. تعب بعد الثالثة. سقطت رجله مرة بدون سقوط." "الثلاثاء: لم يتمرن، كان متعباً بسبب قلة النوم."

المهم هو الاستمرارية والوضوح، وليس شكل الجدول أو عدد الأعمدة.

\textbf{طريقة التعبئة السريعة:} - اكتبوا اسم النشاط (مشي، تمرين كرسي، صعود درج بسيط). - سجّلوا الزمن أو التكرارات التقريبية. - \textbf{شدة الجهد (}\textbf{RPE 0-10):} استهدفوا 3-5 لمعظم الأنشطة؛ 0 يعني بلا جهد، 10 يعني مجهد جداً. - \textbf{الألم (0-10):} إذا وصل إلى 6 أو أكثر أو كان حاداً، أوقفوا التمرين ودوّنوا الموقف. - \textbf{الملاحظات:} أي شيء لافت: تعب أسرع من المعتاد، تحسن ملحوظ، خوف من تمرين معين.

 \textbf{حدود الأمان:} - إذا ظهر \textbf{دوار شديد، ألم حاد، ضيق نفس، أو سقوط} أثناء النشاط → أوقفوا التمرين، دوّنوا ما حدث، واتصلوا بالأخصائي أو الطوارئ حسب الشدة. - الدفتر \textbf{لا يغني} عن التقييم الطبي. أي تغير مفاجئ (إغماء، ضعف مفاجئ في طرف، حرارة مرتفعة) يحتاج طبيباً فوراً.

\section*{كيف نسجل حوادث السقوط والتعثر؟}

 التفاصيل الدقيقة تساعدنا على تعديل البيئة والتمارين لتقليل الخطر لاحقاً.

\begin{longtable}[]{@{}
  >{\raggedright\arraybackslash}p{(\columnwidth - 10\tabcolsep) * \real{0.1786}}
  >{\raggedright\arraybackslash}p{(\columnwidth - 10\tabcolsep) * \real{0.0952}}
  >{\raggedright\arraybackslash}p{(\columnwidth - 10\tabcolsep) * \real{0.2619}}
  >{\raggedright\arraybackslash}p{(\columnwidth - 10\tabcolsep) * \real{0.1667}}
  >{\raggedright\arraybackslash}p{(\columnwidth - 10\tabcolsep) * \real{0.1071}}
  >{\raggedright\arraybackslash}p{(\columnwidth - 10\tabcolsep) * \real{0.1905}}@{}}
\toprule\noalign{}
\begin{minipage}[b]{\linewidth}\raggedright
التاريخ/الوقت
\end{minipage} & \begin{minipage}[b]{\linewidth}\raggedright
المكان
\end{minipage} & \begin{minipage}[b]{\linewidth}\raggedright
ما كان يفعله المريض؟
\end{minipage} & \begin{minipage}[b]{\linewidth}\raggedright
السبب الظاهر
\end{minipage} & \begin{minipage}[b]{\linewidth}\raggedright
النتيجة
\end{minipage} & \begin{minipage}[b]{\linewidth}\raggedright
الإجراء المتخذ
\end{minipage} \\
\begin{minipage}[b]{\linewidth}\raggedright
10 شعبان، 7:30 م
\end{minipage} & \begin{minipage}[b]{\linewidth}\raggedright
غرفة المعيشة
\end{minipage} & \begin{minipage}[b]{\linewidth}\raggedright
دوران للعودة إلى الكرسي
\end{minipage} & \begin{minipage}[b]{\linewidth}\raggedright
سجادة بارزة
\end{minipage} & \begin{minipage}[b]{\linewidth}\raggedright
كدمة بسيطة في الركبة
\end{minipage} & \begin{minipage}[b]{\linewidth}\raggedright
وضع ثلج، أُبلغ الأخصائي
\end{minipage} \\
\midrule\noalign{}
\endhead
\bottomrule\noalign{}
\endlastfoot
\end{longtable}
\begin{itemize}
\item
    \textbf{المكان:} داخل البيت/خارجه + الغرفة.
  
\item
    \textbf{النشاط:} مشي، قيام من الكرسي، دخول الحمام\ldots }
  
\item
    \textbf{السبب الظاهر:} سجادة، إضاءة ضعيفة، تعب شديد، دوران سريع.
  
\item
    \textbf{النتيجة:} لا إصابة، كدمة، طلب طوارئ.
  
\item
    \textbf{الإجراء:} اتصلنا بالطبيب، راحة، إزالة السجادة.
  
\end{itemize}

\textbf{🛑 اتصلوا بالطوارئ (997) فوراً إذا:} - ظهر \textbf{ألم صدر أو ضيق نفس حاد} بعد السقوط. - حدث \textbf{فقدان وعي} أو ارتباك شديد. - عجز المريض عن الوقوف أو شعر بألم حاد في الورك/الركبة.

\section*{ملاحظة التغير الوظيفي في الأنشطة اليومية}

 ركزوا على 3-5 مهام أساسية يمكن قياسها كل أسبوع.

\begin{longtable}[]{@{}
  >{\raggedright\arraybackslash}p{(\columnwidth - 8\tabcolsep) * \real{0.1159}}
  >{\raggedright\arraybackslash}p{(\columnwidth - 8\tabcolsep) * \real{0.2174}}
  >{\raggedright\arraybackslash}p{(\columnwidth - 8\tabcolsep) * \real{0.1884}}
  >{\raggedright\arraybackslash}p{(\columnwidth - 8\tabcolsep) * \real{0.3478}}
  >{\raggedright\arraybackslash}p{(\columnwidth - 8\tabcolsep) * \real{0.1305}}@{}}
\toprule\noalign{}
\begin{minipage}[b]{\linewidth}\raggedright
المهمة
\end{minipage} & \begin{minipage}[b]{\linewidth}\raggedright
الشهر الماضي
\end{minipage} & \begin{minipage}[b]{\linewidth}\raggedright
هذا الأسبوع
\end{minipage} & \begin{minipage}[b]{\linewidth}\raggedright
المساعدة المطلوبة اليوم
\end{minipage} & \begin{minipage}[b]{\linewidth}\raggedright
ملاحظات
\end{minipage} \\
\begin{minipage}[b]{\linewidth}\raggedright
القيام من الكرسي
\end{minipage} & \begin{minipage}[b]{\linewidth}\raggedright
يحتاج سحب باليدين + دفع من شخص
\end{minipage} & \begin{minipage}[b]{\linewidth}\raggedright
يقوم مع لمسة خفيفة
\end{minipage} & \begin{minipage}[b]{\linewidth}\raggedright
لمسة خفيفة
\end{minipage} & \begin{minipage}[b]{\linewidth}\raggedright
تقدم واضح، ثقة أعلى
\end{minipage} \\
\begin{minipage}[b]{\linewidth}\raggedright
المشي للحمام
\end{minipage} & \begin{minipage}[b]{\linewidth}\raggedright
يحتاج مرافقة قريبة
\end{minipage} & \begin{minipage}[b]{\linewidth}\raggedright
يمشي بالعصا وحده
\end{minipage} & \begin{minipage}[b]{\linewidth}\raggedright
مراقبة من بعيد
\end{minipage} & \begin{minipage}[b]{\linewidth}\raggedright
لا سقوط
\end{minipage} \\
\midrule\noalign{}
\endhead
\bottomrule\noalign{}
\endlastfoot
\end{longtable}
\begin{itemize}
\item
    اختاروا مهام مثل: القيام من الكرسي، المشي داخل البيت، دخول الحمام، صعود درج قصير (إن وجد).
  
\item
    استخدموا كلمات بسيطة للمساعدة: \textbf{مستقل، مراقبة قريبة، لمسة خفيفة، مساعدة جزئية، مساعدة كاملة}.
  
\item
    إذا احتاج المريض \textbf{مساعدة أكبر فجأة} أو زادت مدة الإنجاز بشكل واضح، سجّلوا ذلك وتواصلوا مع المختص.
  
\end{itemize}

\section*{كيف أستخدم الدفتر مع أخصائي العلاج الطبيعي؟}

في بداية كل زيارة، ضعوا الدفتر أمام الأخصائي واتركوه يقرأ بهدوء. خلال دقائق سيعرف: - هل التمارين مناسبة أم تحتاج تعديل. - هل هناك وقت معين في اليوم يزيد فيه التعب أو الألم. - هل السقوط يتكرر في مكان محدد (مثل الحمام أو الممر الليلي).

بهذه الطريقة، تصبحون شركاء حقيقيين في الخطة، لا مجرد متلقين لتعليمات.

\section*{كيف نحافظ على الحماس؟}
\begin{itemize}
\item
    اجعلوا التدوين لا يتجاوز \textbf{3 دقائق} بعد النشاط.
  
\item
    وزّعوا المسؤولية بين أفراد الأسرة حتى لا يتعب شخص واحد.
  
\item
    استخدموا ملصقات أو ألوان:  لليوم المنجز،  عندما ظهرت صعوبة.
  
\item
    احتفلوا بالتحسن الصغير: من "مساعدة جزئية" إلى "لمسة خفيفة" يستحق كلمة شكر وتشجيع.
  
\item
    ضعوا الدفتر في مكان ظاهر (قرب الكرسي أو على الثلاجة) ليكون تذكيراً دائماً.
  
\end{itemize}

\section*{خلاصة دافئة}

دفتر المتابعة المنزلي ليس إجراءً معقداً، بل \textbf{أداة بسيطة} تجعل الرحلة أوضح وأسهل: - يسجل التمارين والجهد والألم، - يوثق السقوط بدقة لحماية المريض، - يبيّن التغير الوظيفي أسبوعاً بعد أسبوع، - ويختصر وقت الزيارة مع الأخصائي لصالح قرارات أدق وأسرع.

ضعوا الدفتر في مكان تحبونه، اكتبوا فيه بصدق، واحتفلوا بكل سطر يثبت أنكم تمضون خطوة للأمام، ولو ببطء. نحن معكم في كل زيارة، والدفتر صديقكم بين الزيارات.
