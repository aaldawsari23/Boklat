\chapter{كيف نميّز الطبيعي من الخطر؟}
\label{ch:07}

\section*{من الميدان}

في إحدى زياراتي، كنت أتابع أبا خالد بعد شهر من بدء برنامج المشي الخفيف. قال لي وهو يمسك فخذه: " احس فيه شد بسيط من أمس". سألته: "كم درجة الألم من 10؟" أجاب: "ثلاثة، ويخف إذا تحركت". ابتسمت وقلت: "هذا ألم تدريب طبيعي. لو كان 7 أو جاء فجأة مع تورم، هنا نتوقف". بعد يومين كتب في دفتره: "الألم خف كثير". هذه اللحظات تؤكد لي أن \textbf{الفارق بين الألم الآمن والإنذار الأحمر بسيط حين نعرفه}.

\section*{لماذا نهتم بالتمييز؟}
\begin{itemize}
\item
    الخوف من الألم يجعل بعض الأسر توقف التمارين مبكرًا، فيخسر كبير السن مكاسب الحركة.
  
\item
    تجاهل الألم الخطر قد يؤدي لإصابة حقيقية أو نوبة قلبية لا سمح الله.
  
\item
    التمييز الواضح يمنح الأسرة ثقة: نستمر عندما يكون الألم طبيعيًا، ونتصرف بسرعة عند العلامات التحذيرية.
  
\end{itemize}

\textbf{قاعدة عملية:} الألم المقبول غالبًا \textbf{خفيف، تدريجي، وينحسر خلال 24-72 ساعة}. الألم الخطر يكون \textbf{حادًا، مفاجئًا، أو مصحوبًا بتورم/أعراض عامة}.

\section*{1. ما هو الألم الطبيعي مع الجهد؟}

\subsection*{1.1 شد عضلي متوقع (DOMS)}
\begin{itemize}
\item
    يظهر بعد نشاط جديد أو أشد، يبدأ خلال 12-24 ساعة ويبلغ ذروته خلال 48-72 ساعة ثم يخف وحده [1].}
  
\item
    يكون منتشرًا في العضلة التي عملت، يشبه الشد أو الحرقان الخفيف.
  
\item
    كبار السن ليسوا بالضرورة أكثر عرضة للألم المزعج من الشباب؛ أحيانًا يكون الألم أقل بسبب اختلاف استجابة العضلات [1].}
  
\end{itemize}

\subsection*{1.2 خصائص الألم المقبول أثناء التمرين}
\begin{itemize}
\item
    شد خفيف إلى متوسط (حتى 4/10) يتدرج ببطء [3].}
  
\item
    مرتبط بالعضلات المستخدمة، وليس نقطة مفصل واحدة محددة.
  
\item
    لا يصاحبه تورم سريع، كدمة واضحة، أو فقدان وظيفة مفاجئ.
  
\item
    يهدأ مع الحركة الخفيفة والراحة النسبية خلال 24-72 ساعة [1][}3].}
  
\end{itemize}

\subsection*{1.3 علامات الإنذار الأحمر}
\begin{itemize}
\item
    ألم حاد يشبه الطعنة أثناء التمرين، خصوصًا مع صوت «فرقعة» أو فقدان قدرة تحريك المفصل [1].}
  
\item
    تورم سريع، كدمة تظهر خلال ساعات، أو عرج يمنع تحميل الوزن.
  
\item
    أعراض طارئة: ألم صدر شديد، ضيق نفس حاد، دوار أو إغماء.
  
\end{itemize}

 \textbf{تنبيه سلامة مهم}

\textbf{🛑 توقف فورًا} إذا ظهر ألم حاد في المفصل أو العضلة مع تورم/كدمة، أو إذا حدث ألم صدر أو ضيق نفس شديد أو دوار/إغماء. \textbf{ثم:} اجلس المريض، هدئه، واطلب تقييمًا طبيًا عاجلًا. لا تعاود نفس التمرين قبل مراجعة مختص.

\section*{2. كيف نراقب الألم والجهد؟}

\subsection*{2.1 مقياس الألم العددي (0-10)}
\begin{itemize}
\item
    قبل التمرين: اسأل "كم الألم الآن؟" وسجّل الرقم.
  
\item
    أثناء التمرين: نريد بقاءه في نطاق 3-4/10 للغالبية [3].}
  
\item
    بعد 24 ساعة: إذا عاد الألم كما كان أو خف، فهذا مطمئن. إذا زاد إلى 6-7 واستمر، خفف الشدة في الجلسة التالية.
  
\end{itemize}

\subsection*{2.2 مقياس الجهد (RPE) واختبار الحديث}
\begin{itemize}
\item
    اطلب من كبيرك تقييم الجهد من 0-10. الشدة الآمنة الشائعة 5-6/10 حيث يمكن التحدث بجمل كاملة لكن الغناء يصبح صعبًا [3].}
  
\item
    إذا تعذر قول أكثر من كلمات متقطعة بسبب اللهاث، فالشدة عالية ويجب التخفيف.
  
\end{itemize}

\subsection*{2.3 دفتر متابعة الألم}
\begin{itemize}
\item
    خصص دفترًا بسيطًا بثلاث خانات: \textbf{اليوم/التمرين -- الألم قبل/بعد/بعد 24 ساعة -- ملاحظات}.
  
\item
    يساعد في رصد الأنشطة التي تسبب ألمًا مفرطًا، ويعطي الأخصائي صورة أوضح لتعديل الخطة [1].}
  
\end{itemize}

\section*{3. التدرج والراحة: وصفة الوقاية من الألم المفرط}

\subsection*{3.1 ابدأ قليلًا وتقدم ببطء}
\begin{itemize}
\item
    الزيادة المفاجئة في التكرارات أو مدة المشي ترفع احتمال الألم والإجهاد [1].}
  
\item
    قاعدة عملية: \textbf{زيادة 10\% أسبوعيًا} في الوقت أو التكرارات تكفي لمعظم الحالات.
  
\end{itemize}

\subsection*{3.2 قاعدة 24-72 ساعة}
\begin{itemize}
\item
    إذا خف الألم خلال يومين → يمكنك الاستمرار بنفس الشدة أو زيادتها قليلًا.
  
\item
    إذا استمر الألم المتوسط أكثر من 72 ساعة أو ازداد → خفف الشدة في الجلسة المقبلة أو أضف يوم راحة.
  
\end{itemize}

\subsection*{3.3 تعديل التمرين عند ظهور الألم}
\begin{itemize}
\item
    أوقف التمرين الذي سبب ألمًا حادًا، وجرب بديلًا أخف لنفس العضلة في زيارة لاحقة.
  
\item
    استمر بالحركة الخفيفة (مشي في المكان، تدوير الكتفين) بدل الراحة التامة لتسريع التعافي [1].}
  
\end{itemize}

\subsection*{3.4 "اختبار ساعتين" لخشونة المفاصل}
\begin{itemize}
\item
    إذا كان الألم بعد التمرين بأسوأ من قبله بساعتين، خفف الشدة في الجلسة التالية [7].}
  
\item
    استخدم كراسي ثابتة أو تمارين مائية أو تمارين كرسي لتقليل الحمل على الركبة/الورك [2][}4].}
  
\end{itemize}

\section*{4. حالات مزمنة شائعة: كيف نضبط الشدة بأمان؟}

\subsection*{4.1 خشونة المفاصل}
\begin{itemize}
\item
    التمارين الهوائية الخفيفة وتقوية العضلات تحسن الألم والوظيفة إذا راقبنا الشدة [2][}4].}
  
\item
    اختيارات مريحة: مشي قصير على أرض مستوية، جلوس ووقوف من كرسي مرتفع، تدوير مفصل الكتف أو الورك بدون أثقال كبيرة.
  
\item
    علامات التخفيف: تورم ملحوظ، ألم حاد في مفصل محدد، أو شعور "احتكاك" مؤلم.
  
\end{itemize}

\subsection*{4.2 هشاشة العظام}
\begin{itemize}
\item
    التمرين يحافظ على العظم والتوازن، لكن تجنب الانحناء الشديد للأمام أو الالتفاف العنيف للجذع [2].}
  
\item
    ركز على التوازن قرب دعم ثابت (كرسي/جدار)، وحركات نطاق حركة لطيفة.
  
\end{itemize}

\subsection*{4.3 أمراض القلب أو الرئة المستقرة}
\begin{itemize}
\item
    النشاط المعتدل يحسن التحمل وجودة الحياة [6].}
  
\item
    اعتمد اختبار الحديث وRPE 5-6/10.
  
\item
    
\item
    \textbf{🛑 توقف} إذا ظهر ألم صدر، ضيق نفس شديد، تعرق بارد، أو دوار.
  
\end{itemize}

\section*{5. أسئلة أسمعها كثيرًا}

\textbf{س: هل أوقف كل التمارين عند أي ألم؟}\hfill\break
ج: لا. الألم الخفيف المنتشر في العضلات بعد جهد جديد طبيعي. أوقف التمرين إذا كان الألم حادًا، مفصليًا، أو جاء مع تورم/كدمة أو أعراض عامة.

\textbf{س: كيف أشرح لكبير السن أن بعض الألم طبيعي؟}\hfill\break
ج: استخدم مقياس 0-10. قل له: "إذا كان الألم 3-4 ويخف مع الحركة فهذا يعني أن العضلة تتكيف". وأضف الحدود: "إذا صار 7 أو مع تورم، نتوقف ونكلم المختص".

\textbf{س: ماذا أفعل إذا استمر الألم المتوسط لأكثر من 3 أيام؟}\hfill\break
ج: خفف الشدة (وقت أقل أو تكرارات أقل)، أضف يوم راحة، ثم جرّب بجرعة أخف. إذا لم يتحسن أو ساء، اطلب تقييم أخصائي علاج طبيعي.

\textbf{س: هل يمكن التمرين مع خشونة الركبة أو الورك؟}\hfill\break
ج: نعم، بل هو جزء أساسي من العلاج. اختر تمارين منخفضة الحمل، راقب الألم خلال 24 ساعة. إذا زاد بوضوح بعد الجلسة، قلل الشدة واستشر مختصًا عند الحاجة.

\section*{خلاصة العمل للأسرة}
\begin{itemize}
\item
    الألم الخفيف المتدرج طبيعي مع الجهد الجديد، بينما الألم الحاد المفاجئ أو المصحوب بتورم/أعراض عامة هو إنذار توقف.
  
\item
    استخدم أدوات بسيطة: مقياس الألم (0-10)، مقياس الجهد (RPE)، واختبار الحديث لضبط الشدة بأمان.
  
\item
    التدرج البطيء (≈10\% أسبوعيًا) والراحة المخطط لها يمنعان معظم الإصابات والألم المفرط.
  
\item
    الحالات المزمنة (خشونة، هشاشة، أمراض قلب/رئة مستقرة) تحتاج تعديل وضعيات وحركات لتقليل المخاطر دون إيقاف النشاط.
  
\item
    دفتر متابعة الألم يمنح الأسرة والأخصائي خريطة طريق دقيقة لتعديل البرنامج والبقاء مطمئنين.
  
\end{itemize}

\section*{المراجع}

[1] Fernandes, J. F. T., Vieira, A., \& Silva, P. (2025). Advancing age is not associated with greater exercise-induced muscle damage: A systematic review and meta-analysis. \emph{Journal of Aging and Physical Activity}, 33(6), 606-624. https://doi.org/10.1123/japa.2024-0165}

[2] Skou, S. T., Roos, E. M., Laursen, M. B., et al.~(2018). Physical activity and exercise therapy benefit more than just symptoms and impairments in people with hip and knee osteoarthritis. \emph{JOSPT}, 48(6), 439-447. https://doi.org/10.2519/jospt.2018.7877}

[3] Geneen, L. J., Moore, R. A., Clarke, C., Martin, D., Colvin, L. A., \& Smith, B. H. (2017). Physical activity and exercise for chronic pain in adults: An analysis of Cochrane reviews. \emph{Cochrane Database of Systematic Reviews}, 2017(1). https://doi.org/10.1002/14651858.CD011279.pub2}

[4] Bannuru, R. R., Osani, M. C., Vaysbrot, E. E., et al.~(2019). OARSI guidelines for the non-surgical management of knee, hip, and polyarticular osteoarthritis. \emph{Osteoarthritis and Cartilage}, 27(11), 1578-1589. https://doi.org/10.1016/j.joca.2019.06.011}

[5] American Geriatrics Society Panel. (2001). Exercise prescription for older adults with osteoarthritis pain: Consensus guidelines. \emph{Journal of the American Geriatrics Society}, 49(6), 808-823. https://doi.org/10.1046/j.1532-5415.2001.00496.x}

[6] Bull, F. C., Al-Ansari, S. S., Biddle, S., et al.~(2020). World Health Organization 2020 guidelines on physical activity and sedentary behaviour. \emph{British Journal of Sports Medicine}, 54(24), 1451-1462. https://doi.org/10.1136/bjsports-2020-102955}

[7] American College of Sports Medicine. (2014). \emph{Exercising with osteoarthritis} (Exercise is Medicine patient handout).}
