\chapter{المشي والأجهزة المساعدة (العصا، المشاية، الكرسي المتحرك)}
\label{ch:05}

\section*{لماذا هذا الفصل مهم للأسرة؟}

في زيارات الرعاية المنزلية، أرى مشهداً يتكرر كثيرًا:
\begin{itemize}
\item
    كبير سن يمشي بعصا غير مناسبة (قصيرة جدًا أو طويلة جدًا).
  
\item
    مشاية قديمة مكسورة العجلات يستخدمها شخص توازنه ضعيف جداً.
  
\item
    كرسي متحرك يتحوّل من وسيلة مساعدة مؤقتة إلى "سجن متحرك" يمنع صاحبَه من أي محاولة للمشي.
  
\end{itemize}

أحيانًا يكون سقوط واحد سببه عصا غير مضبوطة، أو مشاية تُستخدم بطريقة خاطئة، أو إصرار الأسرة على أن كبيرهم "لسه يقدر يمشي بدون شيء" رغم أن الواقع يقول العكس.

هذا الفصل لا يهدف إلى إقناع كل أسرة أن تشتري عصا أو مشاية؛ الهدف أن:
\begin{itemize}
\item
    \textbf{تعرف متى} يكون الجهاز المساعد للمشي ضرورة، لا رفاهية.
  
\item
    \textbf{تفهم أي نوع} من الأجهزة يناسب حالة كبير السن في البيت.
  
\item
    \textbf{تتعلّم كيف} تضبط ارتفاع العصا أو المشاية بطريقة صحيحة.
  
\item
    \textbf{تتجنّب الأخطاء الشائعة} التي قد تجعل الجهاز نفسه سببًا في السقوط بدلًا من الوقاية منه [1][}6][}9].}
  
\end{itemize}

\section*{1. متى نبدأ نفكر في جهاز يساعد على المشي؟}

ليس كل كبير سن يحتاج عصا أو مشاية. ولكن هناك "علامات تحذير" إذا رأيناها، يجب أن نأخذ موضوع الأجهزة المساعدة بجدية [2][}4][}9]:}

\subsection*{1.1 علامات من مشية كبير السن نفسه}

اسأل نفسك هذه الأسئلة عن كبيرك في البيت:
\begin{itemize}
\item
    هل يتشبّث بالجدران أو الأثاث وهو يمشي في البيت؟
  
\item
    هل يمشي وهو منحنٍ للأمام بشكل زائد؟
  
\item
    هل خطواته قصيرة جدًا ويجرّ قدميه بدل رفعهما؟
  
\item
    هل تعرّض لسقطة في الأشهر الماضية؟
  
\item
    هل يقول لك: "أخاف أطيح"، أو "أحس رجولي ما تثبت"؟
  
\end{itemize}

إذا كانت الإجابة نعم على أكثر من نقطة، فهذه إشارة أن توازنه لم يعد كما كان، وأن التفكير في عصا أو مشاية قد يكون خطوة حماية، لا إهانة [1][}2][}10].}

\subsection*{1.2 علامات من قوة العضلات والقدرة على التحمل}
\begin{itemize}
\item
    صعوبة القيام من الكرسي بدون مساعدة.
  
\item
    الحاجة للاستناد على الطاولة أو شخص آخر عند الوقوف.
  
\item
    تعب شديد بعد مشي مسافة قصيرة داخل البيت.\\
  هذه العلامات غالبًا ترتبط بضعف عضلات الفخذين والساقين (Sarcopenia) وقلة الحركة [8]. في هذه الحالات، قد يجتمع:}
  
\end{itemize}

ضعف عضلي + توازن غير مستقر + بيئة منزلية فيها مخاطر = \textbf{خطر سقوط مرتفع} [2][}4][}8].}

\subsection*{1.3 عوامل مرضية تزيد أهمية الأجهزة المساعدة}

هناك حالات يكون فيها استخدام جهاز للمشي جزءًا من الخطة العلاجية، وليس خيارًا ثانويًا:
\begin{itemize}
\item
    بعد كسر في الورك أو الحوض أو كسر أسفل الساق [7].}
  
\item
    بعد جلطة دماغية سببت ضعفًا في جهة واحدة من الجسم [4][}11].}
  
\item
    في حالات باركنسون واضطرابات المشي المزمنة.
  
\item
    بعد عملية استبدال مفصل الركبة أو الورك.
  
\end{itemize}

في هذه الحالات، الدراسات تبيّن أن استخدام الجهاز المناسب في الوقت المناسب يساعد على:
\begin{itemize}
\item
    استعادة القدرة على المشي بأمان.
  
\item
    تقليل السقوط خلال فترة التعافي.
  
\item
    تشجيع المريض على الحركة بدل الخوف والجلوس المستمر [1][}7][}11].}
  
\end{itemize}

\section*{2. فكرة أساسية: الجهاز ليس "علامة عجز" بل أداة لاستعادة الاستقلال}

كثير من كبار السن يرفضون فكرة العصا أو المشاية؛ لأنهم يربطونها بصورة "العجز".

دور الأسرة هنا حاسم:
\begin{itemize}
\item
    بدل ما نقول:\\
  \textgreater{ "خلاص كبرت وتحتاج عصا"\\
  نقول:\\
  \textgreater{} "نبي نرتّب لك جهاز يساعدك تمشي بأمان أكثر، عشان تبقى تتحرك وتزور وتروح وتجي وأنت ثابت."}
  
\end{itemize}

الدراسات في كبار السن توضح أن قبولهم للجهاز المساعد للمشي يزيد عندما:
\begin{itemize}
\item
    يشعرون أن القرار تشاركي، ليس مفروضًا عليهم [9][}13].}
  
\item
    يفهمون كيف سيحميهم من السقوط.
  
\item
    لا يشعرون أنه "سجن" يمنعهم من الحركة، بل أداة تشجّعهم على الخروج من الغرفة والصالة [1][}9][}13].}
  
\end{itemize}

أذكر مريضاً في الثمانين رفض المشاية تماماً أول مرة عرضتها عليه. قال لي: "يا ولدي، أنا لم أصل لهذه المرحلة بعد."

لم أجادله. تركت المشاية في زاوية الغرفة وقلت له: "تبقى هنا، لو احتجتها يوماً."

بعد أسبوعين، في زيارتي التالية، وجدته يمشي بها في الصالة. ابنه قال لي: "أبي صار يستخدمها ليلاً حين يذهب للحمام."

الدرس هنا بسيط: القبول يحتاج وقتاً أحياناً، والإجبار يأتي بنتائج عكسية.

\section*{3. أنواع الأجهزة المساعدة للمشي (نظرة شاملة)}

سنتناول في هذا الفصل أكثر الأجهزة شيوعًا في بيوتنا:

\begin{enumerate}
\def\labelenumi{\arabic{enumi}.}
\item
    العصا (Cane) بأنواعها.
  
\item
    المشاية (Walker) بأنواعها.
  
\item
    الكرسي المتحرك (Wheelchair).
  
\end{enumerate}

لن نتوسع في العكازات الرياضية (Crutches) لأنها تُستخدم غالبًا لفترات قصيرة بعد إصابات معينة.

\section*{4. العصا (Cane)}

\subsection*{4.1 متى تكون العصا خيارًا مناسبًا؟}

العصا تناسب عادة:
\begin{itemize}
\item
    كبار السن الذين:

  \begin{itemize}
  \item
        يستطيعون المشي بدون مساعدة شخص، لكن توازنهم ليس مثالياً.
    
  \item
        تعرضوا لسقوط واحد أو أكثر، ويحتاجون "نقطة دعم" إضافية.
    
  \end{itemize}
\item
    حالات ألم مفصل واحد في الطرف السفلي (مثلاً: خشونة ركبة أو مفصل ورك) مع قدرة معقولة على التحكم والتوازن [1][}4][}9][}12].}
  
\end{itemize}

لا تناسب العصا وحدها:
\begin{itemize}
\item
    شخص يتمايل يمينًا ويسارًا بشكل واضح.
  
\item
    من لا يستطيع رفع قدميه جيدًا.
  
\item
    من يحتاج يدين الاثنتين ليتوازن. هؤلاء غالبًا يحتاجون مشاية، لا عصا [6][}9][}12].}
  
\end{itemize}

\subsection*{4.2 أنواع العصا الشائعة}
\begin{itemize}
\item
    \textbf{العصا العادية (مقبض واحد):}\hfill\break
  مناسبة لمن يحتاج دعمًا بسيطًا فقط.
  
\item
    \textbf{عصا رباعية القاعدة (Quad Cane):}\hfill\break
  لها قاعدة بأربع نقاط تماس مع الأرض، تعطي ثباتًا أعلى، لكنها أثقل قليلًا وقد تتعثر في السجاد أو العتبات إن لم تُستخدم صح [1][}9][}12].}
  
\end{itemize}

\subsection*{4.3 كيف نضبط ارتفاع العصا بطريقة صحيحة؟}

طريقة عامة مدعومة في المراجع الإكلينيكية [6][}9][}12]:}

\begin{enumerate}
\def\labelenumi{\arabic{enumi}.}
\item
    نطلب من كبير السن أن يقف منتصبًا قدر استطاعته، وهو \textbf{مرتدي الحذاء} الذي يمشي به غالبًا.\\

\item
    نضع العصا بجانبه من الجهة \textbf{المقابلة للرجل المصابة} (مثلاً: لو الركبة اليمنى تؤلمه، توضع العصا في اليد اليسرى).\\

\item
    نضبط طول العصا بحيث:

  \begin{itemize}
  \item
        يكون مقبض العصا عند مستوى عظمة رسغ اليد تقريبًا.
    
  \item
        عندما يمسك المقبض، يكون \textbf{الكوع مثنيًا حوالي 20--30 درجة} (انحناء بسيط مريح).
    
  \end{itemize}
\end{enumerate}

\textbf{لو العصا أقصر من اللازم:}
\begin{itemize}
\item
    ينحني كبير السن للأمام.
  
\item
    يزداد الضغط على الظهر والكتف.
  
\item
    يقلّ الأمان في المشي.
  
\end{itemize}

\textbf{لو العصا أطول من اللازم:}
\begin{itemize}
\item
    يرتفع الكتف، ويتوتر العنق.
  
\item
    تقل فعالية العصا في دعم التوازن.
  
\end{itemize}

\subsection*{4.4 كيف نستخدم العصا أثناء المشي؟}

قاعدة بسيطة تُعلّم لكبير السن وللمرافق:
\begin{itemize}
\item
    العصا تكون في \textbf{اليد العكسية} للطرف المؤلم أو الأضعف.\\

\item
    تسير العصا مع \textbf{الرجل المصابة} في نفس الوقت:

  \begin{itemize}
  \item
        خطوة بالعصا + الرجل المؤلمة.
    
  \item
        ثم خطوة بالرجل السليمة.
    
  \end{itemize}
\end{itemize}

هذا النمط يساعد على نقل جزء من الوزن من المفصل المؤلم إلى العصا [9][}12].}

\subsection*{4.5 أخطاء شائعة مع العصا}

من الواقع اليومي:
\begin{itemize}
\item
    استخدام العصا في اليد \textbf{نفسها} لجهة الركبة المؤلمة، وهذا يقلل فائدتها ويزيد اللخبطة.
  
\item
    شراء عصا جاهزة دون ضبط ارتفاعها حسب طول صاحبها.
  
\item
    استخدام العصا "شكلًا فقط" مع الاستمرار في التشبث بالجدران أو شخص آخر.
  
\item
    وضع طرف العصا بعيدًا جدًا للأمام أو للجانب، فيصير فيها نوع من "الشد" بدل الدعم.
  
\end{itemize}

القاعدة الذهبية:

\textbf{العصا الجيدة = ارتفاع صحيح + جهة صحيحة + نمط مشي صحيح.}

\section*{5. المشاية (Walker)}

\subsection*{5.1 متى نحتاج مشاية بدل العصا؟}

المشاية خيار منطقي عندما:
\begin{itemize}
\item
    \textbf{التوازن أضعف} من أن تكفيه عصا وحدها.
  
\item
    كبير السن يحتاج دعمًا بيديه الاثنتين، وليس يدًا واحدة.
  
\item
    بعد عمليات عظام أو جلطات دماغية في المرحلة المبكرة من إعادة التأهيل [1][}4][}6][}11].}
  
\end{itemize}

أنواع شائعة:

\begin{enumerate}
\def\labelenumi{\arabic{enumi}.}
\item
    \textbf{مشاية ثابتة بدون عجلات:}\hfill\break
  تعطي أعلى مستوى من الثبات، لكنها تحتاج قوة أكبر لرفعها خطوة بخطوة.
  
\item
    \textbf{مشاية بعجلتين أماميتين:}\hfill\break
  تسمح بدفع المشاية بدل رفعها، مناسبة لكبار السن الذين يستطيعون التحكم في سرعة المشي نوعًا ما.
  
\item
    \textbf{روليتور (Rollator -- أربع عجلات + فرامل + كرسي صغير):}\hfill\break
  شائع في الدول الغربية، يحتاج قدرة إدراكية حركية جيدة؛ لأن التحكم بالفرامل أساسي [6][}9][}12].}
  
\end{enumerate}

في بيئتنا المنزلية، الأكثر عمليًا لكبار السن غالبًا:
\begin{itemize}
\item
    مشاية ثابتة أو بعجلتين، حسب قوة العضلات والتوازن.
  
\end{itemize}

\subsection*{5.2 ضبط ارتفاع المشاية}

نفس فكرة العصا تقريبًا [6][}9][}12]:}

\begin{enumerate}
\def\labelenumi{\arabic{enumi}.}
\item
    يقف كبير السن داخل المشاية، يده على المقبضين.\\

\item
    يكون \textbf{الكوع مثنيًا حوالي 20--30 درجة}.\\

\item
    لو كان ينحني كثيرًا للأمام لإمساك المشاية، فهذا يعني أنها منخفضة جدًا.
  
\end{enumerate}

\subsection*{5.3 طريقة المشي بالمشاية (خطوات بسيطة)}

نشرح للأسرة طريقة تسلسلية:

\begin{enumerate}
\def\labelenumi{\arabic{enumi}.}
\item
    \textbf{تقديم المشاية قليلاً للأمام} (حوالي خطوة صغيرة).\\

\item
    \textbf{خطوة بالرجل الأضعف أو المؤلمة} إلى داخل إطار المشاية.\\

\item
    \textbf{ثم خطوة بالرجل الأقوى} للحاق بها.
  
\end{enumerate}

بهذا الترتيب، يكون الوزن مدعومًا بالمشاية عندما تتحرك الرجل الأضعف، مما يحميها [1][}6][}9].}

\subsection*{5.4 أخطاء شائعة مع المشاية}
\begin{itemize}
\item
    دفع المشاية بعيدًا جدًا للأمام، فيصبح الجسم خلفها ولا يمكن اللحاق بها بأمان.
  
\item
    الانحناء الشديد فوق المشاية، فيزيد ألم الظهر ويقل التوازن.
  
\item
    المشي بالمشاية على سجاد سميك أو عتبات عالية بدون تعديل بيئة البيت.
  
\item
    استخدام روليتور بأربع عجلات مع شخص لا يستطيع التحكم في الفرامل أو إدراك المخاطر جيدًا [1][}6][}9][}13].}
  
\end{itemize}

\section*{6. الكرسي المتحرك (Wheelchair)}

\subsection*{6.1 متى يكون الكرسي المتحرك ضرورة؟}
\begin{itemize}
\item
    بعد عمليات كبيرة (مثل كسور الحوض أو الورك) في المرحلة الأولى قبل السماح بالتحميل الكامل على الرجل [4][}7].}
  
\item
    في حالات ضعف شديد لا يسمح بالمشي لمسافات كافية داخل البيت حتى مع المشاية.
  
\item
    عند وجود مشاكل عصبية أو عضلية متقدمة تجعل الوقوف نفسه خطيرًا.
  
\end{itemize}

الهدف من الكرسي المتحرك في هذه الحالات:
\begin{itemize}
\item
    تمكين كبير السن من الخروج من غرفة النوم إلى الصالة.
  
\item
    الذهاب للحمام أو العيادات أو الحوش بأمان.
  
\item
    تجنب الجلوس المستمر في السرير (الذي يسبب تقرحات، وضعفًا أشد، ومضاعفات أخرى) [7].}
  
\end{itemize}

\subsection*{6.2 الخطر الخفي: متى يتحوّل الكرسي إلى "سجن"؟}

أحيانًا تبدأ القصة كحل مؤقت:

"خلونا نرتاح ونحطه على كرسي متحرك كم أسبوع."

ثم تتحول تدريجيًا إلى:

"خلاص تعودنا عليه في الكرسي."

الدراسات في تأهيل ما بعد كسور الورك تشير إلى أن الاستخدام المفرط للكرسي، دون برامج حركة وتمارين، مرتبط بنتائج أسوأ على المدى البعيد [1][}7][}8]. لذلك:}
\begin{itemize}
\item
    نستخدم الكرسي كأداة تسهيل للحركة، \textbf{مع خطة واضحة} لتمارين الوقوف والمشي متى ما كان ذلك طبيًا آمنًا.
  
\item
    الكرسي لا يلغي الحاجة لتمارين قوة وتوازن؛ بل يجعل نقل كبير السن إلى مكان التمرين أسهل.
  
\end{itemize}

\subsection*{6.3 اعتبارات أساسية عند استخدام الكرسي في البيت}
\begin{itemize}
\item
    التأكد من وجود \textbf{فرامل تعمل جيدًا} واستعمالها قبل نقل المريض من الكرسي إلى السرير أو العكس.\\

\item
    وجود \textbf{مسند قدمين} مضبوط؛ عدم ترك القدمين تتدلّى؛ لأن هذا يعرضهما للضرب في العتبات والأبواب.\\

\item
    وضع \textbf{وسادة مناسبة} لتخفيف الضغط على منطقة العجز، خصوصاً في الاستخدام اليومي الطويل.\\

\item
    الانتباه لعرض الكرسي: ضيّق جدًا يسبب احتكاكاً؛ واسع جدًا يقلل الثبات داخل الكرسي.
  
\end{itemize}

\section*{7. كيف نختار الجهاز المناسب؟ (دليل مبسط للأسرة)}

يمكن للأسرة أن تستخدم هذه الأسئلة كـ "خريطة قرار" أولية، مع التأكيد أن رأي الطبيب وأخصائي العلاج الطبيعي هو الأساس [1][}4][}6][}9]:}

\begin{enumerate}
\def\labelenumi{\arabic{enumi}.}
\item
    \textbf{هل يستطيع كبير السن المشي دون أن يمسك أحدًا؟}

  \begin{itemize}
  \item
        نعم، لكن توازنه متوسط → غالبًا \textbf{عصا} (مع ضبط جيد وتعليم صحيح).
    
  \item
        لا، يحتاج دعمًا متكررًا باليدين أو بالجدران → نفكر في \textbf{مشاية}.
    
  \end{itemize}
\item
    \textbf{هل يستطيع استخدام اليدين جيدًا؟}

  \begin{itemize}
  \item
        نعم → العصا أو المشاية خيار ممكن.\\

  \item
        لا (مثلاً: ضعف في اليدين أو ألم شديد في الكتفين) → نحتاج تقييمًا أدق، وأحيانًا يكون الكرسي المتحرك أو حلول أخرى أنسب.
    
  \end{itemize}
\item
    \textbf{هل المشكلة الأساسية ألم في مفصل واحد، أم توازن عام وضعف عضلات؟}

  \begin{itemize}
  \item
        ألم وحيد في مفصل (مثلاً الركبة اليمنى) مع توازن مقبول → العصا غالبًا كافية.\\

  \item
        ضعف عام في العضلات، تمايل واضح، تاريخ سقوط متكرر → المشاية أكثر أمانًا [1][}4][}8][}11].}
    
  \end{itemize}
\item
    \textbf{هل الهدف الأساسي هو المشي داخل البيت، أم التنقل لمسافات أكبر (مثلاً خارج البيت، للمستشفى أو الحوش)؟}

  \begin{itemize}
  \item
        داخل البيت فقط، مع مسافات قصيرة → عصا أو مشاية حسب الحالة.\\

  \item
        لمسافات طويلة، مع ضعف في القدرة على التحمل → يمكن أن نجمع بين:

    \begin{itemize}
    \item
            تدريب على المشي لمسافات قصيرة بعصا/مشاية.
      
    \item
            واستخدام كرسي متحرك في المسافات الأطول.
      
    \end{itemize}
  \end{itemize}
\end{enumerate}

\section*{8. نصائح عملية للأسر من واقع الزيارات المنزلية}

من التجربة اليومية، هذه نقاط صغيرة لكنها فارقة:

\begin{enumerate}
\def\labelenumi{\arabic{enumi}.}
\item
    \textbf{لا تشتري الجهاز من نفسك ثم تحاول "إقناع" كبير السن به بعدين.}\hfill\break
  الأفضل تشاركه من البداية في القرار، وتسمع مخاوفه، وتشرح له فوائد الجهاز.
  
\item
    \textbf{لا تسيء استخدام الأجهزة كعقاب أو ضغط نفسي:}\hfill\break
  مثل: "إذا ما سمعت الكلام بنحطك على كرسي متحرك."\\
  هذه الجملة تجعل الكرسي رمزًا للعقاب، بدل أن يكون وسيلة مساعدة.
  
\item
    \textbf{عدّل البيت مع الجهاز، لا الجهاز وحده:}\hfill\break
  عصا + سجاد قابل للانزلاق + إضاءة ضعيفة = الخطر ما زال عاليًا.\\
  جهاز المشي يحتاج بيئة آمنة حوله ليعمل بفعالية.
  
\item
    \textbf{راجع استخدام الجهاز بعد شهر أو شهرين:}

  \begin{itemize}
  \item
        هل تحسّن المشي أم ساء؟\\

  \item
        هل كثر الجلوس على الكرسي المتحرك بدل المشي؟\\

  \item
        هل ظهرت آلام جديدة في الكتف أو الظهر من طريقة استخدام العصا؟
    
  \end{itemize}
\item
    \textbf{استثمر زيارة واحدة مع أخصائي علاج طبيعي للتعليم والتعديل:}\hfill\break
  جلسة واحدة صحيحة لتعليم استخدام العصا أو المشاية قد تساوي عشرات المحاولات العشوائية في البيت [1][}6][}9][}11][}12].}
  
\end{enumerate}

\section*{9. أسئلة شائعة من الأسر (FAQ مختصر)}

\subsection*{س: "أخاف لو استخدم عصا/مشاية يعتمد عليها ويكسل عن المشي!"}

ج: الخطر الحقيقي ليس في العصا نفسها، بل في \textbf{طريقة استخدامها}. إذا استخدمت كجزء من خطة:
\begin{itemize}
\item
    تمارين بسيطة للقوة والتوازن.
  
\item
    تشجيع على المشي اليومي لمسافات مناسبة.
  
\end{itemize}

فهي تساعد على زيادة الحركة بدل تقليلها [1][}8][}9][}11][}13].}

\subsection*{س: "هل الأفضل أبدأ بعصا ولا أدخل مباشرة في مشاية؟"}

ج: يعتمد على:
\begin{itemize}
\item
    مستوى التوازن.
  
\item
    قوة العضلات.
  
\item
    التاريخ السابق للسقوط.
  
\end{itemize}

الأبحاث تبيّن أن مطابقة الجهاز لدرجة العجز أفضل من التقليل أو المبالغة [1][}4][}6][}11].\\
في الشك، زيارة تقييم واحدة عند أخصائي علاج طبيعي تختصر عليك التخمين.}

\subsection*{س: "كيف أعرف أن الوقت حان للتفكير في كرسي متحرك؟"}

ج: عندما:
\begin{itemize}
\item
    يصبح المشي لمسافة قصيرة داخل البيت مرهقًا جدًا، حتى بالعصا أو المشاية.
  
\item
    أو يكون خطر السقوط عاليًا لدرجة تهدد سلامته في كل مرة يحاول الوقوف.
  
\end{itemize}

هنا الكرسي قد يكون حماية، بشرط أن يكون جزءًا من خطة شاملة للحركة، لا بديلًا دائمًا عنها [4][}7][}8].}

\section*{10. خلاصة الفصل: الجهاز الصحيح في الوقت الصحيح}

هذا الفصل يمكن تلخيصه في نقاط عملية:

\begin{enumerate}
\def\labelenumi{\arabic{enumi}.}
\item
    \textbf{لسنا ضد الأجهزة ولا معها بشكل مطلق؛ نحن مع "الجهاز المناسب للشخص المناسب في الوقت المناسب".}\hfill\break

\item
    العصا تناسب من يحتاج دعمًا بسيطًا إضافيًا، مع ضبط جيد للطول وطريقة المشي [6][}9][}12].\\
  }
  
\item
    المشاية تناسب من يحتاج دعمًا ثنائي اليدين وتوازنًا أعلى، خصوصًا بعد العمليات أو في حالات الضعف العضلي العام [1][}4][}6][}11].\\
  }
  
\item
    الكرسي المتحرك أداة مهمة للحفاظ على الحركة خارج السرير، لكن استخدامه يجب ألا يلغي برامج التمارين والمشي متى ما كان ذلك ممكنًا طبيًا [4][}7][}8].\\
  }
  
\item
    قبول كبير السن للجهاز يزيد عندما يفهم أنه وسيلة لحماية استقلاليته، وليس إعلان نهاية قدرته على المشي [9][}10][}13].\\
  }
  
\item
    \textbf{أهم خطوة}: لا نتعامل مع الجهاز كبديل عن تعديل البيت وتمارين التوازن؛ بل كجزء من منظومة حماية تشمل:

  \begin{itemize}
  \item
        بيئة منزلية آمنة.
    
  \item
        برنامج حركة بسيط يناسب عمره وحالته الصحية.
    
  \item
        متابعة طبية وعلاج طبيعي عند الحاجة.
    
  \end{itemize}
\end{enumerate}

الآن بعد أن عرفنا متى نستخدم العصا والمشاية والكرسي المتحرك وكيف نختارها، حان الوقت للسؤال الأهم: كيف نبني \textbf{برنامج تمارين منزلية} يقوّي العضلات ويحسّن التوازن، حتى يحتاج كبير السن لهذه الأجهزة أقل وأقل مع الوقت؟

\section*{المراجع}

[1] Bateni, H., \& Maki, B. E. (2005). Assistive devices for balance and mobility: benefits, demands, and adverse consequences. \emph{Archives of Physical Medicine and Rehabilitation}, 86(1), 134--145.}

[2] Deandrea, S., Lucenteforte, E., Bravi, F., Foschi, R., La Vecchia, C., \& Negri, E. (2010). Risk factors for falls in community-dwelling older people: a systematic review and meta-analysis. \emph{Epidemiology}, 21(5), 658--668.}

[3] Alshammari, S. A., Alhassan, A. M., Aldawsari, M. A., et al.~(2018). Falls among elderly and its relation with their health problems and surrounding environmental factors in Riyadh. \emph{Journal of Family and Community Medicine}, 25(1), 29--34.}

[4] Liu, H. H. (2009). Assessment and management of falls and gait disorders in older adults. \emph{Medical Clinics of North America}, 93(2), 355--369.}

[5] Laufer, Y. (2003). The effect of walking aids on balance and weight-bearing patterns of patients with hemiparesis in various stance positions. \emph{Physical Therapy}, 83(2), 112--122.}

[6] Van Hook, F. W., Demonbreun, D., \& Weiss, B. D. (2003). Ambulatory devices for chronic gait disorders in the elderly. \emph{American Family Physician}, 67(8), 1717--1724.}

[7] Handoll, H. H., Cameron, I. D., Mak, J. C., \& Finnegan, T. P. (2021). Multidisciplinary rehabilitation for older people with hip fractures. \emph{Cochrane Database of Systematic Reviews}, 2021(11).}

[8] Cruz-Jentoft, A. J., Bahat, G., Bauer, J., et al.~(2019). Sarcopenia: revised European consensus on definition and diagnosis. \emph{Age and Ageing}, 48(1), 16--31.}

[9] Bradley, S. M., \& Hernandez, C. R. (2011). Geriatric assistive devices. \emph{American Family Physician}, 84(4), 405--411.}

[10] Kumar, A., Carpenter, H., Morris, R., Iliffe, S., \& Kendrick, D. (2014). Which factors are associated with fear of falling in community-dwelling older people? \emph{Age and Ageing}, 43(1), 76--84.}

[11] Tyson, S. F., \& Rogerson, L. (2009). Assistive walking devices in nonambulant patients undergoing rehabilitation after stroke: the effects on functional mobility, walking impairments, and patients' opinion. \emph{Archives of Physical Medicine and Rehabilitation}, 90(3), 475--479.}

[12] Joyce, B. M., \& Kirby, R. L. (1991). Canes, crutches and walkers. \emph{American Family Physician}, 43(2), 535--542.}

[13] Gooberman-Hill, R., Ebrahim, S., \& GHAS. (2007). Making decisions about simple interventions: older people's use of walking aids. \emph{Age and Ageing}, 36(5), 569--573.}

[14] American Physical Therapy Association (APTA). (2019). \emph{Physical Therapist Practice and the Movement System}. APTA Guidelines.}
