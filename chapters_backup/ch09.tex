\chapter{كيف نحمي الحركة في البيت؟}
\label{ch:09}

\section*{من الميدان}

في إحدى زياراتي لبيت في أبها، وجدت الأسرة ترفع ذراع والدتهم التي أصيبت بجلطة قبل شهر، وهي تصرخ من الألم. كانوا يحاولون إلباسها عبر سحب الذراع المصابة لأنهم خائفون من تحريكها "خطأ". وفي بيت آخر بعد عملية استبدال مفصل الورك، كان الكرسي منخفضاً جداً، فكانت المريضة تضطر للانحناء العميق كل مرة تجلس أو تقوم. هذان المشهدان يتكرران كثيراً: \textbf{خوف من الحركة يقابله جهل بالاحتياطات البسيطة}. في هذا الفصل سنمشي معاً خطوة بخطوة لنوازن بين الحماية والتحفيز.

\subsection*{قبل أن نبدأ: ماذا سيفيدكم هذا الفصل؟}

هذا الفصل ليس بروتوكولاً طبياً أو جراحياً، ولن يغنيكم عن تعليمات الجراح أو فريق التأهيل. ما ستقرؤونه هنا هو "خريطة مبادئ" للأسرة: - أشياء يجب تجنّبها لأنها قد تؤذي. - أشياء نشجع عليها لأنها تساعد على استعادة الوظيفة. - إشارات حمراء تخبركم أن الوقت حان للعودة للطبيب أو الأخصائي.

الهدف أن لا تبقوا في حيرة بين "نخاف نحركه" و"نخاف نخليه بدون حركة".

\section*{1. بعد الجلطة الدماغية: حماية الطرف الضعيف دون تجميده}

\subsection*{لماذا؟}

الكتف بعد الجلطة عرضة للخلع الجزئي إذا تم سحب الذراع بعنف، وفي الوقت نفسه، إهمال الطرف يفاقم الضعف ويزيد الإهمال العصبي. التوازن هو كلمة السر.

\subsection*{ما الذي أفعله مع أسرتي؟}
\begin{itemize}
\item
     \textbf{أدعم الكتف والذراع المصابة دائماً} بوسادة عند الجلوس أو أثناء النقل.
  
\item
     \textbf{أذكر المريض باستخدام الطرف المصاب في مهام آمنة} مثل إمساك كوب فارغ أو لمس منشفة.
  
\item
    \textbf{لا أسحب المريض من الذراع المصابة} عند القيام أو اللبس.
  
\item
    \textbf{لا أترك الذراع تتدلى بلا دعم} أثناء الجلوس أو المشي.
  
\end{itemize}

\paragraph{من الميدان بعد الجلطة}}

بعد جلطة دماغية خفيفة، زرت مريضاً كان يستخدم يده الضعيفة قليلاً عندما يكون وحده، لكن أسرته كانت تمنعه من استخدامها خوفاً من أن "تلتوي" أو "تتعب".

جلسنا مع الأسرة وشرحنا أن اليد الضعيفة تحتاج استخداماً آمناً، لا حبسا كاملاً ولا حملاً ثقيلاً. بعد أسابيع من تشجيع بسيط على استخدامها في أشياء خفيفة (مسك الملعقة، مسح الطاولة)، تحسنت وظيفتها أكثر مما لو ظلت "محمية" زيادة عن اللزوم.

الرسالة: الحماية الزائدة قد تبطئ التعافي مثل الإهمال تماماً.

\subsection*{ ‍ تمرين منزلي: تحريك اليد المصابة بمهام بسيطة}

\textbf{الهدف:} تحفيز استخدام الطرف العلوي المصاب وتقليل الإهمال الحسي والحركي.

\textbf{مناسب لـ:} من يستطيعون الجلوس بدعم خفيف ولديهم حركة بسيطة في الأصابع أو المعصم.

\textbf{غير مناسب لـ:} وجود ألم كتف حاد أو خلع جزئي نشط لم يُعالج بعد.

\textbf{الأدوات:} كوب خفيف فارغ أو منشفة صغيرة، طاولة ووسادة.

\textbf{الوضعية البدائية:} - المريض جالس على كرسي بظهر، الذراع المصابة مدعومة على الطاولة بوسادة. - المرافق بجانب الطرف المصاب لدعم الكتف والمعصم. - الطاولة خالية من الفوضى مع إضاءة جيدة.

\textbf{الخطوات:} 1. \textbf{تهيئة:} ضع الكوب أو المنشفة أمام اليد المصابة على الطاولة. ذكّر المريض أن الهدف لمس الأداة دون ألم. 2. \textbf{استعداد:} ادعم الكوع المصاب بيدك بلطف، وتأكّد أن الكتف محايد وغير متدلي. 3. \textbf{محاولة:} اطلب منه فتح الأصابع ولمس الكوب. إن استطاع، يمسك الكوب ويرفعه سنتيمترات قليلة فقط. 4. \textbf{عودة:} ساعده على إعادة الكوب بهدوء، خذ استراحة 5--10 ثوان، ثم كرر.

 \textbf{تحذيرات السلامة:} - لا ترفع الذراع فوق مستوى الكتف إذا كان التحكم ضعيفاً. - لا تترك الكتف ينجرف للأمام أو يتدلى؛ الدعم أساسي. -  إذا ظهر تشنج شديد في الأصابع أو المعصم، أرح اليد وعدّل الوضعية.

\textbf{🛑 توقف فوراً إذا:} - شعر المريض بألم حاد في الكتف أو سمعتم صوت طقة. - ظهر تورم أو احمرار مفاجئ في الكتف أو الذراع. - فقد السيطرة وسقط الكوب بقوة. - \textbf{ثم:} أوقف التمرين، ادعم الذراع بوضع مريح، واستشر المعالج أو الطبيب، أو اتصل بالطوارئ (997) إذا كان الألم شديداً.

\textbf{التكرار المقترح:} 5--8 محاولات، مرة إلى مرتين يومياً حسب التحمل. زد الارتفاع أو زمن الإمساك تدريجياً.

\textbf{نصائح للمرافق:} ادعم الكتف طوال التمرين، عد بصوت مسموع للمساعدة على التركيز، واجعل المهمة ذات معنى (كوب قهوته المفضل).

\textbf{المرجع:} [6] Winstein CJ, et al.~(2016). \emph{Guidelines for Adult Stroke Rehabilitation and Recovery}.}

\section*{2. بعد استبدال الورك أو الركبة: حماية المفصل الصناعي مع الحفاظ على الحركة}

\subsection*{لماذا؟}

المفصل الصناعي يحتاج وقتاً ليستقر. ثني عميق أو دوران داخلي أو تقاطع الساقين قد يزيد خطر الخلع أو الألم. تعديل البيئة يقلل الحركات الخطرة ويشجع المشي الآمن.

\subsection*{احتياطات سريعة للأسرة}
\begin{itemize}
\item
     استخدم كراسٍ مرتفعة بمساند ذراع لتسهيل الجلوس والقيام دون ثني مفرط.
  
\item
     ثبّت مسكات حائطية قرب المرحاض والاستحمام، واستخدم مقعد حمام مرتفع إذا لزم.
  
\item
     شجع المشي بالمشاية أو العكازات حسب توجيه المعالج مع مراقبة التدرج.
  
\item
     لا تسمح بالجلوس على أرضية منخفضة أو كراسٍ رخوة وعميقة في الأسابيع الأولى.
  
\item
     لا تسمح بتقاطع الساقين أو دوران القدم المصابة للداخل عند الجلوس أو الالتفاف.
  
\end{itemize}

\subsection*{ إجراء منزلي: الجلوس والقيام من كرسي مرتفع بعد استبدال المفصل}

\textbf{الهدف:} تسهيل الجلوس والقيام بأمان دون ثني زائد للورك أو الركبة.

\textbf{مناسب لـ:} من سُمح لهم بالتحميل الجزئي أو الكامل باستخدام جهاز مساعد.

\textbf{غير مناسب لـ:} من لديهم ألم حاد أو دوار يمنع الوقوف بأمان.

\textbf{الأدوات:} كرسي مرتفع ثابت بمساند ذراع، مشاية أو عكازات حسب وصف المعالج.

\textbf{الوضعية البدائية:} - المريض واقف أمام الكرسي، ظهره نحو المقعد، والمشاية أمامه إذا يستخدمها. - المرافق يقف بجانب الطرف المصاب قليلاً للخلف لمراقبة التوازن. - أرضية خالية من السجاد المتعثر أو الأسلاك، والكرسي ملاصق للحائط إن أمكن.

\textbf{الخطوات:} 1. \textbf{تهيئة:} تأكد من ثبات الكرسي. ذكّر بقاعدة "القدمين للأمام دون تقاطع أو دوران داخلي". 2. \textbf{الجلوس:} قدم المصابة نصف خطوة للأمام، ارجع ببطء حتى يحس بالمقعد خلف ركبتيه، ضع اليدين على المساند ثم انزل بهدوء والظهر مستقيم. 3. \textbf{القيام:} قدم المصابة للأمام والسليمة للخلف قليلاً، ضع اليدين على المساند، ميل خفيف للأمام دون ثني حاد، ادفع عبر الذراعين والساق السليمة للوقوف، ثم أمسك المشاية قبل البدء بالمشي.

 \textbf{تحذيرات السلامة:} - لا تدع المريض يهبط سريعاً أو يجلس بقوة. - لا تستخدم مقعداً منخفضاً يجعل الركبة أعلى من الورك. -  إذا ظهر دوار أو عدم اتزان، أوقف الحركة وأعد التهيئة.

\textbf{🛑 توقف فوراً إذا:} - شعر بألم حاد أو طقة في الورك/الركبة. - حدث فقدان توازن كاد يؤدي إلى سقوط. - ظهر تورم سريع أو إحساس بخلع. - \textbf{ثم:} ساعده للجلوس بأمان، ثبّت المفصل بوضع مريح، واطلب استشارة طبية عاجلة أو اتصل بالطوارئ (997) إذا استمر الألم أو عدم الثبات.

\textbf{التكرار المقترح:} 3--5 مرات في اليوم ضمن الانتقالات اليومية، مع مراقبة التعب.

\textbf{نصائح للمرافق:} تأكد أن زاوية الورك 90 درجة على الأقل عند الجلوس، أضف وسادة إذا لزم، وذكّر بخطوات صغيرة عند الالتفاف لتجنب دوران الورك للداخل.

\textbf{المرجع:} [9] Sheth NP \& Foran JR. (2022). \emph{Activities After Total Hip Replacement}. AAOS OrthoInfo.}

\section*{3. بعد الكسور الشائعة: فهم تعليمات تحميل الوزن والحركة المبكرة الآمنة}

\subsection*{لماذا؟}

تحميل الوزن قبل الأوان قد يعيق الالتئام أو يسبب فشل التثبيت، لكن الخمول التام يزيد خطر الجلطات وقرح الفراش. الاتزان بين الحماية والحركة المسموح بها ضروري.

\subsection*{مبادئ للأسرة}
\begin{itemize}
\item
     اسأل الطبيب بوضوح: هل الوزن المسموح "عدم تحميل"، "تحميل جزئي"، أم "حسب التحمل"؟ طبق التعليمات حرفياً.
  
\item
     استخدم جهاز المشي أو العكازات حسب مرحلة التحميل واضبط ارتفاعها جيداً.
  
\item
     شجع تمارين بسيطة في السرير (تحريك الكاحل، شد عضلة الفخذ) إذا سمح الطبيب لتقليل الخمول.
  
\item
     لا تسمح بالوقوف الكامل على الطرف المصاب قبل الإذن.
  
\item
     لا تترك المريض طريح الفراش خوفاً من الأذى؛ الحركة الخفيفة المسموح بها أفضل.
  
\end{itemize}

\subsection*{ ‍ تمرين منزلي: تحريك الكاحل لمنع الجلطات بعد الكسر}

\textbf{الهدف:} تنشيط الدورة الدموية وتقليل تيبس الكاحل أثناء فترة عدم التحميل.

\textbf{مناسب لـ:} من يُسمح لهم بالحركة في السرير دون تحميل على الطرف المصاب.

\textbf{غير مناسب لـ:} ألم حاد في الكاحل أو كسر غير مستقر في أسفل الساق.

\textbf{الأدوات:} لا شيء، يمكن استخدام وسادة خفيفة تحت الساق للدعم.

\textbf{الوضعية البدائية:} - المريض مستلقٍ على ظهره، الساق المصابة ممدودة ومدعومة بوسادة لتقليل التورم. - المرافق بجانب الساق لمراقبة الألم وتشجيع العد.

\textbf{الخطوات:} 1. \textbf{تحضير:} تأكد من راحة المريض وثبات الجبس أو الدعامة، واشرح الهدف: "نحرك الكاحل لتحسين الدورة". 2. \textbf{ثني لأعلى ولأسفل:} اطلب سحب أصابع القدم نحو الرأس ببطء ثم دفعها كما لو يضغط على دواسة، مع إبقاء الركبة مستقيمة. 3. \textbf{دوائر خفيفة:} إذا مسموح ودون ألم، اطلب رسم دوائر صغيرة بالكاحل باتجاه عقارب الساعة ثم عكسها. 4. \textbf{راحة وتكرار:} استرح 5 ثوانٍ بين المجموعات.

 \textbf{تحذيرات السلامة:} - لا تدفع القدم بيدك بقوة؛ الحركة يجب أن تكون نشطة من المريض. - لا تستمر إذا زاد الألم أو انفتح الجبس. -  إذا لاحظت تورماً متزايداً أو سخونة موضعية، توقف وأبلغ الطبيب.

\textbf{🛑 توقف فوراً إذا:} - ظهر ألم حاد أو إحساس بفرقعة في الكاحل. - أصبح لون القدم أزرق/أحمر داكناً فجأة أو حدث انتفاخ سريع. - شعر المريض بضيق نفس مفاجئ أو ألم صدر أثناء التمرين. - \textbf{ثم:} أوقف التمرين، ارفع الساق بلطف، واتصل بالطبيب فوراً أو بالطوارئ (997) إذا ظهرت أعراض تنفسية.

\textbf{التكرار المقترح:} 10--15 حركة ثني كل ساعة يكون المريض مستيقظاً، و5 دوائر لكل اتجاه مرتين يومياً إذا سمح الطبيب ولم يوجد ألم.

\textbf{نصائح للمرافق:} ذكّر المريض بالتنفس أثناء الحركة، اضبط الوسادة تحت الساق ليكون الكعب حراً للحركة، ودون أي تغير في الألم أو التورم لإبلاغ الفريق الطبي.

\textbf{المرجع:} [10] American Academy of Orthopaedic Surgeons. (2021). \emph{Management of Hip Fractures in Older Adults}.}

\section*{إشارات حمراء بعد العمليات والكسور}

بغض النظر عن نوع العملية (استبدال مفصل، تثبيت كسر\ldots) انتبهوا لهذه العلامات: - ألم جديد وشديد يختلف عن "ألم العملية المعتاد". - تورم مفاجئ في الساق مع احمرار وحرارة موضعية. - ضيق نفس مفاجئ أو ألم في الصدر. - عدم القدرة على تحريك الطرف كما كان قبل يوم أو يومين.

عند ظهور هذه الأعراض، لا نستمر في التمارين ولا نحاول "نتحمّل" الألم؛ بل نتواصل مع الطبيب فوراً أو نذهب للطوارئ حسب الحالة.

\section*{4. متى أقول: لا يكفي المتابعة المنزلية، أحتاج أخصائي علاج طبيعي بشكل مكثف؟}
\begin{itemize}
\item
    إذا مرّت أسابيع بعد الجلطة أو العملية بدون أي تحسن واضح في الحركة أو التوازن.
  
\item
    إذا كان كبير السن لا يزال يعتمد على الأسرة في كل حركة بسيطة (القيام من الكرسي، المشي لباب الغرفة\ldots).
  
\item
    إذا لاحظتم أنه بدأ يفقد مهارات كان يؤديها بعد العملية ثم تراجع فيها.
  
\item
    إذا زاد الخوف من الحركة لدرجة أنه يرفض كل محاولة للقيام أو المشي.
  
\end{itemize}

في هذه الحالات، المتابعة مع أخصائي علاج طبيعي في العيادة أو مركز التأهيل ليست رفاهية، بل خطوة مهمة لاستعادة أكبر قدر ممكن من الاستقلال.

\section*{الخلاصة العملية للأسرة}
\begin{itemize}
\item
    حماية الطرف الضعيف بعد الجلطة مع تشجيع استخدامه الآمن يقلل مضاعفات الكتف ويحفز التعافي العصبي.
  
\item
    بعد استبدال المفصل، الكراسي المرتفعة ومسكات الجدران والمشاية المضبوطة تقلل خطر الخلع وتدعم المشي المبكر.
  
\item
    بعد الكسور، الالتزام الدقيق بتعليمات تحميل الوزن مع تمارين خفيفة في السرير يحمي العظم ويمنع مضاعفات الخمول.
  
\item
    سجّلوا التقدم، شجعوا المحاولات اليومية، واطلبوا إعادة التقييم عند غياب التحسن أو ظهور أي علامة خطر.
  
\end{itemize}

\textbf{رسالة طمأنة:} الخوف طبيعي، لكن بالاحتياطات البسيطة والتحفيز المتدرج، يمكن للأسرة تحويل البيت إلى مسرح تعافٍ آمن ومحترم لمن تحب.
