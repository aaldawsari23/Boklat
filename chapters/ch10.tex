\chapter{التعامل مع الرفض والعناد والخوف من الحركة}
\label{ch:10}

\subsection*{من الميدان: "ما عاد لي فائدة"}

في أحد البيوت، جلست أمامي سيدة في الثمانين، يداها في حجرها، تنظر للأرض. قالت لي: "يا ولدي، خلاص\ldots  ما عاد لي فائدة. خلّوني على الكرسي، تعبت من الكلام عن التمارين."

في نفس الجلسة، كان ابنها يقول: " هي عنيدة، ما تسمع الكلام."

الحقيقة أن ما رأيته لم يكن عناداً، بل خليطاً من الخوف، والتعب، والإحباط، وخسارة الإحساس بالقيمة. هذا الفصل يحاول أن يقرّبكم خطوة من فهم ما وراء "الرفض"، وكيف نحوله إلى مشاركة هادئة ومحترمة.

 \textbf{من تجاربي:} «مرة زرت أبا محمد في قرية قرب بيشة. كان يرفض المشي تماماً بعد سقطة قديمة. لم يكن عناداً، كان خوفاً من أن يحرج نفسه أمام أحفاده.»

\section*{المقدمة}

عندما تقول الأم «اتركوني، لا أستطيع»، أو يرفض الأب أن ينهض من الكرسي رغم أننا نعرف أنه قادر، نشعر بالعجز. بعد عشر سنوات من الزيارات المنزلية، تعلمت أن الرفض غالباً يخفي خوفاً أو إحراجاً أو ألماً، لا مجرد مزاج سيئ. هذا الفصل هو صندوق أدوات عملي للأسرة: كيف نفهم دوافع الرفض، وكيف نتحدث مع كبيرنا بأسلوب يحفظ كرامته، وكيف نعيده للحركة خطوة بخطوة دون صدام.

سأشاركك عبارات مجرّبة، وخطوات تدرّج بسيطة، وأفكاراً لدمج التمرين في روتين الحياة اليومية (قبل القهوة، بعد الصلاة). وسأوضح العلامات التي تقول إن الرفض إشارة إلى مشكلة أعمق تحتاج رأي طبي أو نفسي. هدفنا: حركة آمنة تعيد الثقة، وتضمن أن يبقى كبير العائلة محوراً محترماً في البيت.

\section*{لماذا يرفض كبير السن الحركة؟}

\subsection*{خوف وإحراج أكثر من عناد}

كثير من الأسر ترى الرفض على أنه "عناد" فقط، بينما الواقع أن خلفه أسباباً عميقة غالباً: - أحياناً يكون خوفاً صريحاً من السقوط: المريض يتذكر آخر سقوط، فيخاف أن يتكرر. - وأحياناً يكون إحساساً بالضعف أو فقدان الكرامة: "كيف كنت أقوم وحدي، والآن تحتاجون أن ترفعوني؟". - وفي بعض الأحيان يكون جزءاً من اكتئاب أو قلق لم يُشخَّص بعد.

عندما نغيّر نظرتنا من "عناد" إلى "ألم داخلي"، يتغير أسلوب كلامنا وتصبر قلوبنا أكثر.

\section*{أسلوب الكلام: الدعم لا الأمر}

\subsection*{كيف نفتح الحوار؟}
\begin{itemize}
\item
    \textbf{اسأل واستمع:} «ما أكثر شيء يقلقك في هذا التمرين؟» ثم أعد ما سمعت: «إذن أنت تخاف أن تتعب بسرعة، صحيح؟» [6].
  
\item
    \textbf{اعرض خيارات:} «تفضل نبدأ بدقيقتين مشي داخل الممر أو جلوس ووقوف من الكرسي؟» [5].
  
\item
    \textbf{اربط الهدف بما يحبه:} «لن نطيل، نريد فقط أن يساعدك هذا على الوضوء براحة أو الجلوس مع أحفادك» [10].
  
\end{itemize}

\subsection*{عبارات مجرّبة}
\begin{itemize}
\item
    «نحن معك، ونتوقف فوراً إذا شعرت بألم غير مريح.»
  
\item
    «دعنا نجرب حركة قصيرة، وإذا أعجبتك نزيدها غداً.»
  
\item
    «هدفنا أن تعود تمشي للمسجد بثقة.»
  
\end{itemize}

\subsection*{مثال حواري يوضح الفرق}

بدلاً من سرد قائمة طويلة لما يُقال وما لا يُقال، دعونا نرى مثالاً حقيقياً بتغيير الأسلوب فقط: - \textbf{أسلوب مؤذٍ (بدون قصد):} "لو تبغى تتحسن فعلاً، قم. فلان أكبر منك ويقوم وحده، وأنت طول اليوم جالس!" - \textbf{أسلوب بديل يحترم الكرامة:} "أنا أعرف أن التمرين متعب، لكن كل خطوة اليوم تحمينا من السقوط بكرة إن شاء الله. ما رأيك نجرّب معاً دقيقتين فقط، وإذا تعبت نرتاح؟"

الفرق ليس في المعلومة، بل في الطريقة: الأول يجرح ويقارن، والثاني يشرح ويشجع ويعطي خياراً معقولاً.

\section*{مبدأ «الخطوة الصغيرة»: التدرج الذي يكسر الخوف}

مبدأ الخطوة الصغيرة

قاعدة بسيطة أحب أن أكررها للأسر: "أفضل تمرين هو التمرين الذي يقبله المريض اليوم، ويقدر أن يكرره غداً."

لا نبدأ ببرنامج طويل وثقيل، بل بدقيقتين أو ثلاث من حركة بسيطة، في وقت مريح له (مثلاً بعد القيلولة، أو بعد الصلاة، أو قبل القهوة التي يحبها). عندما ينجح في خطوة صغيرة ويشعر بالفخر، يصبح مستعداً لخطوة أكبر لاحقاً.

\subsection*{لماذا ينجح؟}

التدرج يبني نجاحات صغيرة تعيد الثقة، ويخفف استجابة الخوف من الألم أو السقوط [8][9]. دراسات الدمج بين التمرين والتعرض التدريجي للخوف خفضت قلق السقوط وسلوك التجنب بشكل واضح [9].

\subsection*{خطة تطبيق سهلة في البيت}

\begin{enumerate}
\item
    \textbf{جرّب دقيقتين فقط:} جلوس ووقوف من الكرسي مرتين مع تقييم الراحة بعدها.
  
\item
    \textbf{زد ببطء:} أضف تكرارين أو دقيقة واحدة كل أسبوع إذا كان مرتاحاً.
  
\item
    \textbf{سجّل النجاح:} ورقة على الثلاجة توضح عدد التكرارات تعطي شعور إنجاز [7].
  
\item
    \textbf{مديح فوري:} كلمة تشجيع بعد كل محاولة تزيد الدافعية الداخلية [5].
  
\end{enumerate}

\subsection*{سلامة أولاً}
\begin{itemize}
\item
    مساحة خالية من العوائق، كرسي ثابت، إضاءة كافية.
  
\item
    \textbf{توقف فوراً إذا:} ظهر ألم حاد جديد، دوار شديد، أو فقدان توازن مفاجئ. في هذه الحالات أوقف التمرين واتصل بالطبيب إذا استمر الشعور أو ساءت الأعراض.
  
\end{itemize}

\section*{إشراك الأسرة بلا ضغط ولا صدام}

الدعم الاجتماعي المخصص للنشاط البدني يزيد التزام كبار السن [7]. عندما يشعر الكبير أن العائلة شريكة وليست محكمة، يقل التوتر.

\subsection*{كيف نتوزع الأدوار؟}
\begin{itemize}
\item
    \textbf{التذكير اللطيف:} ابن أو ابنة يذكره بالموعد بلغة مطمئنة.
  
\item
    \textbf{تهيئة المكان:} أحد أفراد الأسرة يرتب زاوية آمنة للتمرين.
  
\item
    \textbf{مشاركة الأحفاد:} حفيد يرافقه في مشي خفيف يجعل التمرين اجتماعياً وممتعاً.
  
\end{itemize}

\subsection*{قواعد التعامل العائلي}
\begin{itemize}
\item
    اتفقوا على رسالة موحدة («نريدك تتحرك لنراك أقوى») لتجنب التوجيهات المتضاربة.
  
\item
    تجنبوا النقاش الحاد أمامه؛ يشعر بأنه عبء.
  
\item
    عززوا كرامته: امدحوا جهده مهما كان بسيطاً.
  
\end{itemize}

\section*{ربط التمرين بالمعنى والروتين اليومي}

\subsection*{أنشطة ذات معنى ثقافي وروحي}
\begin{itemize}
\item
    \textbf{الصلاة والعبادة:} المشي إلى المسجد، أو تمرين بسيط بعد الوضوء ليحافظ على الوقوف في الصلاة [10].
  
\item
    \textbf{جلسة القهوة:} خمس دقائق إطالة قبل القهوة المسائية تصبح عادة ثابتة.
  
\item
    \textbf{بيئة مألوفة:} في القرى أو البيوت الواسعة، فناء البيت أو المزرعة مكان مريح يقلل الإحساس بالغربة [1].
  
\end{itemize}

\subsection*{أدوات مساعدة بسيطة}
\begin{itemize}
\item
    مؤقت الهاتف أو مسبحة العد لقياس التكرارات.
  
\item
    كرسي ثابت بجانب الجدار لتمارين الجلوس والوقوف.
  
\item
    إضاءة جيدة ليلاً لتقليل الخوف من التعثر [2].
  
\end{itemize}

\textbf{رسالة تشجيع:} ربط الحركة بمعنى عزيز (الصلاة، اللعب مع الأحفاد) يحول التمرين من عبء إلى جزء محبب من اليوم.

\section*{متى يكون الرفض إنذاراً أعمق؟}

أحياناً يكون الرفض صرخة استغاثة تحتاج تقييم متخصص.
\begin{itemize}
\item
    \textbf{حزن شديد أو فقدان اهتمام بالحياة:} قد يشير لاكتئاب يحتاج رعاية مهنية [1].
  
\item
    \textbf{قلق غير منطقي من التمرين:} نوبات هلع أو خوف من «الموت أثناء الحركة» رغم طمأنة الأطباء تحتاج تقييماً نفسياً [6].
  
\item
    \textbf{ارتباك أو نسيان واضح:} قد يدل على تدهور معرفي ويحتاج خطة مختلفة.
  
\item
    \textbf{ألم حاد غير مسيطر عليه:} اضبط الألم مع الطبيب قبل متابعة التمرين.
  
\item
    \textbf{تغير مفاجئ في السلوك أو القوة:} بعد مرض أو دواء جديد يجب فحصه طبياً.
  
\end{itemize}

\textbf{خطوة فورية آمنة:} إذا ظهر ألم حاد مفاجئ أو دوار شديد أثناء التمرين، أوقف الجلسة، أجلس الكبير بأمان، واتصل بالطبيب إذا لم يختفِ الشعور سريعاً. سلامته أولاً.

\section*{خلاصة الفصل}

رفض كبير السن للحركة ليس تحدياً شخصياً ضد الأسرة، بل خوف أو إحراج أو ألم يمكن تفكيكه بتواصل دافئ وتدرج ذكي. عندما نربط التمرين بأهداف يحبها، ونشارك الأسرة في دعم موحد، يصبح التقدم ممكناً ولو بخطوات صغيرة. وإذا ظهرت علامات إنذار، فطلب تقييم طبي أو نفسي هو الخيار الآمن. بهذه المعادلة نحمي كرامته ونساعده على حياة أكثر نشاطاً وأماناً.

في النهاية، تذكّروا أن تحويل الرفض إلى قبول لا يحدث في جلسة واحدة، بل هو رحلة قصيرة من بناء الثقة. في الفصل القادم، سنضيف أداة عملية تساعدكم أن تروا التقدم، حتى لو كان بطيئاً: دفتر متابعة بسيط يحوّل "أحس أنه ما تحسن" إلى أرقام وكلمات ملموسة.

\section*{المراجع العلمية}

[1] Meredith, S. J., Maldeniya, R., \& Smith, R. (2023). Factors that influence older adults' participation in physical activity: A systematic review of qualitative studies. \emph{Age and Ageing}, 52(8), afad145. https://doi.org/10.1093/ageing/afad145

[2] MacKay, S., Ebert, T., \& Musselman, K. (2021). Fear of falling in older adults: A scoping review of recent literature. \emph{Canadian Geriatrics Journal}, 24(4), 379--394. https://doi.org/10.5770/cgj.24.521

[3] Aydin, M. A., Altun, D., \& Kocak, G. (2025). The effect of kinesiophobia and successful aging on quality of life in older adults: Machine learning approach. \emph{BMC Geriatrics}, 25, 811. https://doi.org/10.1186/s12877-025-06482-8

[4] Stanmore, E. K., Crossland, K., \& Todd, C. (2021). Fear-of-falling and associated risk factors in persons with rheumatoid arthritis: A 1-year prospective study. \emph{BMC Musculoskeletal Disorders}, 22, 260. https://doi.org/10.1186/s12891-021-04068-0

[5] Morbée, S., Schouten, B., \& Geuens, M. (2023). The role of communication style and external motivators in predicting health behavior intentions: An experimental vignette study. \emph{Health Communication}, 38(13), 2894--2903. https://doi.org/10.1080/10410236.2022.2125012

[6] Serdarevic, M. (2016). Motivational interviewing to affect behavior change in geriatric patient population of older adults. \emph{Journal of Psychology \& Clinical Psychiatry}, 6(7), 00409. https://doi.org/10.15406/jpcpy.2016.06.00409

[7] Smith, G. L., Banting, L., \& Eime, R. (2017). The association between social support and physical activity in older adults: A systematic review. \emph{International Journal of Behavioral Nutrition and Physical Activity}, 14, 56. https://doi.org/10.1186/s12966-017-0509-8

[8] Kuss, K., Fawcett, R., \& Akuthota, V. (2016). Graded activity for older adults with chronic low back pain: Program development and mixed-methods feasibility cohort study. \emph{Pain Medicine}, 17(12), 2218--2229. https://doi.org/10.1093/pm/pnw062

[9] Loebach Wetherell, J., Petkus, A. J., \& Lenze, E. (2018). Integrated exposure therapy and exercise reduces fear of falling and avoidance in older adults: A randomized pilot study. \emph{American Journal of Geriatric Psychiatry}, 26(8), 849--859. https://doi.org/10.1016/j.jagp.2018.04.001

[10] Clemson, L., Fiatarone Singh, M., \& Cumming, R. (2012). Integration of balance and strength training into daily life activity to reduce rate of falls in older people (LiFE study): Randomised parallel trial. \emph{BMJ}, 345, e4547. https://doi.org/10.1136/bmj.e4547
