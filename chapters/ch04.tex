\chapter{التوازن وخطر السقوط عند كبار السن}
\label{ch:04}

\section*{لماذا هذا الفصل مختلف بالنسبة لي؟}

هذا الفصل ليس مجرد معلومات مهنية عن التوازن والسقوط؛ بالنسبة لي هو ملف شخصي جداً.

سقوط واحد غيّر عندي نظرتي بالكامل لموضوع "كسور كبار السن".\\
أبي -- حفظه الله -- انزلق في صالة البيت بعدما كانت العاملة تنظف الأرضية بالماء. البلاط كان مبلول، وهو قام يمشي كعادته، وفجأة زلّت رجله ووقع بكل وزنه على الجنب.

من تلك اللحظة بدأت فترة مرعبة فعلياً.

نُقل للمستشفى، وتبيّن أنه يحتاج عملية تثبيت لعنق عظمة الفخذ بمسمار ديناميكي. ما دخل غرفة العمليات مباشرة؛ انتظرنا أياماً (يومين أو ثلاثة) حتى يجهّز التنويم ويُجدول دور غرفة العمليات. خلال هذه الفترة كان الألم لا يُحتمل تقريباً، حتى مع حقن المسكنات القوية.\\
كان لا يستطيع القيام للحمام، والاحتياج المستمر للمساعدة في أبسط الأشياء كان يجرح قلبه قبلي. كنت أسمع أنين ألمه في الليل، وأشعر وكأن الفخذ المكسورة هي فخذي أنا.

يوم العملية كان من أصعب الأيام في حياتي:\\
كبير سن فوق السبعين، أمام عملية عظام كبيرة، تحت تخدير كامل.\\
الأسئلة كانت تهاجمني:
\begin{itemize}
\item
    هل سيتحمّل البنج؟
  
\item
    هل ستنجح العملية بدون مضاعفات؟
  
\item
    هل سيرجع يمشي مثل قبل؟
  
\item
    ماذا لو حصلت جلطة؟ ماذا لو لم يعد مثل ما كان؟
  
\end{itemize}

تمّت العملية ولله الحمد، لكن الألم الطبيعي بعد الجراحة عند كبار السن يكون مضاعفاً، خصوصاً إذا كان الشخص أصلاً لا يحب المستشفيات ولا جوها. كانت فترة صعبة جداً، لا أتمنى أن يمر بها أي أب أو أم ولا أي أسرة.

الوجه الآخر للصورة أنني رأيت حالات لم تكن النهاية فيها سعيدة.\\
أذكر شيخاً كبيراً تعرّض أيضاً لكسر في الفخذ، ولكبر سنه تعذر حتى تثبيت الكسر جراحياً، وبعدها أصبح طريح فراش معظم الوقت. حصل عنده ضعف شديد وقصور في عضلات الفخذ الخلفية (لأنه كان لا يمدّ رجله، والوضعية المريحة له تكون ركبته مثنية، واستمر طويلاً على هذه الوضعية حتى أصبح لديه قصور شديد في أوتار الفخذ التي بسببها لم يعد يستطيع المشي)، وبسبب العمر والأمراض المصاحبة صار من الصعب تنفيذ برنامج تأهيلي قوي أو الوصول لنتائج كاملة.\\
الكسر هنا لم يكن مجرد "حادث" وانتهى؛ كان نقطة تحوّل في حياته. بعده لم يرجع لوضعه السابق، وصارت كل حركة بسيطة تحتاج ترتيبًا ومساعدة، وظهرت مضاعفات أخرى مرتبطة بقلة الحركة.

هذه القصص -- قصة أبي الذي عاد بحمد الله تدريجياً لأنشطته السابقة، وقصة الشيخ الذي لم يستطع أن يستعيد مشيه كما كان -- علمتني أن:
\begin{itemize}
\item
    السقوط عند كبير السن ليس موضوعاً بسيطاً أبداً\\

\item
    أحياناً تتغيّر حياة الأسرة بالكامل بسبب كسرة واحدة في الورك\\

\item
    بعض الكسور يمكن تجاوزها بالتأهيل الجيد، وبعضها يترك أثراً دائماً
  
\end{itemize}

وأهم درس خرجت به: الجزء الأكبر مما حصل كان يمكن تقليل احتماله أو منعه من البداية:
\begin{itemize}
\item
    لو كانت الأرضية غير مبلولة\\

\item
    لو كانت هناك بُسط مانعة للانزلاق\\

\item
    لو كان يستخدم أداة مناسبة للمشي داخل البيت\\

\item
    لو كان البيت "مهيأ" لكبير سن يمشي على سيراميك ناعم
  
\end{itemize}

لهذا أتعامل مع كل سقطة لكبير سن اليوم على أنها إنذار مبكر، ورسالة تقول لنا:

"إمّا أن تتعلّموا من هذه السقطة الآن، أو تنتظروا سقطة ثانية قد يكون ثمنها أكبر بكثير."

من هذا اليوم، صار موضوع السقوط بالنسبة لي ليس مجرد "فصل في كتاب" أو "محور في محاضرة"، بل مسؤولية شخصية تجاه كل أب وأم وجد وجدة في بيوتنا.

هذا الفصل أكتبه وأنا أضع عيني على حادثة أبي، وعيني الثانية على كل بيت في بيشة وخارجها فيه كبير سن يمشي على سيراميك ناعم، أو يدخل حمام بدون مقابض، أو يتحرك في ممر مظلم.

\section*{1. لماذا السقوط عند كبار السن خطير لهذه الدرجة؟}

\subsection*{1.1 حجم المشكلة عالميًا ومحليًا}
\begin{itemize}
\item
    \textbf{عالميًا}: تقريبًا ثلث كبار السن فوق 65 سنة يتعرضون لسقوط واحد على الأقل في السنة، والنسبة ترتفع إلى 50\% لمن تجاوزوا الـ 80 سنة. كثير من هذه السقطات ينتج عنها إصابات تحتاج دخول مستشفى أو إعادة تأهيل طويلة [1][9].\\
  }
  
\item
    \textbf{في السعودية}: دراسات في الرياض ومناطق أخرى أظهرت أن نسبة السقوط قد تصل إلى 40--50\% بين كبار السن، وغالبًا يكون السقوط داخل البيت نفسه، وليس خارجه [2][3].
  
\end{itemize}

\subsection*{1.2 ماذا يحدث بعد السقوط؟}

السقوط قد يسبب:
\begin{itemize}
\item
    كسور خطيرة مثل كسور عنق عظمة الفخذ (مثل ما حصل مع والدي)، كسور الرسغ، أو فقرات العمود الفقري.\\

\item
    تدهور مفاجئ في الاستقلالية: كبير كان يمشي لوحده، بعد السقوط يحتاج مساعدة في كل صغيرة وكبيرة.\\

\item
    خوف من السقوط (Fear of Falling): كثير من كبار السن بعد أول سقطة يبدأون يخافون من الحركة، فيقلّ نشاطهم، تضعف عضلاتهم أكثر، ويزيد خطر السقوط مرة ثانية [4][10].
  
\end{itemize}

\subsection*{1.3 لماذا نسميه "حدث يمكن منعه" وليس "قضاء وقدر"؟}

نحن نؤمن بالقضاء والقدر، لكن ديننا وعقلنا يقولان إن علينا الأخذ بالأسباب.\\
الدلائل العلمية واضحة: \textbf{تعديل البيئة + تمارين مناسبة + مراجعة طبية دورية} يمكن أن يقلل نسبة السقوط بشكل ملموس [13][14][15].

\textbf{الهدف من هذا الفصل:}
\begin{itemize}
\item
    نفهم لماذا يسقط كبار السن\\

\item
    نتعلم كيف نقيم الخطر في البيت\\

\item
    نطبق خطوات عملية تقلل من احتمال أن نعيش تجربة سقوط مؤلمة مثل اللي عشتها مع والدي
  
\end{itemize}

\section*{2. كيف يعمل التوازن في جسم الإنسان؟}

التوازن عند الإنسان ليس "حاسة واحدة"، بل نتيجة تعاون ثلاث أنظمة رئيسية [5][6][7]:

\subsection*{2.1 العيون (النظام البصري)}
\begin{itemize}
\item
    تعطينا صورة عن المكان حولنا: الدرج، السجاد، العوائق، مستوى الإضاءة\ldots \\
  }
  
\item
    مع التقدم في العمر، يتأثر النظر بسبب:

  \begin{itemize}
  \item
        إعتام عدسة العين\\

  \item
        ضعف الرؤية الليلية\\

  \item
        أمراض مثل الجلوكوما\\

  \end{itemize}
\item
    الدراسات توضح أن ضعف النظر يزيد خطر السقوط بشكل واضح [6].
  
\end{itemize}

\subsection*{2.2 الأذن الداخلية (النظام الدهليزي)}
\begin{itemize}
\item
    جزء حساس موجود في الأذن الداخلية يرسل للدماغ معلومات عن حركة الرأس واتجاه الجاذبية [5].\\
  }
  
\item
    مع العمر وبعض الأمراض، ممكن يحصل:

  \begin{itemize}
  \item
        دوار\\

  \item
        إحساس "لفّة" أو عدم ثبات\\

  \end{itemize}
\item
    هذه المشاكل تزيد خطر السقوط خاصة عند الوقوف المفاجئ أو الالتفات السريع.
  
\end{itemize}

\subsection*{2.3 الإحساس العميق (Proprioception)}
\begin{itemize}
\item
    مستقبلات حسية في العضلات والمفاصل والجلد تخبر الدماغ:

  \begin{itemize}
  \item
        أين قدمي الآن؟\\

  \item
        هل الركبة مثنية أو ممدودة؟\\

  \end{itemize}
\item
    في كبار السن، يتأثر الإحساس العميق بسبب:

  \begin{itemize}
  \item
        التقدم في السن نفسه\\

  \item
        أمراض مثل سكري الأطراف (اعتلال الأعصاب الطرفية)
    
  \end{itemize}
\end{itemize}

\subsection*{2.4 ماذا يعني هذا عمليًا؟}

الشاب إذا تعثّر:
\begin{itemize}
\item
    عينه تشوف المشكلة\\

\item
    أذنه الداخلية تحس بالاختلال\\

\item
    الإحساس العميق يبلغ عن وضع القدم\\

\item
    والعضلات تستجيب بسرعة لتصحيح التوازن
  
\end{itemize}

كبير السن:
\begin{itemize}
\item
    المعلومات توصل متأخرة\\

\item
    والعضلات أضعف\\

\item
    وردة الفعل أبطأ\\

\item
    فيسقط قبل ما يمدّي جسمه يعدّل وضعه [7][8].
  
\end{itemize}

\section*{3. لماذا يسقط كبار السن؟ (عوامل الخطر)}

\subsection*{3.1 عوامل داخلية (من الشخص نفسه)}

\begin{enumerate}
\item
    \textbf{ضعف العضلات وقلة الحركة}

  \begin{itemize}
  \item
        عضلات الفخذ والساق هي "فرامل" الجسم وقت التعثر.\\

  \item
        قلة النشاط اليومي تضعفها، فيصبح أي دفع بسيط أو تعثر سبباً للسقوط [8][15].
    
  \end{itemize}
\item
    \textbf{اضطراب المشي والتوازن}

  \begin{itemize}
  \item
        مشية بطيئة جدًا\\

  \item
        جر القدمين بدل رفعهما (Shuffling gait)\\

  \item
        فرق واضح بين قوة الجانبين (مثلاً بعد جلطة دماغية) [8][9][11].
    
  \end{itemize}
\item
    \textbf{أمراض مزمنة}

  \begin{itemize}
  \item
        السكري → اعتلال أعصاب القدمين، ضعف الإحساس بالأرض\\

  \item
        باركنسون → تجمّد في الحركة، خطوات قصيرة، فقدان توازن\\

  \item
        خشونة الركبة والورك → ألم، مشية غير ثابتة\\

  \item
        هبوط الضغط الانتصابي → دوخة عند الوقوف المفاجئ [9][10]
    
  \end{itemize}
\item
    \textbf{الأدوية المتعددة}

  \begin{itemize}
  \item
        الأدوية المهدئة، أدوية النوم، بعض أدوية الضغط والقلب قد تسبب:

    \begin{itemize}
    \item
            دوخة، نعاس، بطء في ردود الأفعال\\

    \end{itemize}
  \item
        استخدام عدد كبير من الأدوية يزيد خطر السقوط [10][11]\\
    }
    
  \item
         المهم: لا توقف ولا تعدّل أي دواء بدون طبيب.
    
  \end{itemize}
\item
    \textbf{الخوف من السقوط}

  \begin{itemize}
  \item
        بعد سقطة قوية، بعض كبار السن:

    \begin{itemize}
    \item
            يقللون حركتهم\\

    \item
            يتجنبون المشي داخل أو خارج البيت\\

    \end{itemize}
  \item
        هذا يزيد الضعف العضلي ويزيد خطر سقوط جديد [4][10].
    
  \end{itemize}
\end{enumerate}

\subsection*{3.2 عوامل بيئية (من البيت نفسه)}

الدراسات السعودية توضح أن أغلب السقطات تحصل داخل البيت [2][3]:

\begin{enumerate}
\item
    \textbf{الحمام}

  \begin{itemize}
  \item
        أرضية مبلولة وزلقة\\

  \item
        عدم وجود مقابض في الجدران\\

  \item
        مرحاض منخفض يصعب الوقوف منه
    
  \end{itemize}
\item
    \textbf{الممرات والصالة}

  \begin{itemize}
  \item
        سجاد صغير غير مثبت\\

  \item
        أسلاك كهرباء على الأرض\\

  \item
        طاولات صغيرة بوسط المسار\\

  \item
        ألعاب أطفال أو عوائق متفرقة
    
  \end{itemize}
\item
    \textbf{الإضاءة}

  \begin{itemize}
  \item
        ممرات مظلمة\\

  \item
        عدم وجود إضاءة ليلية\\

  \item
        الانتقال من غرفة مضيئة إلى ممر مظلم
    
  \end{itemize}
\item
    \textbf{الدرج والعتبات}

  \begin{itemize}
  \item
        درج بدون درابزين جيد\\

  \item
        عتبة عالية عند باب الحمام أو البلكونة
    
  \end{itemize}
\item
    \textbf{الأحذية والملابس}

  \begin{itemize}
  \item
        شبشب مفتوح ينزلق بسهولة\\

  \item
        جوارب على أرض سيراميك\\

  \item
        ثياب أو عباءة طويلة تجرّ على الأرض
    
  \end{itemize}
\end{enumerate}

\section*{4. كيف نقيم خطر السقوط في البيت؟}

\vspace{8pt}\noindent\textcolor{warnborder}{\rule{3pt}{12pt}}\hspace{6pt}
\textbf{تنبيه مهم:}  \hfill\break
\vspace{8pt}
لا تجرّب أي اختبار توازن مع كبير السن وهو لوحده. لازم يكون شخص واقف جنبه، جاهز يمسكه إذا اختلّ توازنه.

\subsection*{4.1 اختبار الوقوف بقدمين متلاصقتين}

\textbf{الطريقة:}

\begin{enumerate}
\item
    خلي كبير السن يقف:

  \begin{itemize}
  \item
        القدمين متقاربتين قدر الإمكان.\\

  \item
        العينان مفتوحتان.\\

  \item
        قريب من طاولة أو كرسي يمسكه لو احتاج.\\

  \end{itemize}
\item
    حاول يخلي وضعيته ثابتة لمدة 30 ثانية.
  
\end{enumerate}

\textbf{كيف نقرأ النتيجة؟} [11]
\begin{itemize}
\item
     لو قدر يثبت 30 ثانية بدون تمسك واضح → توازن جيد.\\

\item
     لو تمايل أو فتح رجوله بعد أقل من 30 ثانية → توازن متوسط.\\

\item
     لو ما قدر يثبت أصلاً أو اضطر يمسكك بقوة طوال الوقت → خطر سقوط مرتفع.
  
\end{itemize}

\textbf{🛑 توقف فورًا إذا:}
\begin{itemize}
\item
    حس بدوخة مفاجئة أو غباش في النظر.\\

\item
    قال لك إنه "بيطيح" أو تلاحظ رجوله تنهار.\\

\item
    اشتكى من ألم حاد في الركبة أو الورك أو الصدر.
  
\end{itemize}

لو حصل أي من هذه، أوقف الاختبار فورًا، رجّعه يجلس، وبلغ الطبيب لاحقًا.

\subsection*{4.2 اختبار الوقوف على قدم واحدة}

\textbf{الطريقة:}

\begin{enumerate}
\item
    خله يقف بالقرب من طاولة ثابتة أو سطح يمسكه.\\

\item
    يرفع قدم واحدة قليلًا عن الأرض.\\

\item
    نحسب كم ثانية يقدر يثبت بدون ما يرجع القدم للأرض.
  
\end{enumerate}

\textbf{كيف نقرأ النتيجة؟} [11]
\begin{itemize}
\item
     أكثر من 10 ثوانٍ → ممتاز.\\

\item
    بين 5--10 ثوانٍ → مقبول لكن التوازن قابل للتحسين.\\

\item
     أقل من 5 ثوانٍ أو ما قدر يرفع رجله → خطر سقوط مرتفع.
  
\end{itemize}

\textbf{🛑 توقف فورًا إذا:} 
\begin{itemize}
\item
    بدأت رجله تهتز بقوة وشفته على وشك السقوط.\\

\item
    حس بألم حاد في مفصل الورك أو الركبة أو الكاحل.\\

\item
    ظهرت عليه علامات دوار أو تعرق مفاجئ.
  
\end{itemize}

\subsection*{4.3 اختبار "القيام والمشي والعودة" (Timed Up \& Go -- TUG)}

\textbf{الطريقة:} [12][13]

\begin{enumerate}
\item
    يجلس كبير السن على كرسي عادي (ارتفاع متوسط، له مساند للذراعين إن أمكن).\\

\item
    نحط علامة على الأرض على بعد 3 أمتار.\\

\item
    نعطيه التعليمات:

  \begin{itemize}
  \item
        "لما أقول (ابدأ)، تقوم من الكرسي، تمشي لين العلامة، تلف، ترجع، وتجلس مرة ثانية."\\

  \end{itemize}
\item
    نبدأ توقيت من لحظة ما يبدأ يقوم، إلى أن يجلس مرة أخرى.
  
\end{enumerate}

\textbf{كيف نقرأ النتيجة؟}
\begin{itemize}
\item
     أقل من 10 ثوانٍ → طبيعي (لكبار السن النشيطين).\\

\item
    بين 10--14 ثانية → مقبول، لكن ننتبه لعوامل خطر أخرى.\\

\item
     أكثر من 14 ثانية → خطر سقوط أعلى، يحتاج تقييم من أخصائي علاج طبيعي.
  
\end{itemize}

\textbf{🛑 توقف فورًا إذا:}
\begin{itemize}
\item
    توقف في منتصف الطريق وقال "ما أقدر أكمل".\\

\item
    بدا يميل لجهة واحدة أو فقد توازنه بوضوح.\\

\item
    شكى من ألم حاد في الصدر أو ضيق تنفس أو دوخة قوية.
  
\end{itemize}

\subsection*{4.4 قائمة تدقيق سريعة للبيت}

تقدر تطبعها وتحط علامة ✔ أو ✖:

\textbf{الحمام:}
\begin{itemize}
\item
    هل يوجد بساط مطاطي مانع انزلاق داخل وخارج الدش؟\\

\item
    هل توجد مقابض إمساك بجوار المرحاض وفي منطقة الاستحمام؟\\

\item
    هل ارتفاع المرحاض مناسب (مو منخفض جدًا)؟\\

\item
    هل الإضاءة قوية وواضحة في الحمام والممر المؤدي له؟
  
\end{itemize}

\textbf{غرفة النوم:}
\begin{itemize}
\item
    هل ارتفاع السرير متوسط (مو واطي جدًا)؟\\

\item
    هل يوجد مصباح قريب من السرير سهل الوصول؟\\

\item
    هل الطريق من السرير للحمام خالٍ من العوائق ليلًا؟
  
\end{itemize}

\textbf{الممرات والصالة:}
\begin{itemize}
\item
    هل تمت إزالة أو تثبيت السجاد الصغير؟\\

\item
    هل تم إبعاد أسلاك الكهرباء من طريق المشي؟\\

\item
    هل توجد إضاءة ليلية في الممرات؟\\

\item
    هل الطاولات الصغيرة والأشياء المتناثرة بعيدة عن مسار المشي؟
  
\end{itemize}

كل "✖" في هذه القائمة هو فرصة لتعديل بسيط قد ينقذنا من كسور خطيرة.

\section*{5. كيف نمنع السقوط؟ خطوات عملية للأسرة}

\subsection*{5.1 تعديل البيئة أولًا (أسهل وأرخص خطوة)}

\textbf{في الحمام:}
\begin{itemize}
\item
    تركيب مقابض إمساك بجوار المرحاض وداخل الدش [13].\\
  }
  
\item
    استخدام كرسي حمام مرتفع (أو قاعدة ترفع ارتفاع المرحاض 10--15 سم).\\

\item
    وضع بُسط مطاطية مانعة للانزلاق.\\

\item
    التأكد من تجفيف الأرضية بعد تنظيفها أو بعد الاستحمام.
  
\end{itemize}

\textbf{في الممرات والصالة:}
\begin{itemize}
\item
    إزالة السجاد الصغير أو تثبيته بلاصق قوي.\\

\item
    تنظيم الأسلاك حول الجدران، وليس في وسط الطريق.\\

\item
    إبعاد الطاولات الصغيرة أو الكراسي من مسار الحركة.
  
\end{itemize}

\textbf{الإضاءة:}
\begin{itemize}
\item
    تركيب مصابيح ليلية في الممر من غرفة النوم إلى الحمام.\\

\item
    التأكد من وجود مفتاح إنارة عند باب كل غرفة.\\

\item
    تجنب الانتقال المفاجئ من غرفة مضيئة إلى ممر مظلم.
  
\end{itemize}

\textbf{الأحذية والملابس:}
\begin{itemize}
\item
    تشجيع كبير السن على ارتداء حذاء مغلق بنعل مطاطي مانع للانزلاق.\\

\item
    تجنب الشبشب المفتوح والجوارب على السيراميك.\\

\item
    تعديل طول الثوب أو العباءة بحيث لا يجرّ على الأرض.
  
\end{itemize}

\subsection*{5.2 تمارين بسيطة لتحسين التوازن والقوة}

التفاصيل الكاملة لبرنامج التمارين في الفصل السادس، لكن هنا نضع المبادئ الأساسية:
\begin{itemize}
\item
    خمس إلى عشر دقائق حركة يومية أفضل من الجلوس طول اليوم.\\

\item
    تمارين مثل:

  \begin{itemize}
  \item
        الجلوس والقيام من كرسي ثابت عدة مرات (مع وجود مرافق في البداية).\\

  \item
        الوقوف خلف الكرسي والتعلق بمسند الكرسي، ثم محاولة الوقوف على قدم واحدة لثوانٍ (مع الأمان الكامل).
    
  \end{itemize}
\end{itemize}

الدراسات توضح أن برامج التمارين التي تركز على التوازن وقوة الأطراف السفلية تقلل خطر السقوط بصورة واضحة [14][15].

 لكن:
\begin{itemize}
\item
    لا تبدأ برنامج تمارين مكثف بدون استشارة الطبيب، خاصة لو فيه أمراض قلب أو ضغط أو دوخة متكررة.
  
\end{itemize}

\subsection*{5.3 مراجعة الأدوية والنظر والسمع}
\begin{itemize}
\item
    \textbf{مراجعة الأدوية مع الطبيب كل 3--6 أشهر:}

  \begin{itemize}
  \item
        التركيز على الأدوية المسببة للدوخة أو النعاس.\\

  \item
        إعادة تقييم الجرعات والتوقيت إذا أمكن [10][11].\\
    }
    
  \end{itemize}
\item
    \textbf{فحص النظر والسمع سنويًا:}

  \begin{itemize}
  \item
        تحديث النظارة عند الحاجة.\\

  \item
        علاج مشاكل مثل الماء الأبيض أو الجلوكوما.\\

  \item
        تقييم الأذن الداخلية إذا فيه دوار متكرر [6][9].
    
  \end{itemize}
\end{itemize}

\subsection*{5.4 استخدام الأجهزة المساعدة (العصا، المشاية\ldots)}

إذا كان التوازن ضعيفًا، فاختيار جهاز مشي مناسب (عصا، مشاية، روليتور) يقلل خطر السقوط عندما يُستخدم بالشكل الصحيح.\\
سنفصل ذلك في الفصل الخامس، لكن المهم الآن:
\begin{itemize}
\item
    لا نعطي كبير السن عصا عشوائية "من باب المجاملة".\\

\item
    الأفضل استشارة أخصائي علاج طبيعي لاختيار الارتفاع والنوع المناسب.
  
\end{itemize}

\section*{6. ماذا نفعل بعد السقوط؟}

\subsection*{6.1 اللحظات الأولى}

لو سقط كبيرك:

\begin{enumerate}
\item
    ابقَ هادئًا وطمئنه:

  \begin{itemize}
  \item
        "أنا معك، لا تتحرك للحين، بس بنشوف كيف وضعك."\\

  \end{itemize}
\item
    اسأله بهدوء:

  \begin{itemize}
  \item
        "تحس بألم وين بالضبط؟"\\

  \item
        "تقدر تحرك رجولك؟ يديك؟"\\

  \end{itemize}
\item
    لا تحاول تشيله بسرعة قبل ما تتأكد أنه ما فيه كسر أو إصابة خطيرة.
  
\end{enumerate}

\subsection*{6.2 متى نتصل بالإسعاف فورًا؟}

اتصل على \textbf{997} أو \textbf{999} فورًا إذا لاحظت أي من التالي:
\begin{itemize}
\item
    فقدان وعي أو تشوش شديد.\\

\item
    ألم حاد في الورك (خاصة لو ما يقدر يوقف أو يحرك الرجل) -- مثل حالة كسر عنق عظمة الفخذ.\\

\item
    تشوه واضح في الساق أو الذراع.\\

\item
    نزيف من الرأس أو جرح عميق.\\

\item
    ألم في الصدر، ضيق تنفس، أو تعرق شديد مفاجئ.
  
\end{itemize}

هذه الحالات لا تحتمل التجربة، ولا نحاول "نقومه ونشوف"؛ الإسعاف هو القرار الصحيح.

\subsection*{6.3 لو ما كان فيه علامات خطر واضحة}
\begin{itemize}
\item
    نساعده يجلس أولًا، مو يوقف فورًا.\\

\item
    نخليه يجلس دقائق، نسأله عن الدوخة والألم.\\

\item
    حتى لو قام ومشى "يبدو طبيعي"، يفضّل:

  \begin{itemize}
  \item
        نبلغ الطبيب في أقرب موعد.\\

  \item
        نراجع الأدوية.\\

  \item
        نعمل تقييم جديد للبيت والبيئة.
    
  \end{itemize}
\end{itemize}

\subsection*{6.4 أهم نقطة بعد أول سقطة}

أي سقطة تعتبر جرس إنذار. لو تجاهلناه، احتمال تتكرر السقوطات بنسبة أعلى بكثير [4][9][13].

بعد أول سقطة:

\begin{enumerate}
\item
    نحلل السبب (أين، متى، على أي أرضية؟ بعد دواء معين؟ في الظلام؟).\\

\item
    نعدّل البيئة فورًا بناءً على السبب.\\

\item
    نطلب تقييم من طبيب الأسرة وأخصائي علاج طبيعي.\\

\item
    ندعم كبير السن نفسيًا:

  \begin{itemize}
  \item
        لا نلومه، ولا نقول "ليش ما انتبهت؟"
    
  \item
        نركز على الحلول: "وش نعدل؟ وش نضبط في البيت؟"
    
  \end{itemize}
\end{enumerate}

\section*{عندما يصبح الخوف أكبر من السقوط نفسه}

من أصعب المواقف التي أواجهها: كبير سن سقط مرة واحدة، ومن يومها رفض المشي. هذا ما نسميه \textbf{"متلازمة ما بعد السقوط"} --- الخوف يشلّ الحركة أكثر من الإصابة نفسها.

كيف نتعامل مع هذا الخوف؟ أولاً، لا نستهزئ به أبداً لأنه حقيقي ومشروع. ثانياً، نبدأ بخطوات صغيرة جداً --- حتى الوقوف بجانب السرير لثوانٍ يُعتبر إنجازاً. ثالثاً، نحتفل بكل تقدم مهما بدا بسيطاً. وإذا استمر الخوف لأسابيع رغم المحاولات، قد يحتاج كبير السن دعماً نفسياً متخصصاً --- وهذا ليس عيباً بل خطوة حكيمة.

\section*{7. خلاصة الفصل: كيف نحمي أهلنا من تكرار تجربتي مع والدي؟}

بعد تجربة سقوط والدي وكسور الورك، صرت أرى كل بيت فيه كبير سن على أنه مشروع \textbf{وقاية}، وليس "قنبلة موقوتة".

أهم ما أود أن يخرج به القارئ من هذا الفصل:

\begin{enumerate}
\item
    السقوط عند كبار السن شائع\ldots  لكن يمكن منعه في كثير من الحالات.\\
  }
  
\item
    التوازن يعتمد على:

  \begin{itemize}
  \item
        العيون\\

  \item
        الأذن الداخلية\\

  \item
        الإحساس العميق
    
  \end{itemize}
\end{enumerate}
\begin{itemize}
\item
    ومع العمر والأمراض والأدوية تتأثر هذه الأنظمة [5][6][7][8].
  
\end{itemize}

\begin{enumerate}
\setcounter{enumi}{2}
\item
    العوامل الداخلية (ضعف عضلات، أمراض مزمنة، أدوية متعددة، خوف من السقوط) يمكن التعامل معها بالتعاون مع الطبيب وأخصائي العلاج الطبيعي [4][8][9][10][11].\\
  }
  
\item
    العوامل البيئية (حمام، ممرات، سجاد، إضاءة، أحذية) كثير منها يمكن تعديله خلال أيام قليلة، بتكاليف بسيطة، لكن أثرها كبير [2][3][13][14].\\
  }
  
\item
    تقييم التوازن لا يحتاج أجهزة معقدة؛ اختبارات بسيطة (قدمين متلاصقتين، قدم واحدة، TUG) تعطي فكرة جيدة عن مستوى الخطر -- بشرط الالتزام بتحذيرات السلامة [11][12][13].\\
  }
  
\item
    بعد أي سقطة، لا نكتفي بـ "الحمد لله عدت على خير"، بل:

  \begin{itemize}
  \item
        نبحث عن السبب\\

  \item
        نغير البيئة\\

  \item
        نراجع الأدوية\\

  \item
        ونبني خطة للوقاية من التكرار
    
  \end{itemize}
\end{enumerate}

في الفصول القادمة، راح ندخل في التفاصيل العملية أكثر:
\begin{itemize}
\item
    \textbf{الفصل الخامس}: كيف نختار ونستخدم العصا والمشاية والكرسي المتحرك بأمان؟\\

\item
    \textbf{الفصل السادس}: كيف نصمم برنامج تمارين منزلية بسيط، آمن، وواقعي يناسب كبار السن في بيوتنا؟
  
\end{itemize}

\section*{المراجع العلمية}

[1] World Health Organization. (2021). Falls: Key facts. WHO Fact Sheets.

[2] Almegbel F.Y., et al.~(2018). Period prevalence, risk factors and consequent injuries of falls among the Saudi elderly living in Riyadh, Saudi Arabia: A cross-sectional study. \emph{BMJ Open}, 8(1), e019063.

[3] Alshammari S.A., et al.~(2018). Falls among elderly and its relation with their health problems and surrounding environmental factors in Riyadh. \emph{Journal of Family and Community Medicine}, 25(1), 29--34.

[4] Friedman S.M., et al.~(2002). Falls and fear of falling: Which comes first? A longitudinal prediction model suggests strategies for primary and secondary prevention. \emph{Journal of the American Geriatrics Society}, 50(8), 1329--1335.

[5] Peterka R.J. (2018). Sensory integration for human balance control. \emph{Handbook of Clinical Neurology}, 159, 27--42.

[6] Lord S.R. (2006). Visual risk factors for falls in older people. \emph{Age and Ageing}, 35(suppl\_2), ii42--ii45.

[7] Seidler R.D., et al.~(2010). Motor control and aging: Links to age-related brain structural, functional, and biochemical effects. \emph{Neuroscience \& Biobehavioral Reviews}, 34(5), 721--733.

[8] Montero-Odasso M., et al.~(2022). World guidelines for falls prevention and management for older adults: A global initiative. \emph{Age and Ageing}, 51(9), afac205.

[9] Freeman R., et al.~(2011). Orthostatic hypotension: JACC state-of-the-art review. \emph{Journal of the American College of Cardiology}, 72(11), 1294--1309.

[10] Woolcott J.C., et al.~(2009). Meta-analysis of the impact of 9 medication classes on falls in elderly persons. \emph{Archives of Internal Medicine}, 169(21), 1952--1960.

[11] Vellas B.J., et al.~(1997). One-leg balance is an important predictor of injurious falls in older persons. \emph{Journal of the American Geriatrics Society}, 45(6), 735--738.

[12] Podsiadlo D., Richardson S. (1991). The timed "Up \& Go": A test of basic functional mobility for frail elderly persons. \emph{Journal of the American Geriatrics Society}, 39(2), 142--148.

[13] Gillespie L.D., et al.~(2012). Interventions for preventing falls in older people living in the community. \emph{Cochrane Database of Systematic Reviews}, (9), CD007146.

[14] Sherrington C., et al.~(2019). Exercise to prevent falls in older adults: An updated systematic review and meta-analysis. \emph{British Journal of Sports Medicine}, 51(24), 1750--1758.
