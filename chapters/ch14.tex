\chapter{طريح الفراش — العناية الحركية والوقاية من المضاعفات}

\section*{لماذا هذا الفصل مهم؟}

من واقع عملي في الزيارات المنزلية، أكثر من 40\% من الحالات التي أزورها هي لمرضى طريحي الفراش.

\begin{storybox}
دخلت بيت أحد المرضى وكان طريح فراش منذ 6 أشهر. لاحظت أن عضلات ساقيه أصبحت متيبسة تماماً، وظهرت قرحة ضغط في منطقة العصعص. سألت الأسرة: هل أحد حرّك رجوله؟ قالوا: نحن نخاف نحركه خشية نؤذيه. وهنا المشكلة.
\end{storybox}

الأسرة غالباً تخاف تتحرك مع المريض، فيحصل التيبس والقروح وضمور العضلات — كل هذا كان ممكن نتجنبه بحركات بسيطة يومية.

\section*{من هو طريح الفراش؟}

طريح الفراش (Bedridden) هو المريض الذي لا يستطيع النهوض من السرير بسبب:

\begin{itemize}
\item جلطة دماغية
\item كسر في الورك أو الفخذ
\item ضعف عام شديد
\item مرض مزمن متقدم (قلب، رئة، كلى)
\item مرحلة نهائية من مرض
\end{itemize}

\section*{المضاعفات الخطيرة لطريح الفراش}

\subsection*{1. تيبّس المفاصل}

\begin{itemize}
\item العضلات والأوتار تقصر إذا ما تحركت لفترة طويلة
\item المفصل يتجمد في وضعية واحدة
\item بعد 2-3 أسابيع، التيبس يصير دائم
\end{itemize}

\textbf{مثال واقعي:}
مريض جلطة دماغية، يده مطوية على صدره لمدة شهرين بدون حركة. النتيجة؟ ما قدرنا نفرد يده أبداً، حتى مع العلاج.

\textbf{الوقاية:}
\begin{itemize}
\item حرّك كل مفصل مرتين يومياً (صباح ومساء)
\item لا تترك المريض في نفس الوضعية أكثر من 3 ساعات
\end{itemize}

\subsection*{2. ضمور العضلات}

الحقيقة المرة:

\begin{itemize}
\item أول أسبوع: تفقد 10\% من قوة العضلة
\item أول شهر: تفقد 30-40\% من الكتلة العضلية
\item بعد 3 أشهر: العضلة تضمر لدرجة يصعب استرجاعها
\end{itemize}

\begin{warnbox}
رجل عمره 68 سنة، كان يمشي بشكل عادي قبل عملية القلب. بعد العملية، بقي في الفراش شهرين كاملين "حتى يرتاح". عندما زرته، لم يستطع الوقوف على قدميه — العضلات ضمرت تماماً. احتجنا 4 أشهر علاج مكثف حتى يمشي مرة ثانية.
\end{warnbox}

\textbf{الوقاية:}
\begin{itemize}
\item شدّ العضلات وأنت في السرير (Isometric exercises)
\item حركات مقاومة خفيفة يومياً
\item جلوس على حافة السرير ولو 5 دقائق يومياً
\end{itemize}

\subsection*{3. قرح الضغط (Pressure Ulcers)}

أخطر مضاعفة — قد تؤدي للوفاة بسبب الالتهابات.

\textbf{الأماكن الأكثر خطورة:}
\begin{itemize}
\item العصعص (Sacrum)
\item الكعبين (Heels)
\item المرفقين
\item عظمة الورك الجانبية
\end{itemize}

\textbf{كيف تحدث؟}

الضغط المستمر → انقطاع الدم → موت الأنسجة → جرح عميق

\textbf{مراحل القرحة:}

\begin{longtable}{>{\raggedright\arraybackslash}p{0.20\textwidth}>{\raggedright\arraybackslash}p{0.35\textwidth}>{\raggedright\arraybackslash}p{0.35\textwidth}}
\toprule\noalign{}\tableheadercolor
\thead{المرحلة} & \thead{الوصف} & \thead{العلاج} \\
\midrule
المرحلة 1 & احمرار لا يزول بالضغط & تغيير الوضعية فوراً \\
المرحلة 2 & ظهور فقاعة أو سحجة & كريمات علاجية + تقليب \\
المرحلة 3 & جرح عميق يصل للعضلة & تدخل طبي ضروري \\
المرحلة 4 & جرح يصل للعظم & عملية جراحية غالباً \\
\bottomrule
\end{longtable}

\begin{warnbox}
\textbf{تحذير — اطلب الإسعاف فوراً إذا:}
\begin{itemize}
\item الجرح أعمق من 1 سم
\item خروج صديد أو رائحة كريهة
\item حرارة المريض فوق 38 درجة
\item انتشار الاحمرار بسرعة
\end{itemize}
\end{warnbox}

\subsection*{4. التهاب الرئة}

\begin{itemize}
\item الاستلقاء الدائم → تجمع السوائل في الرئة
\item ضعف السعال → عدم طرد البلغم
\item صعوبة البلع → دخول الطعام للرئة
\end{itemize}

\textbf{علامات الخطر:}
\begin{itemize}
\item سعال متكرر
\item صعوبة في التنفس
\item حرارة مع قشعريرة
\item بلغم أخضر أو أصفر
\end{itemize}

\textbf{الوقاية:}
\begin{itemize}
\item رفع رأس السرير 30-45 درجة (خاصة وقت الأكل)
\item تمارين تنفس عميق 3 مرات يومياً
\item تقليب المريض كل 2-3 ساعات
\end{itemize}

\subsection*{5. جلطات الأوردة العميقة}

خطر حقيقي: قلة الحركة → بطء الدم في الساقين → تخثر الدم → جلطة

\textbf{علامات الجلطة:}
\begin{itemize}
\item تورم في ساق واحدة فقط
\item ألم في ربلة الساق (عضلة السمانة)
\item احمرار أو سخونة في الساق
\item ألم عند الضغط على العضلة
\end{itemize}

\begin{warnbox}
\textbf{تحذير خطير:} الجلطة قد تنفصل وتذهب للرئة — هذا قاتل!

\textbf{إذا شككت في جلطة:}
\begin{itemize}
\item لا تدلك الساق أبداً
\item اتصل بالإسعاف فوراً
\item أبقِ المريض في السرير
\end{itemize}
\end{warnbox}

\textbf{الوقاية:}
\begin{itemize}
\item حركات القدم (دوران الكاحل) 10 مرات كل ساعة
\item ثني وفرد الركبة 5 مرات كل ساعتين
\item جوارب ضاغطة (Compression stockings) إذا وصفها الطبيب
\end{itemize}

\section*{البرنامج اليومي لطريح الفراش}

\subsection*{الصباح (7-9 صباحاً)}

\textbf{1. الفحص اليومي (5 دقائق)}

\begin{itemize}
\item افحص الجلد في الأماكن المعرضة للضغط
\item لاحظ أي احمرار أو تقرحات جديدة
\item افحص الفم (جفاف، تقرحات)
\item راقب التنفس (سلس، صعوبة)
\end{itemize}

\textbf{2. تمارين التنفس (3 مرات)}

الطريقة:
\begin{enumerate}
\item خذ نفس عميق من الأنف (عدّ لـ 4)
\item احبس النفس (عدّ لـ 2)
\item أخرج الهواء من الفم ببطء (عدّ لـ 6)
\item كرر 5 مرات
\end{enumerate}

\textbf{3. التقليب الأول}

\begin{itemize}
\item من الظهر إلى الجانب الأيمن
\item ضع وسادة بين الركبتين
\item تأكد من راحة الوضعية
\end{itemize}

\subsection*{منتصف النهار (11-12 ظهراً)}

\textbf{4. تمارين المدى الحركي (Range of Motion)}

\textbf{القاعدة الذهبية:}
\begin{itemize}
\item حرّك كل مفصل ببطء للنهاية
\item ثبّت 5 ثوانٍ
\item ارجع ببطء
\item كرر 5-10 مرات
\end{itemize}

\textbf{ترتيب التمارين:}

\textit{الطرف العلوي:}
\begin{itemize}
\item الكتف: رفع للأمام وللجنب
\item الكوع: ثني وفرد
\item الرسغ: ثني وفرد، دوران
\item الأصابع: فتح وإغلاق القبضة
\end{itemize}

\textit{الطرف السفلي:}
\begin{itemize}
\item الورك: ثني وفرد، تبعيد وتقريب
\item الركبة: ثني وفرد
\item الكاحل: ثني وفرد، دوران
\item أصابع القدم: ثني وفرد
\end{itemize}

\textbf{الوقت الكلي:} 15-20 دقيقة لكل الجسم

\begin{tipbox}
\textbf{نصيحة من الميدان:} لا تستعجل. الحركة السريعة تسبب ألم وتشنجات. الحركة البطيئة والثابتة هي الأفضل.
\end{tipbox}

\textbf{5. التقليب الثاني}

من الجانب الأيمن إلى الظهر، أو من الأيمن إلى الجانب الأيسر (بالتناوب)

\subsection*{العصر (3-4 عصراً)}

\textbf{6. تمارين التقوية (إذا كان المريض قادر)}

\textit{تمرين 1: شد عضلة الفخذ}
\begin{itemize}
\item اطلب من المريض يشد عضلة الفخذ (كأنه يضغط على السرير)
\item ثبّت 5 ثوانٍ
\item استرخِ 5 ثوانٍ
\item كرر 10 مرات لكل ساق
\end{itemize}

\textit{تمرين 2: رفع الساق المستقيمة}
\begin{itemize}
\item ارفع الساق مستقيمة 10 سم عن السرير
\item ثبّت 3 ثوانٍ
\item أنزل ببطء
\item كرر 5 مرات (إذا قدر)
\end{itemize}

\begin{warnbox}
\textbf{تحذير:} توقف فوراً إذا شعر المريض بألم حاد أو دوخة.
\end{warnbox}

\textbf{7. التقليب الثالث}

\subsection*{المساء (7-8 مساءً)}

\textbf{8. الجلوس على حافة السرير (إذا ممكن)}

\textbf{الفوائد:}
\begin{itemize}
\item يمنع انخفاض ضغط الدم
\item يقوي عضلات الجذع
\item يحسن التنفس
\item يعطي المريض شعور بالاستقلالية
\end{itemize}

\textbf{الطريقة الآمنة:}
\begin{enumerate}
\item ارفع رأس السرير 45 درجة لمدة 5 دقائق أولاً
\item قلّب المريض على جنبه
\item أنزل رجوله من حافة السرير ببطء
\item ساعده يرفع جذعه بالدفع باليد
\item دعه يجلس 2-3 دقائق (راقب علامات الدوخة)
\item أرجعه للسرير ببطء
\end{enumerate}

\begin{warnbox}
\textbf{إشارة حمراء — أوقف فوراً إذا:}
\begin{itemize}
\item شحوب الوجه
\item تعرق شديد
\item دوخة أو غثيان
\item ضيق نفس
\end{itemize}
\end{warnbox}

\textbf{9. التقليب قبل النوم}

اختر الجانب الأقل ضغطاً، وتأكد من الوسائد في الأماكن الصحيحة.

\section*{جدول التقليب — مهم جداً}

\begin{longtable}{>{\raggedright\arraybackslash}p{0.20\textwidth}>{\raggedright\arraybackslash}p{0.25\textwidth}>{\raggedright\arraybackslash}p{0.45\textwidth}}
\toprule\noalign{}\tableheadercolor
\thead{الوقت} & \thead{الوضعية} & \thead{ملاحظات} \\
\midrule
8 صباحاً & الظهر & افحص الجلد \\
11 صباحاً & الجانب الأيمن & وسادة بين الركبتين \\
2 ظهراً & الظهر أو الأيسر & بالتناوب \\
5 عصراً & الجانب الأيسر & \\
8 مساءً & الظهر & \\
11 ليلاً & الجانب الأيمن & قبل النوم \\
\bottomrule
\end{longtable}

\textbf{ملاحظة:} هذا جدول مقترح. الأهم: لا تترك المريض في نفس الوضعية أكثر من 3 ساعات.

\section*{الوسائد والدعامات — أدوات بسيطة فعّالة}

\subsection*{1. وسادة بين الركبتين}
\textbf{الفائدة:} تمنع احتكاك الركبتين وتحافظ على استقامة العمود الفقري

\subsection*{2. وسادة تحت الكعبين}
\textbf{الفائدة:} ترفع الكعب عن السرير، تمنع قرح الكعب

\textbf{الطريقة الصحيحة:} ضع الوسادة تحت ربلة الساق (وليس تحت الركبة مباشرة)، بحيث يكون الكعب معلّق في الهواء.

\subsection*{3. وسادة خلف الظهر}
\textbf{الفائدة:} تثبت المريض في وضعية الجانب، تمنع التدحرج للخلف

\subsection*{4. وسادة تحت الذراع}
\textbf{الفائدة:} تمنع تدلي الذراع، تحافظ على وضعية الكتف

\section*{التغذية لطريح الفراش}

\subsection*{المشكلة الشائعة: قلة الشهية}

\textbf{الأسباب:}
\begin{itemize}
\item قلة الحركة → قلة الجوع
\item الاكتئاب والعزلة
\item جفاف الفم
\item صعوبة البلع
\end{itemize}

\subsection*{نصائح عملية}

\textbf{1. وجبات صغيرة متكررة}
\begin{itemize}
\item 5-6 وجبات صغيرة أفضل من 3 كبيرة
\item اجعل الوجبات ملونة وجذابة
\end{itemize}

\textbf{2. البروتين أولاً}

طريح الفراش يحتاج 1.2-1.5 جرام بروتين لكل كيلو يومياً

\begin{itemize}
\item بيض
\item دجاج أو سمك
\item لبن وزبادي
\item مكملات بروتين (بعد استشارة الطبيب)
\end{itemize}

\textbf{3. السوائل}

الهدف: 1.5-2 لتر يومياً (إلا إذا منعه الطبيب)

\textbf{علامات الجفاف:}
\begin{itemize}
\item جفاف الشفاه
\item قلة التبول
\item بول أصفر غامق
\item دوخة
\end{itemize}

\textbf{4. وضعية الأكل الآمنة}

\begin{itemize}
\item ارفع رأس السرير 45 درجة على الأقل
\item أبقِ المريض جالساً 30 دقيقة بعد الأكل
\item أطعمه ببطء وبلقمات صغيرة
\end{itemize}

\begin{warnbox}
\textbf{تحذير من الاختناق:} إذا دخل الأكل أو الشرب "بالغلط" (Aspiration):
\begin{itemize}
\item توقف عن الأكل فوراً
\item اطلب من المريض يسعل بقوة
\item إذا لم يتحسن أو ظهر ازرقاق، اطلب الإسعاف
\end{itemize}
\end{warnbox}

\section*{النظافة الشخصية}

\subsection*{الاستحمام في السرير}

\textbf{الأدوات:}
\begin{itemize}
\item حوضين ماء (دافئ ونظيف)
\item فوط ناعمة
\item صابون لطيف
\item منشفة كبيرة
\end{itemize}

\textbf{الترتيب (من الأنظف للأقل نظافة):}
\begin{enumerate}
\item الوجه والعنق
\item الذراعين
\item الصدر والبطن
\item الساقين
\item الظهر
\item المناطق الحساسة (بماء جديد)
\end{enumerate}

\textbf{نصائح مهمة:}
\begin{itemize}
\item غطّي المريض بمنشفة كبيرة للخصوصية والدفء
\item جفف كل منطقة فوراً بعد غسلها
\item استخدم كريم مرطب للجلد الجاف
\end{itemize}

\subsection*{العناية بالفم}

الفم الجاف → تقرحات وألم

إهمال الأسنان → التهابات ورائحة

النظافة الجيدة → شهية أفضل

\textbf{مرتين يومياً:}
\begin{itemize}
\item فرّش الأسنان بفرشاة ناعمة
\item إذا كان المريض فاقد وعي: امسح اللثة بشاش مبلل
\item استخدم مرطب للشفاه
\end{itemize}

\section*{العناية بالجلد — الوقاية من القروح}

\subsection*{الفحص اليومي}

افحص هذه الأماكن كل يوم:
\begin{itemize}
\item العصعص (عظمة المؤخرة)
\item الكعبين
\item الكتفين
\item المرفقين
\item عظمة الورك الجانبية
\item الأذنين (إذا كان ينام على جنبه)
\end{itemize}

\textbf{ما تبحث عنه:}
\begin{itemize}
\item احمرار لا يزول بعد الضغط عليه
\item تغير في لون الجلد (أفتح أو أغمق)
\item سخونة أو برودة في المنطقة
\item تورم خفيف
\item جفاف شديد أو تشقق
\end{itemize}

\subsection*{الكريمات الواقية}

للمناطق المعرضة للضغط:
\begin{itemize}
\item كريم الزنك (Zinc Oxide) — يحمي من الرطوبة
\item كريم الحاجز (Barrier Cream)
\item فيتامين E للجلد الجاف
\end{itemize}

\begin{warnbox}
\textbf{تنبيه:} لا تدلّك المناطق المحمرّة بقوة — هذا يزيد الضرر!
\end{warnbox}

\section*{التواصل مع طريح الفراش}

\subsection*{المريض الواعي}

\begin{warnbox}
أكبر خطأ تفعله الأسر: يتكلمون عن المريض وهو أمامهم كأنه غير موجود. هذا يحطم نفسية المريض.
\end{warnbox}

\textbf{افعل:}
\begin{itemize}
\item كلّمه مباشرة، انظر في عينيه
\item اشرح له كل شيء قبل ما تسويه
\item اسأله عن رأيه وشعوره
\item شاركه في القرارات البسيطة (ماذا يأكل، ماذا يشاهد)
\end{itemize}

\textbf{لا تفعل:}
\begin{itemize}
\item لا تتكلم عنه وكأنه غير موجود
\item لا تستخدم لهجة أطفال (هو مريض، ليس طفلاً)
\item لا تتجاهل شكواه من الألم
\end{itemize}

\subsection*{المريض فاقد الوعي}

هل يسمعك؟ الدراسات تقول: ممكن. السمع آخر حاسة تروح.

\begin{itemize}
\item كلّمه برفق
\item أخبره بما تفعل ("الحين بقلبك على جنبك الأيمن")
\item اقرأ له من القرآن
\item شغّل له آذان الصلاة
\end{itemize}

\section*{متى تطلب المساعدة الطبية؟}

\subsection*{اتصل بالطبيب خلال 24 ساعة إذا:}
\begin{itemize}
\item رفض المريض الأكل أو الشرب ليوم كامل
\item قلة التبول (أقل من مرتين في 24 ساعة)
\item إمساك شديد لأكثر من 3 أيام
\item ظهور احمرار جديد رغم التقليب
\end{itemize}

\subsection*{اطلب الإسعاف فوراً إذا:}
\begin{itemize}
\item حرارة فوق 38.5 درجة
\item صعوبة شديدة في التنفس
\item جرح عميق مع صديد أو رائحة
\item تورم مفاجئ في ساق واحدة (شك في جلطة)
\item فقدان وعي مفاجئ
\item نزيف لا يتوقف
\end{itemize}

\section*{رسالة للأسرة المرهقة}

رعاية طريح الفراش من أصعب أنواع الرعاية المنزلية.

أنت تستيقظ كل 3 ساعات للتقليب. تطعمه ملعقة ملعقة. تغير له الحفاظات. تمسح جسمه بالماء. تحرك أطرافه كل يوم.

هذا عمل شاق.

لكن تذكّر:

\begin{tipbox}
\begin{itemize}
\item كل حركة تسويها تمنع مضاعفة خطيرة
\item كل تقليب يحمي جلده من القروح
\item كل تمرين يحافظ على عضلاته من الضمور
\item أنت تسوي فرق حقيقي
\end{itemize}
\end{tipbox}

\textbf{اعتنِ بنفسك أيضاً:}
\begin{itemize}
\item خذ راحة كل يوم (ولو 30 دقيقة)
\item اطلب مساعدة الأهل بالتناوب
\item تواصل مع مجموعات دعم لمقدمي الرعاية
\item صلّي واطلب من الله القوة
\end{itemize}

\section*{جدول المتابعة اليومي}

\begin{longtable}{>{\raggedright\arraybackslash}p{0.15\textwidth}>{\raggedright\arraybackslash}p{0.45\textwidth}>{\centering\arraybackslash}p{0.10\textwidth}>{\raggedright\arraybackslash}p{0.20\textwidth}}
\toprule\noalign{}\tableheadercolor
\thead{الوقت} & \thead{النشاط} & \thead{تم ✓} & \thead{ملاحظات} \\
\midrule
8 صباحاً & فحص الجلد + تقليب & & \\
9 صباحاً & تمارين تنفس & & \\
11 صباحاً & تمارين مدى حركي + تقليب & & \\
2 ظهراً & تقليب & & \\
5 عصراً & تقليب + جلوس (إن أمكن) & & \\
8 مساءً & تقليب + نظافة شخصية & & \\
11 ليلاً & تقليب قبل النوم & & \\
\bottomrule
\end{longtable}

\section*{الخلاصة}

رعاية طريح الفراش تحتاج:

\begin{enumerate}
\item التقليب المنتظم كل 2-3 ساعات
\item تمارين المدى الحركي يومياً
\item فحص الجلد يومياً
\item تغذية جيدة مع بروتين كافٍ
\item نظافة شخصية منتظمة
\item تواصل إنساني ودعم نفسي
\end{enumerate}

تذكر: المضاعفات ممكن تحصل حتى مع أفضل رعاية. لا تلوم نفسك إذا ظهرت مشكلة. المهم إنك تبذل جهدك وتطلب المساعدة الطبية في الوقت المناسب.

اللهم ارحم ضعفهم، واشفِ مرضهم، وأجرنا في خدمتهم.
