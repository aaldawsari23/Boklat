\chapter{قراءة حالة المريض في البيت بعين أخصائي علاج طبيعي}
\label{ch:02}

\vspace{8pt}\noindent\textcolor{storyborder}{\rule{3pt}{12pt}}\hspace{6pt}
\textbf{من تجاربي:}  "العين هي أقوى جهاز تقييم. الأرقام والتحاليل مهمة، لكن ملاحظاتكم اليومية أهم. لذلك سجلوا اللي تلاحظون ."
\vspace{8pt}

\section*{المقدمة}

أذكر زيارة لمريض في الستينات من عمره، الأسرة مطمئنة: "الحمد لله، كل شي تمام." لكن لما دخلت، أول شيء لاحظته - كان جالس على طرف الكنب، رجوله متباعدة كثير، ومتوتر. سألته: "كيف حالك؟"، قال: "بخير الحمد لله، بس أحس بخوف لما أبغى أقوم."

هذه الملاحظة البسيطة - طريقة الجلوس والتوتر في الوجه - كشفت مشكلة خوف من السقوط ما أحد من الأهل منتبه لها. لو كانت الأسرة تعرف كيف تلاحظ، كانوا اكتشفوا المشكلة قبلي بأسابيع.

\textbf{هذا بالضبط هدف هذا الفصل:} أن أعلمك كيف تقرأ حالة كبيرك في البيت بعين أخصائي علاج طبيعي. ما تحتاج أجهزة أو شهادات - فقط عين ملاحظة ومعرفة بسيطة بما تبحث عنه.

\section*{لماذا المراقبة المنزلية مهمة؟}

\subsection*{الكشف المبكر ينقذ}

من واقع سنوات

 أقولها بكل ثقة: \textbf{التغيرات التدريجية الصغيرة هي التي تفوتنا.

تخيل معي: والدك كان يقوم من الكرسي في 5 ثوانٍ. بعد شهر صار يحتاج 7 ثوانٍ. بعد شهرين 10 ثوانٍ. بعد ثلاثة شهور 15 ثانية ويحتاج يدعم بيديه. التغيير يومياً صغير جداً، لكن الصورة الكاملة خطيرة.

الأسرة التي تعيش مع المريض يومياً قد لا تلاحظ هذا التدهور التدريجي [1]. لكن بالمراقبة المنهجية والتسجيل، تصير الصورة واضحة.

\textbf{أمثلة شفتها بعيني:} - أم ما انتبهت إن والدها صار يمشي أبطأ، لين زارته أختها بعد شهر وقالت: "أبوي تغير!" - أب ما لاحظ إن أمه صارت تطلب مساعدة في الحمام أكثر، لين سجلنا ولقينا الطلبات زادت من مرة في الأسبوع لمرتين يومياً - عائلة ما انتبهت إن الوالد "يتعثر" كثير، لين سألتهم: "كم مرة بالأسبوع؟"، قالوا: "تقريباً كل يوم!" - هذي علامة خطر كبيرة!

\subsection*{المعلومات الدقيقة تساعد المختص}

غالباً لما أزور مريض أول مرة، أسأل الأسرة: "كيف حاله في البيت؟"

الإجابات اللي أسمعها: - "الحمد لله بخير" ← طيب، بخير يعني إيش بالضبط؟ - "عادي مثل أي شخص في سنه" ← لكن ما عطيتني معلومة محددة! - "مش عارفين بالضبط" ← هذي صعبة عليّ لأني ما أقدر أبني خطة علاجية على "مش عارفين"

\textbf{الإجابات التي تساعدني فعلاً:}

عندما تقول لي الأسرة: "صار يحتاج أن يمسك الكرسي بيديه الاثنتين ليقوم، وقبل كان يقوم بيد واحدة" --- هذه معلومة قيّمة جداً. أو حين يصفون مشيته: "صارت أبطأ، وأحياناً يجر رجله اليمنى قليلاً" --- هنا أبدأ أفهم موضع المشكلة. والأهم من ذلك حين يحددون: "سقط مرتين هذا الأسبوع، مرة في الحمام بعد الاستحمام، ومرة في الممر الساعة الثانية فجراً" --- التوقيت والمكان يكشفان السبب غالباً.

هذه المعلومات ذهب! تخليني أفهم المشكلة بالضبط وأضع خطة دقيقة [3].

\section*{ملاحظة الجلوس والقيام من الكرسي}

\subsection*{لماذا هذا الأول شيء أراقبه؟}

من أول لحظة أدخل فيها بيت المريض، عيني على حركة واحدة: \textbf{كيف يقوم من مكانه ليستقبلني؟}

هذه الحركة البسيطة تكشف الكثير: - قوة عضلات الفخذين والمقعدة - التوازن - الثقة بالنفس - وجود ألم - الخوف من الحركة

القيام من الكرسي مهارة نستخدمها عشرات المرات يومياً - من السرير، من المرحاض، من كرسي الصلاة [4]. إذا ضعفت هذه المهارة، حياة كبيرك كلها تتأثر.

\subsection*{ما الذي أريدك تلاحظه؟}

\textbf{1. استخدام اليدين:}

راقب والدك وهو يقوم من الكرسي:
\begin{itemize}
\item
     \textbf{ممتاز:} يقوم بدون ما يستخدم يديه نهائياً (قوة عضلية ممتازة)
  
\item
     \textbf{متوسط:} يستخدم يد واحدة دفعة خفيفة (ضعف خفيف)
  
\item
    \textbf{ضعيف:} يدفع بقوة بكلتا يديه، أو يحتاج شخص يساعده [5]
  
\end{itemize}

أذكر عبدالرحمن (اسم مستعار)، رجل في السبعين. أول زيارة، قام بيد واحدة. بعد شهرين، صار يدفع بيديه الاثنتين. هذا تدهور واضح - بدأنا تمارين تقوية فوراً.

\textbf{2. عدد المحاولات:}
\begin{itemize}
\item
    هل ينهض من أول محاولة؟
  
\item
    ولا يحاول، يتوقف في النص، يرجع يجلس، ثم يحاول مرة ثانية؟
  
\item
    هذا التردد يدل على ضعف أو خوف
  
\end{itemize}

\textbf{3. السرعة:}

غالباً ما أطلب من الأسرة: "راقبوا كم ثانية يحتاج للنهوض."
\begin{itemize}
\item
    \textbf{القيام السريع السلس:} قوة جيدة
  
\item
    \textbf{البطء الشديد (أكثر من 5 ثوانٍ):} ضعف أو ألم أو خوف
  
\end{itemize}

\textbf{4. التوازن:}

لاحظ بعد ما يقوم: - هل يترنح أو يتمايل؟ - هل يحتاج ثوانٍ ليستقر؟ - هل يمسك بشيء فوراً؟

كل هذي علامات ضعف توازن.

\textbf{5. تعبيرات الوجه:}

الوجه ما يكذب: - عبوس وتكشير → ألم - احمرار واضح → جهد كبير - خوف في العينين → قلق من السقوط

\subsection*{اختبار بسيط تقدر تسويه في البيت}

\textbf{اختبار القيام خمس مرات متتالية:}

دعني أشارك معك اختبار بسيط استخدمه مع جميع مرضاي:

\begin{enumerate}
\item
    اجلس كبيرك على كرسي عادي (الركبة بزاوية قائمة)
  
\item
    اطلب منه يقوم ويجلس 5 مرات متتالية، بأسرع ما يقدر (لكن بأمان!)
  
\item
    احسب الوقت من بداية أول نهوض إلى آخر جلوس
  
\end{enumerate}

\textbf{النتائج [6]: - \textbf{أقل من 12 ثانية:} ماشاء الله، قوة ممتازة! - \textbf{12-15 ثانية:} جيد، لكن نقدر نحسّن - \textbf{أكثر من 15 ثانية:} ضعف واضح - تحتاجون استشارة وتمارين تقوية

 \textbf{مهم:} لا تسوي هذا الاختبار إذا كان كبيرك عنده دوار شديد أو ألم حاد في الركبتين/الوركين. اسأل المختص أولاً.

 \vspace{8pt}\noindent\textcolor{tipborder}{\rule{3pt}{12pt}}\hspace{6pt}
\textbf{نصيحة من الميدان:} 

أقولها دائماً للعائلات: \textbf{"القوة تُبنى بالتدريج، وكل محاولة للقيام من الكرسي هي تمرين صغير."
\vspace{8pt}

رفع ارتفاع الكرسي بوسادة صلبة أو تعديله قليلاً يقلل المجهود بشكل كبير. شفت عائلات كثيرة، هذا التعديل البسيط أعاد لكبيرهم القدرة على النهوض بنفسه - وكم كانت فرحتهم كبيرة!

\section*{مراقبة نمط المشي}

\subsection*{المشي الطبيعي: ماذا نتوقع؟}

المشي الطبيعي له مواصفات معينة [7]: - طول الخطوة متساوٍ بين القدمين - المسافة بين القدمين (عرض القاعدة) 5-10 سم - سرعة معتدلة ومنتظمة - الذراعين تتأرجح بشكل طبيعي مع الخطوات - النظر للأمام، مش للأرض طول الوقت}

لكن مع التقدم بالعمر أو بعض الأمراض، المشي يتغير. وكل نمط مشي "غير طبيعي" له معنى.

\subsection*{أنماط المشي اللي تحتاج انتباه}

\textbf{1. مشية القدم المسحوبة (Shuffling):}

\textbf{كيف تبين:} - خطوات قصيرة جداً - القدم ما ترتفع عن الأرض (يجرها) - تسمع صوت احتكاك القدم بالأرض

\textbf{قصة حقيقية:}

زرت مريض في الرياض، الأهل يقولون "عادي يمشي". لما طلبت منه يمشي قدامي 5 أمتار، لاحظت رجوله تنسحب على الأرض ويمشي بخطوات صغيرة جداً. سألته: "تحس برعشة؟"، قال: "أيه، خفيفة في يدي."

أرسلته لطبيب أعصاب، تشخيصه مرحلة مبكرة من باركنسون [8]. لو ما انتبهنا للمشية، كان التشخيص تأخر أكثر.

\textbf{ما تسوي الأسرة:} - تأكد الحذاء خفيف ومريح - راقب علامات ثانية (رعشة، بطء) - استشر الطبيب

\textbf{2. مشية الحذر (Cautious Gait):}

\textbf{كيف تبين:} - خطوات قصيرة جداً وبطيئة - المسافة بين القدمين واسعة (لزيادة الثبات) - النظر للأسفل طول الوقت - يمسك الأثاث أو الجدران

\textbf{من تجربتي:}

هذي المشية غالباً ما تعني \textbf{خوف من السقوط} [9]. كبير السن خايف يقع، فيمشي بحذر مبالغ فيه.

أذكر سيدة في السبعين، كانت تمشي عادي. بعد ما تعثرت مرة (ما سقطت حتى!)، صارت تمشي بهذا الأسلوب - خطوات صغيرة، قدمين متباعدات، نظرها للأرض. الخوف غيّر مشيتها تماماً.

\textbf{ما تسوي الأسرة:} - طمأنة وتشجيع - إزالة كل العوائق - إضاءة ممتازة - استشارة مختص لتمارين التوازن

\textbf{3. مشية غير متماثلة (Asymmetric):}

\textbf{كيف تبين:} - الخطوات مش متساوية بين الجانبين - عرج واضح على قدم/جانب - وزن الجسم مش موزع بالتساوي

\textbf{متى شفتها:}

كثير من مرضى ما بعد الجلطة الدماغية يمشون كذا. جانب أقوى من الثاني [10].

لكن أحياناً السبب بسيط: ألم في ركبة أو ورك. سألت مريض: "ليش تعرج؟"، قال: "ركبتي اليمين توجعني." - علاجنا الألم، تحسنت المشية.

\textbf{ما تسوي الأسرة:} - حدد أي جانب أضعف - راقب متى العرج يزيد (بعد الجلوس؟ الصباح؟) - استشر الطبيب لتحديد السبب

\textbf{4. مشية القدم المتدلية (Steppage):}

\textbf{كيف تبين:} - يرفع الركبة عالي جداً في كل خطوة - كأنه يصعد درج وهو ماشي على أرض مستوية - القدم "تصفع" الأرض لما تنزل

\textbf{هذي حالة خاصة:}

رفع الركبة العالي هذا يعني عضلات رفع القدم ضعيفة، فهو يعوّض برفع الركبة كلها [11].

\textbf{حالة شفتها:} رجل بعد عملية في الظهر، صار ما يقدر يرفع قدمه اليسرى. كان يرفع ركبته عالي جداً علشان ما يجر قدمه.

هذي تحتاج تقييم طبي عاجل وربما جبيرة لدعم القدم.

\section*{علامات ضعف التوازن}

\subsection*{لماذا التوازن حياة أو موت؟}

بكل صراحة: \textbf{ضعف التوازن = خطر سقوط. والسقوط عند كبار السن قد يعني كسور، عجز، أو أسوأ.

من واقع شغلي، السقطة الواحدة قد تغير حياة كبير السن كاملة. شفت مرضى بعد كسر الورك ما رجعوا يمشون زي أول [12].

لذلك، اكتشاف ضعف التوازن مبكراً \textbf{قد ينقذ حياة.

\subsection*{العلامات اللي تشوفها بعينك}

\textbf{1. التمسك بالأشياء (Furniture Walking):}

أول شيء أراقبه لما أدخل البيت: كيف يتحرك داخل البيت؟
\begin{itemize}
\item
    هل يمسك الأثاث أثناء المشي؟
  
\item
    حتى لو "لمسة خفيفة"، هذا يعني توازنه مش 100\%
  
\item
    كبير السن اللي توازنه جيد ما يحتاج يمسك شي
  
\end{itemize}

\textbf{قصة:}

مريضة في الستينات، الأسرة ما انتبهت لشي. لما زرتها، لاحظت إنها تمسك كل شي - طرف الكنب، الطاولة، الجدار. سألتهم: "هي دايماً كذا؟"، قالوا: "أيه، عادي!"

هذا مش عادي! هذا علامة ضعف توازن واضحة.

\textbf{2. توسيع قاعدة المشي:}

لما تشوف كبيرك واقف أو ماشي، لاحظ المسافة بين قدميه: - إذا واسعة جداً (أكثر من 15-20 سم)، جسمه يحاول يزيد الثبات تلقائياً - هذي آلية تعويضية

\textbf{3. النظر للأسفل باستمرار:}

عيني على قدمي طول الوقت = ما يثق بتوازنه

المفروض يمشي وهو ناظر قدامه، مش طالع في الأرض!

\textbf{4. التردد في الحركة:}
\begin{itemize}
\item
    يتوقف قبل ما يدور أو يغير اتجاه
  
\item
    يخطط ذهنياً قبل كل حركة
  
\item
    بطء شديد حتى في بيئة آمنة
  
\end{itemize}

كل هذا خوف وضعف توازن.

\subsection*{اختبارات بسيطة (بحذر!)}

 \textbf{مهم جداً:} سوِّ هذي الاختبارات فقط بوجود شخص جنب كبيرك يحميه لو فقد توازنه!

\textbf{اختبار 1: الوقوف بقدمين متلاصقتين}
\begin{itemize}
\item
    اطلب منه يقف بقدمين متلاصقتين تماماً
  
\item
    بدون ما يمسك أي شي
  
\item
    لمدة 30 ثانية
  
\end{itemize}

\textbf{النتيجة [13]: - وقف بثبات ← توازن جيد - تمايل كثير أو فتح قدميه ← ضعف خفيف - ما قدر يقف أصلاً ← ضعف واضح

\textbf{اختبار 2: الوقوف على قدم واحدة}

هذا اختبار أسويه دائماً:
\begin{itemize}
\item
    قف على قدم واحدة
  
\item
    القدم الثانية مرفوعة شوي
  
\item
    احسب كم ثانية يثبت
  
\end{itemize}

\textbf{النتائج [14]: - \textbf{10 ثوانٍ أو أكثر:} ممتاز! - \textbf{5-10 ثوانٍ:} متوسط - \textbf{أقل من 5 ثوانٍ:} خطر سقوط مرتفع - محتاج تدخل

 \textbf{لا تسوي هذا الاختبار إذا:} - كبيرك عنده دوار شديد - سقط مؤخراً - عنده ألم حاد - يستخدم مشاية أو كرسي متحرك

\section*{استخدام اليدين كعلامة ضعف}

\subsection*{ليش ننتبه لهذا؟}

غالباً الأسر تقول: "أبوي كسلان، يبغى يدعم على يديه علشان يقوم!"

لا يا عزيزي، \textbf{هذا مش كسل - هذا ضعف عضلي حقيقي} [15].

استخدام اليدين للنهوض آلية تعويضية طبيعية لما عضلات الفخذين والمقعدة تضعف. جسمه يستخدم قوة الذراعين لتعويض النقص.

\subsection*{وين تلاحظ استخدام اليدين؟}

\textbf{1. النهوض من الكرسي:} - يدفع بقوة على المساند بكلتا يديه - أحياناً يدفع على ركبتيه أول، ثم على الكرسي

\textbf{2. النهوض من السرير:} - يتقلب على جنبه - يدفع بيديه على المرتبة - يجلس على طرف السرير قبل ما يقوم

\textbf{3. صعود الدرج:} - يتمسك بالدرابزين بإحكام بكلتا يديه - يسحب نفسه بقوة - يصعد درجة درجة (قدم بعد قدم)

\textbf{4. النهوض من المرحاض:} - يحتاج مقابض - يدفع بقوة كبيرة

\subsection*{وش يعني هذا؟}

\begin{longtable}[]{@{}
\tableheadercolor
  >{\raggedright\arraybackslash}p{(\columnwidth - 4\tabcolsep) * \real{0.3333}}
  >{\raggedright\arraybackslash}p{(\columnwidth - 4\tabcolsep) * \real{0.3333}}
  >{\raggedright\arraybackslash}p{(\columnwidth - 4\tabcolsep) * \real{0.3333}}@{}}
\toprule\noalign{}
\begin{minipage}[b]{\linewidth}\raggedright
السلوك
\end{minipage} & \begin{minipage}[b]{\linewidth}\raggedright
المعنى
\end{minipage} & \begin{minipage}[b]{\linewidth}\raggedright
وش تسوي؟
\end{minipage} \\
\begin{minipage}[b]{\linewidth}\raggedright
استخدام خفيف (للتوازن بس)
\end{minipage} & \begin{minipage}[b]{\linewidth}\raggedright
ضعف خفيف
\end{minipage} & \begin{minipage}[b]{\linewidth}\raggedright
تمارين وقائية خفيفة
\end{minipage} \\
\begin{minipage}[b]{\linewidth}\raggedright
دفع قوي بكلتا اليدين
\end{minipage} & \begin{minipage}[b]{\linewidth}\raggedright
ضعف واضح
\end{minipage} & \begin{minipage}[b]{\linewidth}\raggedright
تقييم + تمارين مستهدفة [16]
\end{minipage} \\
\begin{minipage}[b]{\linewidth}\raggedright
ما يقدر ينهض حتى بيديه
\end{minipage} & \begin{minipage}[b]{\linewidth}\raggedright
ضعف شديد
\end{minipage} & \begin{minipage}[b]{\linewidth}\raggedright
تدخل عاجل + أدوات مساعدة
\end{minipage} \\
\midrule\noalign{}
\endhead
\bottomrule\noalign{}
\endlastfoot
\end{longtable}

\section*{سلوكيات تدل على الخوف من السقوط}

\subsection*{الخوف أخطر من السقوط نفسه!

خلني أشارك معك حقيقة تعلمتها من الميدان:

\textbf{الخوف من السقوط يؤدي لقلة الحركة → قلة الحركة تضعف العضلات والتوازن → الضعف يزيد احتمال السقوط الفعلي!

دائرة مفرغة خطيرة [17].

\subsection*{السلوكيات اللي تكشف الخوف}

\textbf{1. تجنب الأنشطة:}

أذكر رجل في السبعينات، كان اجتماعي ويحب الزيارات. بعد ما تعثر مرة (ما سقط حتى!), صار يرفض يطلع من البيت. السبب؟ خوف.
\begin{itemize}
\item
    رفض الخروج
  
\item
    تجنب الاستحمام
  
\item
    رفض صعود الدرج حتى لو يقدر
  
\item
    تجنب المناسبات
  
\end{itemize}

\textbf{2. طلب المساعدة المفرط:}
\begin{itemize}
\item
    يطلب مساعدة حتى في أشياء يقدر يسويها
  
\item
    يتصل بأهله لأبسط الأمور
  
\item
    يرفض يتحرك بدون مرافق
  
\end{itemize}

\textbf{3. التمسك الدائم:}
\begin{itemize}
\item
    يمسك بكل شي حواليه
  
\item
    يمشي دائماً جنب الجدار
  
\item
    يرفض يمشي في مناطق مفتوحة
  
\end{itemize}

\textbf{4. البطء المبالغ:}
\begin{itemize}
\item
    حركة بطيئة جداً حتى في بيئة آمنة 100\%
  
\item
    يحتاج وقت طويل لأبسط حركة
  
\item
    تردد واضح قبل كل خطوة
  
\end{itemize}

\subsection*{كيف نتعامل مع الخوف؟}

\textbf{من تجربتي:}

\textbf{1. التفهم والطمأنة:}

لا تستهزئ! لا تقول "مافي شي، توكل على الله وامش!"

قل: "أفهم قلقك، وهذا طبيعي. لكن نقدر نشتغل عليه سوا."

\textbf{2. تحسين البيئة:}
\begin{itemize}
\item
    أزل كل المخاطر
  
\item
    إضاءة ممتازة
  
\item
    مقابض في الأماكن المهمة
  
\end{itemize}

لما يشوف كبيرك إن البيت آمن، خوفه يقل.

\textbf{3. التدرج:}

ابدأ بتمارين بسيطة جداً في بيئة آمنة. زد الصعوبة شوي شوي. احتفل بكل تقدم.

\textbf{4. الاستشارة:}

بعض الحالات تحتاج تدخل نفسي [18] أو برنامج علاج طبيعي متخصص.

\section*{أهمية التسجيل والمتابعة}

\subsection*{ليش أصر على التسجيل؟}

أقولها في كل زيارة: \textbf{"سجلوا اللي تشوفونه!"}

السبب بسيط:

\textbf{1. الذاكرة تخون:} - "متى آخر مرة سقط؟" → "ما أذكر بالضبط، قبل أسبوع ولا عشرة أيام؟" - لو كان مسجل: "يوم الثلاثاء 28/11 الساعة 3 العصر في الحمام"

\textbf{2. التغيرات التدريجية:} - يومياً التغيير صغير وما يبين - لما تقارن الأسبوع الماضي بهذا الأسبوع، الفرق واضح

\textbf{3. تقييم التحسن:} - هل التمارين نفعت؟ - هل صار يحتاج مساعدة أقل أو أكثر؟

\subsection*{وش نسجل؟}

\textbf{دفتر بسيط فيه:}

\textbf{1. القدرات اليومية:} - "نهض من الكرسي بمفرده؟ نعم/لا" - "احتاج مساعدة في الحمام؟ نعم/لا" - "كم مرة طلب مساعدة للقيام؟ \_\_\_\_ مرات"

\textbf{2. حوادث:} - التاريخ والوقت - المكان - السبب الظاهر - الإصابات (إن وجدت)

\textbf{3. التغيرات:} - "بدأ يمشي أبطأ" - "صار يحتاج وقت أطول للنهوض"

\textbf{4. الألم:} - "يشتكي من ألم في الركبة اليمنى عند القيام"

\subsection*{نموذج جدول بسيط}

\begin{longtable}[]{@{}
\tableheadercolor
  >{\raggedright\arraybackslash}p{(\columnwidth - 6\tabcolsep) * \real{0.2500}}
  >{\raggedright\arraybackslash}p{(\columnwidth - 6\tabcolsep) * \real{0.2500}}
  >{\raggedright\arraybackslash}p{(\columnwidth - 6\tabcolsep) * \real{0.2500}}
  >{\raggedright\arraybackslash}p{(\columnwidth - 6\tabcolsep) * \real{0.2500}}@{}}
\toprule\noalign{}
\begin{minipage}[b]{\linewidth}\raggedright
التاريخ
\end{minipage} & \begin{minipage}[b]{\linewidth}\raggedright
النشاط
\end{minipage} & \begin{minipage}[b]{\linewidth}\raggedright
الملاحظة
\end{minipage} & \begin{minipage}[b]{\linewidth}\raggedright
مستوى المساعدة
\end{minipage} \\
\begin{minipage}[b]{\linewidth}\raggedright
1/12
\end{minipage} & \begin{minipage}[b]{\linewidth}\raggedright
القيام
\end{minipage} & \begin{minipage}[b]{\linewidth}\raggedright
استخدم يديه
\end{minipage} & \begin{minipage}[b]{\linewidth}\raggedright
بدون مساعدة
\end{minipage} \\
\begin{minipage}[b]{\linewidth}\raggedright
1/12
\end{minipage} & \begin{minipage}[b]{\linewidth}\raggedright
المشي للحمام
\end{minipage} & \begin{minipage}[b]{\linewidth}\raggedright
مسك الجدار
\end{minipage} & \begin{minipage}[b]{\linewidth}\raggedright
بدون مساعدة
\end{minipage} \\
\begin{minipage}[b]{\linewidth}\raggedright
2/12
\end{minipage} & \begin{minipage}[b]{\linewidth}\raggedright
صعود السرير
\end{minipage} & \begin{minipage}[b]{\linewidth}\raggedright
احتاج يدي
\end{minipage} & \begin{minipage}[b]{\linewidth}\raggedright
مساعدة خفيفة
\end{minipage} \\
\begin{minipage}[b]{\linewidth}\raggedright
3/12
\end{minipage} & \begin{minipage}[b]{\linewidth}\raggedright
تعثر
\end{minipage} & \begin{minipage}[b]{\linewidth}\raggedright
بسجادة، ما سقط
\end{minipage} & \begin{minipage}[b]{\linewidth}\raggedright
-
\end{minipage} \\
\midrule\noalign{}
\endhead
\bottomrule\noalign{}
\endlastfoot
\end{longtable}

\textbf{رسالة طمأنة}

أعرف إن الموضوع قد يبدو كثير ومعقد. لكن صدقني، بعد أسبوع أو أسبوعين من الملاحظة الواعية، راح يصير عندك "عين أخصائي" طبيعية. راح تلاحظ التغيرات تلقائياً.

أنت أفضل شخص يقدر يراقب كبيرك - لأنك معه يومياً. الأطباء والأخصائيين نشوفه ساعة كل كم أسبوع. أنت الشريك الأساسي في الرعاية.

\section*{خلاصة الفصل}

مراقبة الحالة الوظيفية في البيت مهارة ثمينة تنقذ من مشاكل كثيرة:

 \textbf{ملاحظة الجلوس والقيام} - أول وأهم شي تراقبه، يكشف القوة والتوازن

 \textbf{مراقبة نمط المشي} - كل نمط له معنى ويوجهك للمشكلة

 \textbf{علامات ضعف التوازن} - الاكتشاف المبكر ينقذ من سقوط خطير

 \textbf{استخدام اليدين} - مش كسل، بل ضعف عضلي يحتاج تقوية

 \textbf{سلوكيات الخوف} - دائرة مفرغة تحتاج تفهم وتدرج في العلاج

 \textbf{التسجيل المنتظم} - يكشف التغيرات ويساعد الفريق الطبي

تذكر: \textbf{عينك أقوى أداة تقييم عندك. استخدمها.

في الفصل التالي، سأعلمك فن النقل الآمن - من السرير إلى الكرسي إلى الحمام - مع حماية ظهرك أنت كمرافق.

\section*{المراجع العلمية}

[1] Guralnik, J. M., et al.~(1994). A short physical performance battery assessing lower extremity function. Journal of Gerontology, 49(2), M85-M94.

[2] Stuck, A. E., et al.~(1993). Risk factors for functional status decline in community-living elderly people. Social Science \& Medicine, 37(4), 445-464.

[3] Fortinsky, R. H., et al.~(2006). Effects of functional status changes before and during hospitalization. The Journals of Gerontology Series A, 61(1), 83-87.

[4] Bohannon, R. W. (2006). Reference values for the five-repetition sit-to-stand test. Perceptual and Motor Skills, 103(1), 215-222.

[5] Lord, S. R., et al.~(2002). Sit-to-stand performance depends on sensation, speed, balance, and psychological status. The Journals of Gerontology Series A, 57(8), M539-M543.

[6] Bohannon, R. W., et al.~(2010). Five-repetition sit-to-stand test performance by community-dwelling adults. Isokinetics and Exercise Science, 18(3), 141-147.

[7] Bohannon, R. W., \& Williams Andrews, A. (2011). Normal walking speed: a descriptive meta-analysis. Physiotherapy, 97(3), 182-189.

[8] Morris, M. E., et al.~(2001). Gait disorders and gait rehabilitation in Parkinson's disease. Advances in Neurology, 87, 347-361.

[9] Verghese, J., et al.~(2010). Gait dysfunction in mild cognitive impairment syndromes. Journal of the American Geriatrics Society, 58(7), 1263-1269.

[10] Patterson, K. K., et al.~(2008). Gait asymmetry in community-ambulating stroke survivors. Archives of Physical Medicine and Rehabilitation, 89(2), 304-310.

[11] Stewart, J. D. (2008). Foot drop: where, why and what to do? Practical Neurology, 8(3), 158-169.

[12] Shumway-Cook, A., \& Woollacott, M. H. (2016). Motor control: translating research into clinical practice. Lippincott Williams \& Wilkins.

[13] Khasnis, A., \& Gokula, R. M. (2003). Romberg's test. Journal of Postgraduate Medicine, 49(2), 169.

[14] Vellas, B. J., et al.~(1997). One-leg balance is an important predictor of injurious falls. Journal of the American Geriatrics Society, 45(6), 735-738.

[15] Alexander, N. B., et al.~(2000). Chair and bed rise performance in ADL-impaired congregate housing residents. Journal of the American Geriatrics Society, 48(5), 526-533.

[16] Hughes, M. A., et al.~(1996). The role of strength in rising from a chair. Journal of Biomechanics, 29(12), 1509-1513.

[17] Friedman, S. M., et al.~(2002). Falls and fear of falling: which comes first? Journal of the American Geriatrics Society, 50(8), 1329-1335.

[18] Delbaere, K., et al.~(2010). The falls efficacy scale international (FES-I). Age and Ageing, 39(2), 210-216.

\emph{ملاحظة: جميع الأسماء والقصص الواردة في هذا الكتاب حقيقية، لكن تم تغيير الأسماء والتفاصيل الشخصية للحفاظ على خصوصية المرضى.
