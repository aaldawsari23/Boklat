\chapter*{مسرد المصطلحات الطبية}
\addcontentsline{toc}{chapter}{مسرد المصطلحات الطبية}

\vspace{0.5cm}

\textbf{تعريف:} هذا المسرد يحتوي على أهم المصطلحات الطبية والتأهيلية الواردة في الكتاب، مع شرح مبسّط لكل مصطلح لتسهيل الفهم على الأسر ومقدمي الرعاية.

\vspace{1cm}

\section*{حرف الألف — أ}

\textbf{أنشطة الحياة اليومية (ADLs):}
الأنشطة الأساسية التي يقوم بها الإنسان يومياً مثل الأكل، الشرب، الاستحمام، ارتداء الملابس، والتنقل.

\textbf{إعادة التأهيل (Rehabilitation):}
عملية منظّمة تهدف لاستعادة القدرات الحركية والوظيفية بعد الإصابة أو المرض.

\textbf{الاستقلالية الوظيفية (Functional Independence):}
قدرة الشخص على أداء مهامه اليومية دون مساعدة الآخرين.

\textbf{الانقباض (Contracture):}
تيبّس دائم في المفصل يحدث نتيجة قلة الحركة لفترة طويلة، مما يؤدي لتقصير العضلات والأوتار.

\textbf{الالتهاب الرئوي (Pneumonia):}
عدوى تصيب الرئتين، شائعة عند المرضى طريحي الفراش بسبب تراكم السوائل وضعف التنفس.

\vspace{0.7cm}

\section*{حرف الباء — ب}

\textbf{برنامج التمارين المنزلية (Home Exercise Program):}
مجموعة من التمارين المصممة خصيصاً للمريض ليؤديها في المنزل تحت إشراف الأسرة.

\textbf{البتر (Amputation):}
إزالة جراحية لجزء من الطرف (ذراع أو ساق) نتيجة إصابة أو مرض.

\vspace{0.7cm}

\section*{حرف التاء — ت}

\textbf{التوازن الثابت (Static Balance):}
القدرة على الحفاظ على الاستقرار أثناء الوقوف أو الجلوس دون حركة.

\textbf{التوازن الديناميكي (Dynamic Balance):}
القدرة على الحفاظ على الاستقرار أثناء الحركة مثل المشي أو تغيير الاتجاه.

\textbf{التشنج العضلي (Spasticity):}
زيادة غير طبيعية في توتر العضلات، شائعة بعد الجلطات الدماغية أو إصابات الحبل الشوكي.

\textbf{تمارين المدى الحركي (Range of Motion Exercises):}
تمارين تهدف للحفاظ على مرونة المفاصل ومنع التيبّس.

\textbf{التهاب الوريد الخثاري (Thrombophlebitis):}
التهاب في جدار الوريد مع تكوّن جلطة دموية، خطر شائع عند المرضى قليلي الحركة.

\vspace{0.7cm}

\section*{حرف الجيم — ج}

\textbf{الجلطة الدماغية (Stroke / CVA):}
انقطاع تدفق الدم لجزء من الدماغ، مما يسبب تلفاً في خلايا المخ وفقداناً للوظائف الحركية أو الحسية.

\textbf{جهاز المشاية (Walker):}
جهاز مساعد على المشي يوفر دعماً ثابتاً ومتوازناً، مناسب لكبار السن وضعيفي التوازن.

\vspace{0.7cm}

\section*{حرف الحاء — ح}

\textbf{الحمل الجزئي على الطرف (Partial Weight Bearing):}
قدرة المريض على وضع جزء فقط من وزنه على الطرف المصاب أثناء المشي.

\vspace{0.7cm}

\section*{حرف الخاء — خ}

\textbf{خشونة المفاصل (Osteoarthritis):}
تآكل تدريجي في الغضروف المحيط بالمفصل، يسبب ألماً وتيبساً وصعوبة في الحركة.

\textbf{الخمول البدني (Physical Inactivity):}
قلة الحركة والنشاط البدني، عامل خطر رئيسي لضعف العضلات وأمراض القلب.

\vspace{0.7cm}

\section*{حرف الدال — د}

\textbf{الدوار (Vertigo):}
شعور بأن المحيط يدور أو أن الشخص نفسه يدور، قد يكون مؤشراً على مشاكل في الأذن الداخلية أو الدماغ.

\vspace{0.7cm}

\section*{حرف الراء — ر}

\textbf{الرعاية الصحية المنزلية (Home Healthcare):}
خدمات طبية وتمريضية وتأهيلية تُقدّم للمريض في منزله بدلاً من المستشفى.

\textbf{الرعاية المديدة (Long-term Care):}
رعاية طبية وتمريضية مستمرة للمرضى الذين يحتاجون مساعدة طويلة الأمد في أنشطة الحياة اليومية.

\vspace{0.7cm}

\section*{حرف السين — س}

\textbf{السقوط (Fall):}
فقدان التوازن والهبوط على الأرض بشكل غير مقصود، خطر صحي كبير لكبار السن.

\textbf{سوء التغذية (Malnutrition):}
نقص أو عدم توازن في العناصر الغذائية الضرورية، شائع عند كبار السن وطريحي الفراش.

\textbf{الساركوبينيا (Sarcopenia):}
فقدان تدريجي في كتلة العضلات وقوتها مع التقدم في العمر.

\vspace{0.7cm}

\section*{حرف الشين — ش}

\textbf{الشرقة / الاختناق (Choking / Aspiration):}
دخول الطعام أو السوائل إلى مجرى التنفس بدلاً من المريء، خطر صحي خطير قد يسبب الاختناق أو الالتهاب الرئوي.

\textbf{الشلل النصفي (Hemiplegia):}
فقدان كامل للحركة في نصف الجسم (ذراع وساق) عادةً بعد جلطة دماغية.

\textbf{الشد العضلي (Muscle Strain):}
تمزّق صغير في ألياف العضلة نتيجة حركة مفاجئة أو تمرين مفرط.

\vspace{0.7cm}

\section*{حرف الصاد — ص}

\textbf{الصلابة / التيبّس (Stiffness):}
صعوبة في تحريك المفصل بسلاسة، غالباً بسبب قلة الحركة أو التهاب المفاصل.

\vspace{0.7cm}

\section*{حرف الضاد — ض}

\textbf{الضمور العضلي (Muscle Atrophy):}
انكماش وضعف في العضلات نتيجة قلة الاستخدام أو مرض عصبي.

\textbf{ضغط الدم الانتصابي (Orthostatic Hypotension):}
انخفاض مفاجئ في ضغط الدم عند الوقوف، يسبب دواراً وخطر السقوط.

\vspace{0.7cm}

\section*{حرف الطاء — ط}

\textbf{طريح الفراش (Bedridden):}
شخص غير قادر على مغادرة السرير بسبب مرض أو إعاقة شديدة.

\textbf{الطرف الاصطناعي (Prosthesis):}
جهاز صناعي يستخدم لتعويض طرف مبتور.

\vspace{0.7cm}

\section*{حرف العين — ع}

\textbf{العلاج الطبيعي (Physical Therapy / Physiotherapy):}
تخصص طبي يستخدم التمارين، الحركة، والتقنيات اليدوية لاستعادة الوظائف الحركية وتخفيف الألم.

\textbf{العلاج الوظيفي (Occupational Therapy):}
تخصص يساعد المرضى على استعادة القدرة على أداء أنشطتهم اليومية بأمان واستقلالية.

\textbf{العصا (Cane):}
جهاز مساعد بسيط على المشي، يوفر دعماً جزئياً ويحسّن التوازن.

\textbf{العصا رباعية القوائم (Quad Cane):}
عصا لها أربع قوائم في القاعدة، توفر استقراراً أكبر من العصا العادية.

\vspace{0.7cm}

\section*{حرف الغين — غ}

\textbf{الغرغرينا (Gangrene):}
موت الأنسجة نتيجة انقطاع تدفق الدم، خطر خطير عند مرضى السكري وقرح الضغط المهملة.

\vspace{0.7cm}

\section*{حرف القاف — ق}

\textbf{قرحة الضغط (Pressure Ulcer / Bedsore):}
جرح في الجلد ينشأ نتيجة الضغط المستمر على منطقة معينة، خطر شائع عند طريحي الفراش.

\textbf{القوة العضلية (Muscle Strength):}
قدرة العضلة على إنتاج قوة ضد مقاومة.

\textbf{قياس درجة الاستقلالية (Independence Level):}
تقييم قدرة المريض على أداء المهام اليومية دون مساعدة، يُستخدم لمتابعة التقدم.

\vspace{0.7cm}

\section*{حرف الكاف — ك}

\textbf{الكرسي المتحرك (Wheelchair):}
كرسي ذو عجلات يستخدمه الأشخاص غير القادرين على المشي للتنقل بأمان واستقلالية.

\textbf{كسر الورك (Hip Fracture):}
كسر في أعلى عظمة الفخذ، شائع عند كبار السن وخطير جداً.

\textbf{الكتلة العضلية (Muscle Mass):}
حجم العضلات في الجسم، تنخفض مع التقدم في العمر وقلة الحركة.

\vspace{0.7cm}

\section*{حرف الميم — م}

\textbf{المشاية ذات العجلات (Rollator):}
مشاية لها عجلات ومكابح، تسهّل الحركة وتقلل الجهد مقارنة بالمشاية التقليدية.

\textbf{المدى الحركي (Range of Motion):}
مقدار الحركة المتاح في مفصل معين، يُقاس بالدرجات.

\textbf{مركز الثقل (Center of Gravity):}
النقطة التي يتوزع حولها وزن الجسم بالتساوي، تغيّره يؤثر على التوازن.

\textbf{المقاومة التدريجية (Progressive Resistance):}
زيادة تدريجية في صعوبة التمارين لتحسين القوة العضلية بأمان.

\vspace{0.7cm}

\section*{حرف النون — ن}

\textbf{النقل الآمن (Safe Transfer):}
تقنيات صحيحة لنقل المريض من السرير للكرسي أو العكس دون إصابة.

\textbf{نوبة نقص تروية عابرة (TIA):}
"جلطة صغيرة" مؤقتة تسبب أعراضاً عصبية لفترة قصيرة ثم تزول، علامة تحذيرية لجلطة كبيرة محتملة.

\vspace{0.7cm}

\section*{حرف الهاء — هـ}

\textbf{هشاشة العظام (Osteoporosis):}
ضعف وترقق في العظام يزيد من خطر الكسور، شائع عند النساء بعد سن اليأس.

\textbf{الهذيان (Delirium):}
حالة من الارتباك الحاد وتغيّر مفاجئ في الوعي، قد تحدث عند كبار السن بسبب عدوى أو أدوية.

\vspace{0.7cm}

\section*{حرف الواو — و}

\textbf{الوذمة (Edema):}
تجمّع السوائل في الأنسجة يسبب انتفاخاً، شائع في القدمين عند المرضى قليلي الحركة.

\textbf{الوقاية من السقوط (Fall Prevention):}
إجراءات وتعديلات بيئية تهدف لتقليل خطر السقوط عند كبار السن.

\vspace{1cm}

\begin{center}
\rule{0.5\textwidth}{0.5pt}
\end{center}

\vspace{0.5cm}

\textbf{ملاحظة:} هذا المسرد يغطي المصطلحات الأساسية الواردة في الكتاب. للمزيد من التفاصيل، يُرجى الرجوع للفصول ذات العلاقة أو استشارة الفريق الطبي.

\clearpage
