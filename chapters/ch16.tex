\chapter{الأجهزة المساندة والجبائر من منظور عملي}

\section*{مقدمة قصيرة}

الأجهزة المساندة قد تكون سببًا في أمان كبير السن واستقلاله، وقد تكون سببًا في سقوطه أو تيبّسه إذا استُخدمت بطريقة خاطئة. الفكرة بسيطة: الجهاز يضيف أمانًا فقط عندما يكون مناسبًا للحالة، مضبوطًا بشكل مريح، ويُستخدم بوعي داخل البيت.

\section*{قاعدة أساسية قبل أي شيء}

الجهاز ليس ترفًا ولا عقوبة. هو وسيلة أمان واستقلال. بعض الأسر تمنع الجهاز بحجة أنه يضعف الكبير، وبعضها تفرضه كعقوبة لأنه يتعبها. الاثنان خطأ. الجهاز يُستخدم ليحميه ويخليه ينجز أموره بأقل مخاطرة.

\section*{الأجهزة المساعدة للمشي}

\subsection*{العصا}

\textbf{متى تناسب؟}
\begin{itemize}
\item إذا كان لديه عدم توازن بسيط.
\item إذا كان الألم خفيفًا في مفصل واحد أثناء المشي.
\item إذا كان يحتاج دعمًا خفيفًا فقط داخل البيت.
\end{itemize}

\textbf{أخطاء شائعة:}
\begin{itemize}
\item حملها في الجهة الخطأ.
\item وضع العصا بعيدًا ثم التقدم بخطوة كبيرة.
\item الاعتماد عليها وهي غير ثابتة أو مطاطها تالف.
\end{itemize}

\textbf{كيف تُمسك وتُستخدم داخل البيت؟}
\begin{itemize}
\item تمسك في الجهة المقابلة للرجل الأضعف أو المؤلمة.
\item تتحرك العصا بخطوة قصيرة للأمام ثم يتحرك الجسم معها.
\item لا تُستخدم على أرضية زلقة أو أمام سجادة غير ثابتة.
\end{itemize}

\begin{storybox}
في بيت بجدة، كان الأب يستخدم العصا في نفس جهة رجله المصابة، ويستغرب أنه يتعب أكثر. لما جرّب يمسكها في الجهة المقابلة، صار المشي أسهل وأثبت.
\end{storybox}

\subsection*{المشاية (الووكر)}

\textbf{متى أفضل من العصا؟}
\begin{itemize}
\item إذا كان التوازن ضعيفًا بشكل واضح.
\item إذا كان يحتاج دعمًا من جهتين.
\item بعد سقوط متكرر أو خوف شديد من الحركة.
\end{itemize}

\textbf{أخطاء شائعة:}
\begin{itemize}
\item رفع المشاية بالكامل ثم التحرك (يُفقد الثبات).
\item الركض بها داخل البيت.
\item الاعتماد الكامل عليها مع تراجع العضلات.
\end{itemize}

\textbf{قاعدة عملية:} المشاية تتحرك قليلًا للأمام، ثم خطوة قصيرة داخل إطارها، وليس خارجها.

\subsection*{العكاز/الكراكي}

\textbf{متى تُستخدم؟}
\begin{itemize}
\item بعد إصابة أو عملية، عندما يوصي المختص بتخفيف تحميل الوزن.
\item عند وجود ضعف شديد في طرف سفلي.
\end{itemize}

\textbf{تحذيرات عامة:}
\begin{itemize}
\item استخدامها بدون تدريب قد يسبب سقوطًا أو ألم كتف.
\item لا تُستخدم داخل الحمام أو فوق العتبات دون مرافق.
\end{itemize}

\subsection*{الكرسي المتحرك}

\textbf{متى يكون حلًا مؤقتًا؟}
\begin{itemize}
\item بعد جراحة أو سقوط حين يمنع الطبيب المشي مؤقتًا.
\item عند تعب شديد يستدعي الراحة لساعات محدودة.
\end{itemize}

\textbf{متى قد يصبح سبب تراجع؟}
\begin{itemize}
\item إذا استخدمناه بدل خطوات يستطيع القيام بها.
\item إذا منعناه من الوقوف نهائيًا خوفًا من السقوط.
\end{itemize}

\textbf{قواعد أمان أساسية:}
\begin{itemize}
\item تأكد من وجود فرامل وتفعيلها قبل الوقوف أو النقل.
\item لا تتركه فوق سطح مائل دون مرافق.
\item عند عبور عتبة بسيطة، الأفضل وجود مرافق أو منحدر صغير ثابت.
\end{itemize}

\section*{مبادئ اختيار ارتفاع الجهاز بشكل آمن وغير رقمي}

\textbf{قاعدة عامة بدون أرقام:}
\begin{itemize}
\item عندما يمسك الجهاز، تكون الكتف مرتخية وليست مرفوعة.
\item الكوع يكون مثنيًا قليلًا بشكل مريح.
\item اليد لا تُشد للأعلى ولا تنزل للأسفل بشكل مبالغ.
\end{itemize}

\begin{warnbox}
الضبط النهائي عند مختص ضروري، خصوصًا لمن لديهم سقوط متكرر أو آلام كتف أو ظهر. التعديل الخاطئ قد يزيد الألم أو يغيّر طريقة المشي بشكل ضار.
\end{warnbox}

\section*{الجبائر بطريقة مبسطة}

\subsection*{جبائر الراحة/النوم}

\textbf{متى تُستخدم؟}
\begin{itemize}
\item لمنع وضعية خاطئة أثناء النوم.
\item لتهدئة شد خفيف في مفصل معين.
\end{itemize}

\textbf{متى قد تزيد التيبس؟}
\begin{itemize}
\item إذا استُخدمت لساعات طويلة دون حركة لطيفة.
\item إذا كان المفصل بحاجة حركة يومية بسيطة.
\end{itemize}

\subsection*{جبائر المشي/الدعم}

هي جبائر تُلبس أثناء الحركة لتحسين الثبات أو تقليل انحراف المفصل. فائدتها مرتبطة بالضبط الجيد والمراجعة الدورية. استخدامها بدون تقييم قد يسبب ألمًا أو يغيّر المشية بطريقة خاطئة.

\subsection*{AFO/KFO كأمثلة}

\textbf{لماذا تُستخدم؟}
\begin{itemize}
\item لتثبيت الكاحل أو الركبة عند وجود ضعف أو عدم تحكم.
\item لتقليل خطر التعثر وسحب القدم.
\end{itemize}

\textbf{متى خطر استخدامها بدون ضبط؟}
\begin{itemize}
\item إذا كانت شدّتها زائدة فتسبب ألمًا أو تغير المشي.
\item إذا كانت رخوة فتفقد فائدتها وتسبب احتكاكًا.
\end{itemize}

\subsection*{قاعدة الجلد}

أي جبيرة قد تسبب تقرحًا إذا أهملنا الجلد. افحص الجلد يوميًا: احمرار مستمر، ألم، أو سخونة غير طبيعية تحتاج توقف مؤقت ومراجعة مختص.

\section*{أخطاء قاتلة شائعة}

\begin{itemize}
\item شراء جهاز غير مناسب لأن الأسرة تعودت عليه.
\item استخدام جبيرة ساعات طويلة بدون فحص الجلد.
\item الاعتماد على الكرسي المتحرك بينما يستطيع المشي خطوات.
\item جهاز رديء، زلق، أو كفراته تالفة.
\item استخدام الجهاز داخل حمام أو أرضية مبللة بدون خطة أمان.
\end{itemize}

\section*{أمان البيت مع الأجهزة}

\textbf{أمور بسيطة تصنع فرقًا كبيرًا:}
\begin{itemize}
\item تثبيت السجاد أو إزالته في مسار المشي.
\item تقليل العتبات أو وضع شريط تنبيه واضح.
\item إضاءة قوية ليلًا خصوصًا قرب الحمام.
\item ترتيب الأثاث لفتح ممر واضح.
\end{itemize}

\textbf{كيف نجهز مسار مشي داخل البيت؟}
\begin{itemize}
\item اختر طريقًا واحدًا رئيسيًا من الغرفة للحمام والصالة.
\item أزل العوائق الصغيرة (طاولة صغيرة، سلك كهرباء).
\item ضع كرسيًا ثابتًا كنقطة راحة في منتصف الطريق إذا لزم.
\end{itemize}

\section*{متى نحتاج مختص فورًا؟}

\begin{itemize}
\item سقوط متكرر خلال أيام أو أسابيع.
\item ألم جديد بسبب الجهاز.
\item احمرار أو جرح بسبب جبيرة.
\item تدهور مفاجئ في المشي أو التوازن.
\item ضعف شديد أو شلل جديد.
\end{itemize}

\section*{خاتمة قصيرة}

هدف الجهاز ليس تقييد الكبير، بل حفظ أمانه واستقلاله وكرامته. الجهاز الصحيح في الوقت الصحيح يساوي خطوات أكثر، سقوط أقل، وراحة نفسية للأسرة كلها.

\section*{المراجع}

[1] World Health Organization. (2015). \emph{World report on ageing and health}. WHO Press.

[2] World Health Organization. (2017). \emph{Rehabilitation in health systems}. WHO.

[3] National Institute for Health and Care Excellence. (2013). \emph{Falls in older people: assessing risk and prevention} (CG161).

[4] Centers for Disease Control and Prevention. (2023). \emph{STEADI: Older Adult Fall Prevention}. CDC.

[5] Panel on Prevention of Falls in Older Persons, American Geriatrics Society and British Geriatrics Society. (2011). Summary of the updated AGS/BGS clinical practice guideline for prevention of falls in older persons. \emph{Journal of the American Geriatrics Society}, 59(1), 148--157.

[6] Hsu, J. D., Michael, J. W., \& Fisk, J. R. (Eds.). (2008). \emph{AAOS Atlas of Orthoses and Assistive Devices} (4th ed.). Mosby.

[7] World Health Organization. (2008). \emph{WHO Guidelines on the Provision of Manual Wheelchairs in Less Resourced Settings}. WHO Press.

[8] Lusardi, M. M., Jorge, M., \& Nielsen, C. C. (2013). \emph{Orthotics and Prosthetics in Rehabilitation} (3rd ed.). Elsevier.

[9] European Pressure Ulcer Advisory Panel, National Pressure Injury Advisory Panel, \& Pan Pacific Pressure Injury Alliance. (2019). \emph{Prevention and Treatment of Pressure Ulcers/Injuries: Clinical Practice Guideline}.
