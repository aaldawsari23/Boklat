\chapter{التغيّرات الذهنية والنفسية والسلوكية عند كبار السن داخل المنزل}

\section*{لماذا هذا الفصل مهم؟}

في زيارة منزلية قبل سنتين، استقبلتني ابنة مريضة عمرها 78 سنة. قالت: "أمي فجأة صارت تتكلم كلام ما نفهمه، وتصحى الليل وتقول فيه ناس في الغرفة. أمس حاولت تمشي للحمام وطاحت."

سألتها: متى بدأ هذا؟ قالت: من يومين فقط.

فحصت الأم ولاحظت حرارة خفيفة وجفاف شديد. اتصلنا بالطبيب، اكتشفوا التهاب بولي حاد. بعد العلاج بثلاثة أيام، رجعت الأم طبيعية تماماً.

هذا الموقف يختلف كلياً عن موقف آخر: رجل عمره 82 سنة، بدأ ينسى تدريجياً على مدى سنة كاملة. يسأل نفس السؤال عشر مرات في اليوم، ويتوه في الحي الذي عاش فيه 40 سنة.

\textbf{الفرق بينهما كبير جداً:}
\begin{itemize}
\item الأولى: تغيّر مفاجئ (هذيان) — خطر وطارئ لكن قابل للعلاج غالباً
\item الثاني: تدهور تدريجي (خرف) — مزمن ويحتاج إدارة طويلة الأمد
\end{itemize}

\begin{warnbox}
كثير من الأسر تخلط بين الاثنين. والنتيجة: إما تأخير علاج طارئ، أو رحلة مستشفيات غير مجدية لحالة مزمنة.
\end{warnbox}

\textbf{دوري في هذا الفصل:}
\begin{itemize}
\item أعلمك تفرّق بين الأنواع الثلاثة الأساسية
\item أعطيك أدوات عملية للتعامل في البيت
\item أوضح لك متى الموضوع طارئ ومتى هو إدارة يومية
\item أريحك من شعور "أنا ما أعرف أتصرف معه"
\end{itemize}

\section*{التمييز السريع — علامات فارقة للأسرة}

قبل ما ندخل في التفاصيل، دعني أعطيك جدول بسيط تحتفظ فيه:

\begin{longtable}{>{\raggedright\arraybackslash}p{0.18\textwidth}>{\raggedright\arraybackslash}p{0.25\textwidth}>{\raggedright\arraybackslash}p{0.25\textwidth}>{\raggedright\arraybackslash}p{0.22\textwidth}}
\toprule\noalign{}\tableheadercolor
\thead{الحالة} & \thead{كيف يبدأ؟} & \thead{علامات مميزة} & \thead{ماذا تفعل؟} \\
\midrule
\textbf{هذيان} (Delirium) & مفاجئ (ساعات-أيام) & تشوش الوعي، هلوسة، نوم مضطرب & \textcolor{red}{طارئ — اتصل بالطبيب فوراً} \\
\midrule
\textbf{خرف} (Dementia) & تدريجي (شهور-سنوات) & نسيان، توهان، تكرار أسئلة & إدارة يومية + متابعة طبية \\
\midrule
\textbf{اكتئاب} (Depression) & تدريجي أو بعد حدث & انعزال، بكاء، فقد شهية، قلة حركة & دعم نفسي + تحويل للطبيب \\
\bottomrule
\end{longtable}

\begin{tipbox}
\textbf{قاعدة ذهبية:} أي تغير مفاجئ في الوعي أو السلوك (خلال ساعات أو أيام) = طارئ. لا تنتظر.
\end{tipbox}

\section*{الهذيان (Delirium) — الطارئ الخفي}

\subsection*{كيف يظهر في البيت؟}

الهذيان هو تشوش حاد ومفاجئ في الوعي. الأسرة تقول عادة:

\begin{storybox}
"أبوي من أمس وهو مو طبيعي. يقول فيه ناس في الغرفة. ساعات يعرفني وساعات لا. النهار ينام والليل يصحى ويتحرك بدون منطق."
\end{storybox}

\textbf{علامات الهذيان الواضحة:}
\begin{itemize}
\item تشوش في التركيز والانتباه (يبدأ كلام ويقطعه فجأة)
\item عدم معرفة الوقت والمكان (يظن أنه في بيته القديم أو في العمل)
\item كلام غير مترابط أو غير مفهوم
\item هلوسة بصرية أو سمعية (يرى أو يسمع أشياء غير موجودة)
\item نوم مقلوب (نهاره ليل وليله نهار)
\item تقلب سريع بين الهدوء والهياج
\end{itemize}

\subsection*{أسباب شائعة (بصياغة عامة للأسرة)}

\textbf{السبب ليس في الدماغ غالباً — بل في الجسم:}

\begin{itemize}
\item \textbf{عدوى/التهاب:} التهاب بولي، التهاب رئة، حتى التهاب بسيط في اللثة
\item \textbf{جفاف:} قلة شرب الماء خاصة في الصيف
\item \textbf{أدوية جديدة أو تغيير جرعة:} كثير من الأدوية تؤثر على الوعي
\item \textbf{إمساك شديد أو احتباس بول}
\item \textbf{قلة نوم أو ألم مستمر}
\item \textbf{نقص أكسجين:} مشاكل قلب أو رئة
\item \textbf{انخفاض أو ارتفاع سكر الدم}
\end{itemize}

\begin{warnbox}
\textbf{تحذير مهم:} كبار السن قد لا تظهر عليهم أعراض العدوى الواضحة (حرارة عالية مثلاً). أحياناً الهذيان هو العلامة الوحيدة على وجود مشكلة جسدية خطيرة.
\end{warnbox}

\subsection*{ماذا نفعل فوراً في البيت؟}

\textbf{1. لا تجادل أو تصحح الهلوسة}

خطأ شائع: "يا أبوي، ما فيه أحد في الغرفة! أنت تتخيل!"

الأفضل: "أنا معك، أنا هنا. كل شيء بخير. خلني أطمّن عليك."

\textbf{2. أمّن البيئة}
\begin{itemize}
\item أبعد أي أدوات خطرة
\item أقفل باب الخروج من البيت إذا كان يحاول يطلع
\item شغّل إضاءة خفيفة حتى في الليل (الظلام يزيد الهلوسة)
\item ابق معه أو راقبه عن قرب
\end{itemize}

\textbf{3. حاول تهدئة البيئة}
\begin{itemize}
\item صوت واحد فقط يكلمه (لا تجتمعون كلكم عليه)
\item خفف الإزعاج والأصوات
\item حاول تذكّره بالمكان والوقت بهدوء: "أنت في بيتك، أنا ولدك فلان، اليوم الخميس"
\end{itemize}

\textbf{4. تحقق من الأساسيات}
\begin{itemize}
\item قِس الحرارة
\item تأكد أنه شرب ماء كافي (الشفاه جافة؟ البول أصفر غامق؟)
\item تأكد أنه تبوّل وتبرّز بشكل طبيعي
\item راجع الأدوية: هل أخذ جرعة زيادة؟ هل في دواء جديد؟
\end{itemize}

\subsection*{متى تطلب تقييم طبي عاجل؟}

\textbf{اتصل بالطبيب أو الطوارئ فوراً إذا:}
\begin{itemize}
\item أول مرة يحصل هذا التشوش
\item مصاحب بحرارة أو صعوبة تنفس
\item رفض شرب ماء أو أكل لأكثر من 12 ساعة
\item محاولة إيذاء نفسه أو الآخرين
\item فقدان وعي متكرر أو شديد
\item ضعف مفاجئ في جانب من الجسم (احتمال جلطة)
\end{itemize}

\textbf{الهذيان ليس "جنون" ولا "خرف فجأة" — هو علامة أن الجسم في مشكلة. والخبر السار: إذا عالجنا السبب، غالباً يرجع طبيعي.}

\section*{الخرف وألزهايمر — الإدارة اليومية للسلوك}

\subsection*{الفرق بين النسيان الطبيعي والخرف}

كلنا ننسى مفتاح السيارة أو اسم شخص. لكن الخرف مختلف:

\begin{longtable}{>{\raggedright\arraybackslash}p{0.45\textwidth}>{\raggedright\arraybackslash}p{0.45\textwidth}}
\toprule\noalign{}\tableheadercolor
\thead{نسيان طبيعي} & \thead{علامات خرف محتملة} \\
\midrule
ينسى أين وضع النظارة، لكن يتذكر لاحقاً & ينسى أنه يحتاج نظارة أصلاً \\
\midrule
ينسى موعد، لكن يتذكر بالتنبيه & ينسى أن كان عنده موعد حتى بعد التذكير \\
\midrule
يضيع أحياناً في مكان جديد & يتوه في الحي الذي عاش فيه سنين \\
\midrule
يخطئ في كلمة أو اثنتين & يستخدم كلمات غريبة أو غير مناسبة باستمرار \\
\midrule
يحتاج تفكير لحل مشكلة & لا يستطيع حل مشاكل بسيطة (مثل دفع الفواتير) \\
\bottomrule
\end{longtable}

\textbf{قاعدة عملية:} إذا النسيان بدأ يؤثر على الحياة اليومية (ما يعرف يستخدم التلفون، ينسى يأكل، يتوه في البيت) — وقتها نحتاج تقييم طبي.

\subsection*{كيف نتعامل مع السلوكيات الصعبة؟}

\textbf{1. الروتين الثابت — صديقك الأول}

مريض الخرف يرتاح مع الروتين. التغيير المفاجئ يزيد التشوش والقلق.

\begin{itemize}
\item موعد نوم واستيقاظ ثابت
\item وجبات في نفس الأوقات
\item أنشطة يومية بسيطة ومتكررة
\item نفس الأشخاص قدر الإمكان (وجوه جديدة تشوشه)
\end{itemize}

\textbf{2. تبسيط البيئة}

\begin{itemize}
\item أبعد الفوضى — كل شيء في مكانه
\item علامات واضحة على الأبواب (حمام، غرفة نوم) بصور بسيطة
\item خزانة ملابس فيها 2-3 خيارات فقط (الخيارات الكثيرة تشوشه)
\item إضاءة جيدة — الظلال والإضاءة الخافتة تخوّف
\end{itemize}

\textbf{3. الكلام القصير والواضح}

بدلاً من: "يا أبوي، حبيت أقول لك إن الدكتور قال لازم تأخذ الحبة البيضاء مع الأكل والزرقاء بعد الأكل..."

قل: "تفضل، خذ الدواء." (ومعك الحبة والماء جاهزين)

\begin{tipbox}
\textbf{نصيحة من الميدان:} الجمل الطويلة والخيارات الكثيرة تشوّش مريض الخرف. كلام قصير + فعل واحد في كل مرة.
\end{tipbox}

\textbf{4. لا تجادل ولا تصحّح الذاكرة}

\begin{storybox}
مريضة تقول: "أبغى أروح أشوف أمي." (وأمها توفيت من 20 سنة)

ردّ خطأ: "يا أمي أمك توفيت، نسيتي؟" (هذا يحطمها نفسياً كل مرة)

ردّ أفضل: "أمك الله يرحمها. تحبينها كثير صح؟ تذكريني عنها." (تحوّل الموضوع بلطف)
\end{storybox}

\textbf{القاعدة الذهبية: ادخل عالمه، لا تجبره يدخل عالمك.}

\subsection*{التعامل مع سلوكيات محددة}

\textbf{الرفض (رفض الاستحمام، رفض الدواء، رفض الأكل)}

\textbf{لماذا يحدث؟}
\begin{itemize}
\item خوف من الماء أو البرد
\item نسي لماذا يحتاج هذا
\item يشعر بفقدان السيطرة
\item الوقت غير مناسب (تعبان أو جوعان)
\end{itemize}

\textbf{كيف نتعامل؟}
\begin{itemize}
\item لا تجبر — جرب بعد نصف ساعة أو ساعة
\item حوّل الموضوع لشيء ممتع: "تعال نستحمم ونلبس ثوبك الجديد"
\item خليه يحس أنه يساعدك: "ممكن تمسك لي الصابونة؟"
\item جرب شخص آخر يطلب منه (أحياناً يرفض من ابنه ويقبل من بنته)
\end{itemize}

\textbf{التوهان داخل البيت}

\textbf{كيف نمنعه؟}
\begin{itemize}
\item علامات واضحة بالعربي والصور على أبواب الغرف
\item إضاءة الممرات ليلاً (كثير يتوهون في طريق الحمام بالليل)
\item قفل أبواب الغرف الخطرة (المطبخ، المخزن)
\item سوار تعريف عليه رقم تلفون (إذا طلع من البيت)
\end{itemize}

\textbf{تكرار نفس السؤال 20 مرة}

"متى بنروح المستشفى؟" — تجاوب. بعد دقيقتين: "متى بنروح المستشفى؟"

\begin{itemize}
\item لا تقل "قلت لك قبل شوي!" — هو فعلاً ما يتذكر
\item جاوب بنفس الهدوء كل مرة، جملة قصيرة
\item ممكن تحوّل انتباهه بنشاط بسيط
\item لو السؤال عن موعد معين، اكتب له في ورقة واضحة وعلقها قدامه
\end{itemize}

\textbf{الاندفاع والحركة الزائدة}

بعض مرضى الخرف يمشون بدون توقف، أو يفتحون الأدراج ويخربطون فيها.

\begin{itemize}
\item ممكن يكون يدور على شيء (حمام، أكل، أمان)
\item ممكن قلق أو ملل
\item وفّر له نشاط آمن: طي فوط، ترتيب أشياء بسيطة، كرة إسفنج يلعب فيها
\item امشِ معه — المشي يهدي كثير من المرضى
\end{itemize}

\subsection*{السلامة — تجنب السقوط والحوادث}

\textbf{قفل مصادر الخطر (بدون تخويف، بحكمة)}

\begin{itemize}
\item البوتاجاز: أقفله من المصدر، أو استخدم قفل أمان للأطفال
\item السكاكين والأدوات الحادة: في درج مقفل
\item الأدوية: في خزانة مقفلة بعيدة عن متناوله
\item المنظفات والكيماويات: مخبّاة أو مقفلة
\end{itemize}

\textbf{منع الخروج المفاجئ من البيت}

بعض المرضى يفتحون الباب ويطلعون بدون انتباه، وممكن يتوهون.

\begin{itemize}
\item قفل إضافي في أعلى الباب (صعب يوصل له)
\item جرس على الباب ينبهك إذا فتح
\item لو خرج، لا تجري وراه بعصبية — هذا يخوفه. امشِ معه بهدوء وحوّل اتجاهه تدريجياً
\end{itemize}

\begin{warnbox}
\textbf{تحذير — اتصل بالطبيب إذا:}
\begin{itemize}
\item التدهور صار سريع جداً (خلال أسابيع)
\item عنف جسدي تجاهك أو تجاه نفسه
\item رفض كامل للأكل والشرب لأكثر من يوم
\item هلوسة مخيفة ومستمرة
\end{itemize}
\end{warnbox}

\section*{الاكتئاب والقلق عند كبار السن}

\subsection*{متى نشك أنها مشكلة نفسية؟}

الاكتئاب عند كبار السن غالباً ما ينلبس بلبس جسدي. الأسرة تقول:

"أبوي صار كسلان، ما يبغى يقوم من السرير، يقول تعبان."

السؤال: هل هذا كسل؟ أم اكتئاب؟

\textbf{قاعدة الأسبوعين + علامات وظيفية:}

إذا استمرت هذه الأعراض أسبوعين أو أكثر، وأثّرت على الحياة اليومية، ممكن يكون اكتئاب:

\begin{itemize}
\item حزن مستمر أو بكاء بدون سبب واضح
\item فقدان الاهتمام بأشياء كان يحبها (القهوة مع الأصدقاء، مشاهدة التلفزيون، حتى الصلاة)
\item تغيّر في النوم (ينام كثير جداً أو ما ينام)
\item تغيّر في الشهية (فقد شهية أو أكل كثير)
\item تعب وخمول رغم عدم بذل مجهود
\item كلام عن الموت أو "أبغى أموت وأريح الناس"
\item قلة حركة وانسحاب من الناس
\end{itemize}

\subsection*{كيف نفرّق بين "كسل/عناد" وبين اكتئاب؟}

\begin{longtable}{>{\raggedright\arraybackslash}p{0.45\textwidth}>{\raggedright\arraybackslash}p{0.45\textwidth}}
\toprule\noalign{}\tableheadercolor
\thead{كسل أو عناد} & \thead{اكتئاب محتمل} \\
\midrule
يرفض نشاط معين لكن يقبل غيره & يرفض كل شيء، حتى الأشياء اللي كان يحبها \\
\midrule
مزاجه يتحسن مع الكلام أو التشجيع & مزاجه منخفض طول الوقت، ما يتحسن \\
\midrule
يضحك ويتفاعل أحياناً & وجهه جامد، كلامه قليل، ما فيه تفاعل \\
\midrule
يشتكي بصوت عالي ويبغى انتباه & ساكت ومنعزل، أو يقول "خلوني لحالي" \\
\bottomrule
\end{longtable}

\textbf{نقطة مهمة:} الألم المزمن (ركبة، ظهر، رأس) ممكن يسبب اكتئاب. والاكتئاب يزيد الإحساس بالألم. دائرة صعبة.

\subsection*{ماذا نفعل في البيت لدعم المزاج؟}

\textbf{لا وعود بالشفاء — لكن دعم حقيقي:}

\textbf{1. حركة خفيفة ومنتظمة}

الحركة الجسدية من أقوى علاجات الاكتئاب (هذا مثبت علمياً).

\begin{itemize}
\item مشي خفيف 10 دقائق يومياً (حتى داخل البيت)
\item جلوس في الشمس 15 دقيقة صباحاً
\item تمارين تنفس بسيطة
\end{itemize}

\textbf{ملاحظة مهمة:} المكتئب ما بيبغى يتحرك. لا تقول له "قوم تحرك بتتحسن!" — هذا يزيد شعوره بالفشل.

بدلاً من ذلك: "تعال نمشي لباب البيت ونرجع، أبغى أتمشى شوي ونتكلم."

\textbf{2. روتين يومي ثابت}

الاكتئاب يحب الفوضى. الروتين يعطي إحساس بالتحكم:

\begin{itemize}
\item استيقاظ في نفس الوقت (حتى لو ما نام زين)
\item وجبات منتظمة
\item نشاط بسيط ثابت (ولو صغير: سقي نبتة، طي فوط، ترتيب أغراض)
\end{itemize}

\textbf{3. تواصل إنساني حقيقي}

الانعزال يزيد الاكتئاب. والاكتئاب يدفع للانعزال.

\begin{itemize}
\item جلسة يومية 15 دقيقة تكلمه فيها (لو حتى ما رد كثير)
\item اسأله عن قصص الماضي (غالباً الذكريات القديمة تريحهم)
\item شغّل له آذان أو قرآن — كثير منهم يرتاح له
\item لا تتركه وحده فترات طويلة
\end{itemize}

\textbf{4. أهداف صغيرة قابلة للتحقيق}

لا تقول: "لازم تتحسن وتصير مثل أول!"

قل: "اليوم هدفنا تفطر وتمشي لباب الغرفة. بكرة نزيد شوي."

\textbf{النجاحات الصغيرة تبني الثقة. الأهداف الكبيرة تحطمه.}

\subsection*{متى نحوّل لطبيب/أخصائي نفسي/اجتماعي؟}

\textbf{لا تنتظر — حوّل مباشرة إذا:}
\begin{itemize}
\item كلام عن إيذاء نفسه أو الانتحار (حتى لو "مجرد كلام")
\item رفض أكل أو شرب لأكثر من يومين
\item عدم القدرة على النوم نهائياً لعدة أيام
\item اكتئاب مستمر رغم الدعم الأسري لأكثر من شهر
\item قلق شديد أو نوبات هلع متكررة
\item انسحاب كامل من الحياة (ما يكلم أحد، ما يطلع من الغرفة)
\end{itemize}

\begin{tipbox}
\textbf{طمأنة للأسرة:} طلب مساعدة نفسية أو استشارة اجتماعية ليس عيب ولا ضعف. بالعكس — هذا دليل حكمة واهتمام بمن تحب.
\end{tipbox}

\section*{قسم خاص بالمرافق — كيف تحمي نفسك ولا تكون سبب ضرر}

\subsection*{كيف لا تكون سبب ضرر؟}

\textbf{أخطاء شائعة تزيد المشكلة:}

\textbf{1. الضغط والإلحاح}

"يا أبوي ليش ما تتذكر؟ ركّز! أنا قلت لك قبل شوي!"

هذا يزيد القلق والتشوش. المريض يحس بالفشل.

\textbf{2. العتاب على التصرفات}

"ليش سويت كذا؟ كم مرة قلت لك لا تفتح الباب!"

المريض ما يتحكم. العتاب ما ينفع — بل يحطم نفسيته.

\textbf{3. التحدي العلني}

"أنت غلطان! أمك توفيت! كيف نسيت؟!"

التحدي المباشر يخلق صراع ما له داعي.

\subsection*{جمل جاهزة للتواصل — تقنع بدون صدام}

\textbf{استخدم هذه الجمل بدلاً من الجدال:}

\begin{enumerate}
\item \textbf{بدلاً من "لا":} "خلنا نسوي كذا أحسن" أو "ممكن نجرب كذا؟"

\item \textbf{عند الرفض:} "فهمتك. ممكن نجرب بعدين؟" (ثم حاول بعد 20 دقيقة)

\item \textbf{عند التشوش:} "كل شيء تمام. أنا هنا معك." (صوت هادئ + لمسة يد خفيفة)

\item \textbf{عند السؤال المتكرر:} نفس الجواب الهادئ، كل مرة (ما تقول "قلت لك!")

\item \textbf{عند الخوف:} "أنت آمن. أنا جنبك. ما فيه خطر."

\item \textbf{عند الهلوسة:} "أنا ما أشوف هالشيء، لكن أفهم إنك خايف. تعال نروح مكان ثاني."

\item \textbf{عند الغضب:} "أشوف إنك زعلان. كيف أقدر أساعدك؟"

\item \textbf{للتشجيع البسيط:} "أنت سويت زين اليوم. أنا فخور فيك."
\end{enumerate}

\begin{tipbox}
\textbf{قاعدة التواصل الذهبية:} صوت هادئ + جملة قصيرة + تواصل بصري + لمسة يد (إذا كان يرتاح لها) = تواصل فعّال.
\end{tipbox}

\subsection*{كيف تحمي نفسك من الإنهاك؟}

رعاية مريض خرف أو اكتئاب من أصعب أنواع الرعاية. أنت تعطي بدون توقف، والمريض ما يتحسن — بل أحياناً يزيد سوءاً.

\textbf{علامات إنهاك المرافق:}
\begin{itemize}
\item أنت نفسك حاس بالاكتئاب أو القلق
\item نومك تعبان
\item صرت تنفعل بسرعة على المريض
\item تشعر بالذنب طول الوقت
\item صحتك الجسدية تدهورت
\item انعزلت عن أهلك وأصدقائك
\end{itemize}

\textbf{كيف تحمي نفسك؟}

\textbf{1. اطلب مساعدة — لا تتحمل لحالك}

\begin{itemize}
\item تناوب مع أفراد الأسرة (يوم أنت، يوم أخوك، يوم أختك)
\item ممرض منزلي بضع ساعات في الأسبوع (إذا القدرة المادية تسمح)
\item مراكز رعاية نهارية لكبار السن (في بعض المدن)
\end{itemize}

\textbf{2. خذ راحة حقيقية}

ليس عيب ولا قلة احترام أنك تاخذ استراحة.

\begin{itemize}
\item ساعة أو ساعتين يومياً لنفسك (امشِ، اقرأ، اتصل بصديق)
\item يوم كامل في الأسبوع إجازة (أحد يجلس مكانك)
\end{itemize}

\textbf{3. تواصل مع مجموعات دعم}

كثير مستشفيات ومراكز فيها مجموعات دعم لأهالي مرضى الخرف والاكتئاب. الكلام مع ناس يفهمون وضعك يخفف كثير.

\textbf{4. لا تشعر بالذنب}

مشاعر الذنب طبيعية لكن غير عادلة على نفسك:

\begin{itemize}
\item "أنا مقصّر" — لا، أنت تبذل جهد هائل
\item "لو سويت أكثر كان تحسن" — المرض له مساره، مو ذنبك
\item "أنا أحياناً أتمنى ينتهي هالموضوع" — هذا شعور إنساني طبيعي، ما يعني إنك سيء
\end{itemize}

\begin{warnbox}
\textbf{تحذير للمرافق:} إذا وصلت لمرحلة تفكر في إيذاء نفسك أو المريض — توقف فوراً واطلب مساعدة نفسية عاجلة. هذا ليس ضعف — هذا إنقاذ لك وله.
\end{warnbox}

\section*{علامات الخطر — حالات تستدعي تدخل عاجل}

\textbf{اتصل بالطوارئ أو الطبيب فوراً إذا حدث أي من التالي:}

\subsection*{علامات طبية طارئة:}
\begin{itemize}
\item تشوش مفاجئ في الوعي (هذيان) خلال ساعات أو يوم — خاصة مع حرارة أو صعوبة تنفس
\item ضعف مفاجئ في جانب من الجسم، أو صعوبة في الكلام (احتمال جلطة دماغية)
\item فقدان وعي أو نوبة تشنج
\item رفض كامل للأكل والشرب لأكثر من 24 ساعة
\item حرارة فوق 38.5 درجة
\item ألم صدر أو ضيق نفس شديد
\end{itemize}

\subsection*{علامات خطر نفسي/سلوكي:}
\begin{itemize}
\item كلام عن الانتحار أو إيذاء النفس (حتى لو يبدو "مجرد كلام")
\item محاولة إيذاء نفسه أو الآخرين
\item عنف جسدي متكرر لا تستطيع السيطرة عليه
\item هياج شديد ومستمر مع خطر على السلامة
\item هلوسة مخيفة تسبب ذعر شديد للمريض
\end{itemize}

\subsection*{علامات تدهور سريع تحتاج تقييم طبي عاجل (خلال يوم أو يومين):}
\begin{itemize}
\item تدهور ذهني سريع جداً (خلال أسابيع بدلاً من شهور)
\item توقف كامل عن الحركة أو الكلام فجأة
\item سقوط متكرر (أكثر من مرتين في أسبوع)
\item فقدان القدرة على البلع أو الشرب
\end{itemize}

\section*{خاتمة — الرضا وبذل السبب وكرامة كبير السن}

رعاية كبير السن الذي تغيّر ذهنياً أو نفسياً من أشق أنواع الرعاية.

أنت تتعامل مع شخص تحبه، لكنه صار "غير نفسه". أحياناً لا يعرفك. أحياناً يرفضك. أحياناً يتهمك. وأحياناً يبكي بدون سبب تفهمه.

هذا مؤلم. وثقيل. وطبيعي أن تتعب.

\textbf{لكن دعني أقول لك شيء من واقع عملي مع مئات الأسر:}

كبير السن — حتى لو فقد ذاكرته — ما فقد إحساسه.

هو يحس بالصوت الهادئ. يحس باللمسة الحانية. يحس بالصبر. ويحس بالغضب والعجلة.

قد لا يتذكر اسمك، لكنه يتذكر — في مكان ما عميق — أنك آمان.

\textbf{دورك ليس "إصلاحه" أو "إرجاعه لطبيعته".} دورك حمايته، وتوفير الأمان، والحفاظ على كرامته.

\begin{tipbox}
\textbf{رسالة للمرافق المتعب:}

أنت لست مسؤولاً عن شفائه. أنت مسؤول عن بذل السبب والرعاية بقدر طاقتك.

طلب المساعدة ليس تقصير — بل حكمة.

الراحة لنفسك ليست أنانية — بل ضرورة لتستمر.

والدعاء بالفرج — لك وله — من أقوى ما تملك.
\end{tipbox}

\textbf{في النهاية:}

الله يكتب أجرك في كل لحظة صبر. في كل جملة هادئة. في كل ليلة ما نمت فيها. في كل دمعة بكيتها من التعب.

احتسب. واطلب العون. ولا تستحي تقول "أنا تعبت وأحتاج مساعدة."

اللهم ارحم ضعفهم، واشفِ مرضاهم، وأجرنا في خدمتهم، وخفف عن كل مرافق متعب.

\section*{المراجع العلمية}

[1] Inouye, S. K., et al.~(2014). Delirium in elderly people. The Lancet, 383(9920), 911-922. https://doi.org/10.1016/S0140-6736(13)60688-1

[2] Fong, T. G., Tulebaev, S. R., \& Inouye, S. K. (2009). Delirium in elderly adults: diagnosis, prevention and treatment. Nature Reviews Neurology, 5(4), 210-220. https://doi.org/10.1038/nrneurol.2009.24

[3] Livingston, G., et al.~(2020). Dementia prevention, intervention, and care: 2020 report of the Lancet Commission. The Lancet, 396(10248), 413-446. https://doi.org/10.1016/S0140-6736(20)30367-6

[4] Alzheimer's Association. (2023). 2023 Alzheimer's disease facts and figures. Alzheimer's \& Dementia, 19(4), 1598-1695. https://doi.org/10.1002/alz.13016

[5] Brodaty, H., \& Donkin, M. (2009). Family caregivers of people with dementia. Dialogues in Clinical Neuroscience, 11(2), 217-228.

[6] Zhao, Q. F., et al.~(2016). The prevalence of neuropsychiatric symptoms in Alzheimer's disease: Systematic review and meta-analysis. Journal of Affective Disorders, 190, 264-271. https://doi.org/10.1016/j.jad.2015.09.069

[7] Lyketsos, C. G., et al.~(2011). Neuropsychiatric symptoms in Alzheimer's disease. Alzheimer's \& Dementia, 7(5), 532-539. https://doi.org/10.1016/j.jalz.2011.05.2410

[8] Fiske, A., Wetherell, J. L., \& Gatz, M. (2009). Depression in older adults. Annual Review of Clinical Psychology, 5, 363-389. https://doi.org/10.1146/annurev.clinpsy.032408.153621

[9] Blazer, D. G. (2003). Depression in late life: review and commentary. The Journals of Gerontology Series A: Biological Sciences and Medical Sciences, 58(3), M249-M265. https://doi.org/10.1093/gerona/58.3.M249

[10] Stuck, A., et al.~(2017). Effect of physical exercise on depression in nursing home residents: A systematic review. Geriatric Nursing, 38(6), 569-576. https://doi.org/10.1016/j.gerinurse.2017.04.001

[11] Gitlin, L. N., \& Hodgson, N. (2015). Caregivers as therapeutic agents in dementia care: The evidence-base for interventions supporting their role. In Family Caregiving in the New Normal (pp. 305-353). Academic Press. https://doi.org/10.1016/B978-0-12-417046-9.00019-5

[12] Sörensen, S., Duberstein, P., Gill, D., \& Pinquart, M. (2006). Dementia care: mental health effects, intervention strategies, and clinical implications. The Lancet Neurology, 5(11), 961-973. https://doi.org/10.1016/S1474-4422(06)70599-3

\emph{ملاحظة: جميع القصص الواردة في هذا الفصل حقيقية، لكن تم تغيير الأسماء والتفاصيل الشخصية للحفاظ على خصوصية المرضى.}
