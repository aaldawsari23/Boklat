\chapter{المضاعفات عندما تحدث — خطة التعامل بعد وقوع المشكلة}

\section*{مقدمة قصيرة}

الوقاية مهمة، لكن وجود خطة واضحة بعد حدوث المشكلة أهم مما نتخيل. كثير من الأسر ترتبك عندما يظهر تيبّس مفاجئ، أو احمرار جلد، أو سقوط. هذا الفصل يعطيك مسارًا عمليًا: ماذا نفعل اليوم، ماذا نوقف، ومتى نحول لمختص.

\section*{قاعدة ذهبية: أولويات التعامل}

رتّب خطواتك دائمًا بهذا التسلسل:
\begin{itemize}
\item \textbf{السلامة أولاً:} أوقف أي نشاط قد يزيد الضرر.
\item \textbf{إيقاف السبب:} قلل الضغط أو الحركة الخاطئة فورًا.
\item \textbf{تقييم بسيط:} ماذا تغيّر؟ متى بدأ؟ هل هناك ألم جديد؟
\item \textbf{خطة يومية واضحة:} روتين ثابت بدل التشتت.
\end{itemize}

\section*{التيبّس والشدّ العضلي إذا صار}

\subsection*{كيف أعرف أنه تيبّس خطير أم طبيعي بعد قلة حركة؟}

\textbf{تيبّس طبيعي:}
\begin{itemize}
\item يظهر بعد قلة حركة واضحة.
\item يتحسن تدريجيًا مع الحركة اللطيفة.
\item لا يصاحبه ألم شديد أو احمرار.
\end{itemize}

\textbf{تيبّس مقلق:}
\begin{itemize}
\item ألم شديد أو مفاجئ مع الحركة البسيطة.
\item مفصل ساخن أو منتفخ.
\item تيبّس يزيد يومًا بعد يوم رغم الحركة اللطيفة.
\end{itemize}

\subsection*{ماذا نفعل خلال 48 ساعة الأولى؟}
\begin{itemize}
\item خفف الضغط على المفصل المؤلم ولا تجبره على مدى كبير.
\item حرّك المفصل حركات لطيفة ضمن المدى المريح.
\item بدّل الوضعيات كل فترة حتى لا يثبت الجسم على وضعية واحدة.
\item راقب الألم: إذا زاد بشكل واضح، توقّف واطلب تقييمًا.
\end{itemize}

\subsection*{روتين يومي 10--15 دقيقة}
\begin{itemize}
\item حركات لطيفة للمفاصل القريبة (5 دقائق).
\item تبديل وضعيات: جلوس، وقوف بمساندة، استلقاء (5 دقائق).
\item تنفّس هادئ وتحريك الأطراف بخفة (3--5 دقائق).
\end{itemize}

\subsection*{أخطاء شائعة تزيده}
\begin{itemize}
\item شد قوي أو مفاجئ.
\item تسخين عشوائي أو تعريض مباشر لحرارة عالية.
\item إجبار كبير السن على حركة يرفضها جسمه.
\end{itemize}

\subsection*{متى نحتاج مختص؟}
\begin{itemize}
\item ألم شديد لا يهدأ.
\item تورم واضح أو سخونة غير طبيعية.
\item فقدان حركة مفاجئ في مفصل.
\end{itemize}

\section*{قرحة الفراش إذا بدأت}

\subsection*{كيف أميّز احمرار عابر أم بداية قرحة؟}
\begin{itemize}
\item احمرار يختفي بعد تخفيف الضغط غالبًا طبيعي.
\item احمرار ثابت لا يختفي بعد وقت مع ألم أو سخونة = بداية مشكلة.
\end{itemize}

\subsection*{ماذا نفعل فورًا؟}
\begin{itemize}
\item خفف الضغط عن المنطقة مباشرة.
\item غيّر الوضعيات بشكل منتظم خلال اليوم.
\item حافظ على الجلد نظيفًا وجافًا بلطف.
\item استخدم سطح جلوس أو نوم مناسب يقلل الضغط.
\end{itemize}

\subsection*{ماذا نمنع؟}
\begin{itemize}
\item تدليك عنيف للمنطقة الحمراء.
\item ترك الجلد مكشوفًا للهواء الحار أو البارد لفترة طويلة.
\item تجاهل الألم بحجة أنه طبيعي.
\end{itemize}

\subsection*{متى لازم تقييم طبي عاجل؟}
\begin{itemize}
\item ظهور جرح مفتوح.
\item احمرار مع إفرازات أو رائحة.
\item ألم شديد أو ارتفاع حرارة عام.
\end{itemize}

\section*{السقوط المتكرر أو قرب السقوط}

\subsection*{ماذا نسوي بعد سقطة؟}

\textbf{اليوم الأول:}
\begin{itemize}
\item افحص أي ألم جديد في الورك أو الظهر أو الرأس.
\item خفف الحركة العشوائية وراقب المشي.
\item إذا كان هناك دوخة أو ارتباك، لا تتركه وحده.
\end{itemize}

\textbf{الأسبوع الأول:}
\begin{itemize}
\item راجع بيئة البيت: السجاد، العتبات، الإضاءة.
\item تأكد أن الجهاز المساعد مناسب ومتين.
\item ابدأ تمارين بسيطة للتوازن تحت إشراف مختص إن أمكن.
\end{itemize}

\subsection*{متى نحتاج تصوير أو تقييم؟}
\begin{itemize}
\item ألم شديد في الورك أو عدم القدرة على الوقوف.
\item ضربة رأس مع دوخة أو نعاس غير معتاد.
\item كدمات كبيرة أو شك بوجود كسر.
\end{itemize}

\subsection*{خطة تقليل السقوط}
\begin{itemize}
\item بيئة: مسار مشي واضح وإضاءة كافية.
\item جهاز مساعدة مناسب وتقييم ارتفاعه.
\item تمارين بسيطة للتوازن والقوة بشكل تدريجي.
\end{itemize}

\section*{الشرقة المتكررة وصعوبة البلع}

\subsection*{علامات الخطر}
\begin{itemize}
\item كحة متكررة أثناء الأكل.
\item صوت مبلل بعد الشرب.
\item تكرار الالتهاب الرئوي أو الحمى بعد الأكل.
\end{itemize}

\subsection*{ماذا نفعل مباشرة؟}
\begin{itemize}
\item اجعل الجلسة مستقيمة والذقن بوضع محايد.
\item قسّم اللقيمات إلى أحجام صغيرة.
\item عدّل القوام بشكل عام بحسب ما يتحمله.
\end{itemize}

\subsection*{متى لازم تقييم مختص؟}
\begin{itemize}
\item شرقة متكررة يومية.
\item فقدان وزن ملحوظ.
\item تغيّر مفاجئ في القدرة على البلع.
\end{itemize}

\section*{متى نوقف كل شيء ونتصل بطوارئ؟}

\begin{itemize}
\item فقدان وعي أو دوخة شديدة بعد سقوط.
\item ألم صدر أو صعوبة تنفس مفاجئة.
\item ضعف مفاجئ في جانب من الجسم أو اضطراب كلام.
\item نزيف لا يتوقف أو جرح عميق.
\item تدهور سريع وغير مفسر في الحركة أو الوعي.
\end{itemize}

\section*{خاتمة قصيرة}

المضاعفات ليست نهاية الطريق، لكنها تحتاج عقل وخطة. كلما كان ردّك هادئًا ومنظمًا، قلّت الخسائر وزادت فرصة العودة للاستقرار.

\section*{المراجع}

[1] National Institute for Health and Care Excellence. (2014). \emph{Pressure ulcers: prevention and management} (CG179).

[2] European Pressure Ulcer Advisory Panel, National Pressure Injury Advisory Panel, \& Pan Pacific Pressure Injury Alliance. (2019). \emph{Prevention and Treatment of Pressure Ulcers/Injuries: Clinical Practice Guideline}.

[3] Sherrington, C., Fairhall, N. J., Wallbank, G. K., et al.~(2019). Exercise for preventing falls in older people living in the community. \emph{Cochrane Database of Systematic Reviews}, 2019(1).

[4] Panel on Prevention of Falls in Older Persons, American Geriatrics Society and British Geriatrics Society. (2011). Summary of the updated AGS/BGS clinical practice guideline for prevention of falls in older persons. \emph{Journal of the American Geriatrics Society}, 59(1), 148--157.

[5] National Institute on Aging. (2022). \emph{Falls and Fractures}. NIH.

[6] American Speech-Language-Hearing Association. (2023). \emph{Adult Dysphagia Practice Portal}. ASHA.

[7] Cichero, J. A. Y., Lam, P., Steele, C. M., et al.~(2017). Development of International Terminology and Definitions for texture-modified foods and thickened fluids used in dysphagia management: The IDDSI Framework. \emph{Dysphagia}, 32(2), 293--314.

[8] Winstein, C. J., Stein, J., Arena, R., et al.~(2016). Guidelines for adult stroke rehabilitation and recovery. \emph{Stroke}, 47(6), e98--e169.
