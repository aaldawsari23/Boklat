\chapter{الحركة كدواء آمن}
\label{ch:08}

\subsection*{من الميدان}

في إحدى الزيارات، كانت الجدة "أم سالم" لا تتحرك من سريرها تقريبًا منذ أسابيع. الأسرة تقول: "نخشى أن تحريكها سيُتعبها أو أن تنكسر عظامها." عندما كشفت على ظهرها، وجدت بداية قرحة حمراء في العجز (أسفل الظهر). لم يكن السبب ضعف العظام، بل كثرة الاستلقاء في نفس الوضعية، وقلة تغيير الجهة.

بعد أن اتفقنا على جدول بسيط لتغيير الوضعية كل ساعتين، مع وسائد بين الركبتين وتحت الكعبين، وبعد عناية لطيفة بالجلد، بدأ الاحمرار يخف خلال أيام، وتجنّبنا تطور الجرح إلى قرحة عميقة.

الرسالة للأسرة: الخوف المفرط من الحركة قد يكون أخطر من الحركة نفسها عندما يتعلّق الأمر بقرح الفراش.

\section*{لماذا هذا الموضوع قريب لقلبي؟}

في إحدى زياراتي المنزلية، وجدت والدًا مسنًا يبتسم رغم أنه يجلس على نفس الكرسي منذ الصباح. لاحظت احمرارًا خفيفًا فوق العجز. قلت لابنه: "نقلّب الوالد الآن، قبل أن يتحول هذا الاحمرار لجرح متعب". خلال دقائق، وضعنا وسادة تحت الساقين ووسادة خلف الظهر، وطلبت منهم ضبط منبه كل ساعتين. بعد أسبوعين، اختفى الاحمرار تمامًا. هذه التجربة أكدت لي أن \textbf{الحركة البسيطة خير من علاج متأخر}.

هذا الفصل يضع خطة وقاية عملية للأسرة: كيف تفهم قرح الفراش، كيف تغيّر الوضعيات بأمان، وكيف تدعم الجلوس بوسائد وتمارين خفيفة. كل خطوة هنا جربتها مع عائلات حقيقية، وأعلم أنها قابلة للتطبيق في البيت.

\section*{1. فهم سريع لحالة كبير السن قبل التفكير في قرح الفراش}

قبل أن نقلق من قرح الفراش، نحتاج أولاً أن نرى الصورة كاملة: هل يتحرك كبير الأسرة فعلاً؟ كم ساعة يقضيها في السرير أو الكرسي؟ وهل يغيّر وضعيته بنفسه أم ينتظر من يحركه؟

تخيّل أن عندك ثلاثة مستويات: - كبير سن نشِط نسبياً: يتحرك داخل البيت، يقوم للصلاة، يجلس في الصالة، ويرجع لسريره. هذا أقل عرضة للقرح، لكن يحتاج متابعة. - كبير سن قليل الحركة: يجلس أغلب اليوم في الكرسي أو السرير، لكن يستطيع تغيير وضعيته إذا تذكر أو ذكّره أحد. - كبير سن طريح الفراش: لا يغيّر وضعه وحده، ويحتاج مساعدة في كل حركة.

كلما نزلنا درجة في هذه المستويات، زاد خطر قرح الفراش، وزاد دور الأسرة في الحماية.

\section*{2. جدول التقليب وتخفيف الضغط (أسلوب عملي)}
\begin{itemize}
\item
    في السرير: غيّر الوضعية كل 2-3 ساعات (ظهر → جانب يمين → جانب يسار). استخدم وسادة بين الركبتين وعند الكعبين، ووسادة خلف الظهر لتثبيت الميلان.
  
\item
    على الكرسي: كل 15-30 دقيقة اطلب من كبيرك الميل للأمام أو للجانب 10-15 ثانية، أو ساعده برفع جسمه قليلًا باستخدام مسندي الذراعين.
  
\item
    علّق جدولًا على الحائط، واضبط منبهًا في هاتفك ليذكرك. المشاركة بين أفراد الأسرة تحمي ظهر المرافق وتضمن الاستمرارية.
  
\end{itemize}

\subsection*{ إجراء: تقليب هادئ من الظهر إلى الجانب}

\textbf{الهدف:} تخفيف الضغط عن العجز والكعبين.

\textbf{كيف ننفذ؟} 1. أبلغ كبيرك: "سنقلبك على جانبك ليريح ظهرك". 2. ضع يدًا على الكتف ويدًا على الورك البعيد، وادفعهما ككتلة واحدة نحوك مع ثني ركبتيك (لا ظهرك). 3. ضع وسادة خلف الظهر، ووسادة بين الركبتين، وارفع الساقين قليلًا لرفع الكعبين عن المرتبة. 4. تأكد من راحة الرأس والأذن، واسأل المريض عن إحساسه.

\textbf{🛑 توقف فورًا إذا:} ألم حاد أو مقاومة شديدة، دوار أو شحوب، خروج أنبوب طبي. \textbf{ثم:} أعده للوضع السابق واتصل بالممرض/الطبيب.

\section*{3. جلوس مريح وآمن}
\begin{itemize}
\item
    \textbf{ارتفاع الكرسي:} الركبتان بزاوية 90 تقريبًا والقدمان ثابتتان.
  
\item
    \textbf{مسندي الذراعين:} يساعدان على تغيير الوزن والوقوف بأمان.
  
\item
    \textbf{وسادة مقعدة داعمة:} فوم متماسك أو وسادة هوائية بسيطة لتوزيع الضغط، مع فحص أسبوعي لحالتها.
  
\item
    \textbf{ضبط الحوض:} كل 30-60 دقيقة، أعد الجذع لوضع مستقيم لتجنب الانزلاق والاحتكاك في الظهر.
  
\end{itemize}

\subsection*{‍♂️ تمرين: رفع الكعبين وتحريك الدم}

\textbf{الهدف:} تنشيط الدورة الدموية وتخفيف ضغط الكعبين أثناء الجلوس الطويل.

\textbf{الخطوات:} 1. جالس بظهر مسنود وقدمان ثابتتان. 2. ارفع الكعبين ببطء (أصابع على الأرض) واثبت 3-5 ثوانٍ، ثم أنزل. كرر 5 مرات. 3. ارفع رؤوس الأصابع (الكعبان على الأرض) 3-5 ثوانٍ، ثم أنزل. كرر 5 مرات. 4. نفّذ الدورة كل ساعة جلوس.

\textbf{🛑 توقف فورًا إذا:} - شعر كبير السن بألم حاد مفاجئ في الساق أو الورك (وليس مجرد تعب عادي). - ظهر تشنج قوي في ربلة الساق أو الفخذ ولم يختفِ خلال ثوانٍ. - شعر بدوار شديد أو غثيان، أو أصبح لونه شاحباً بشكل واضح. - ظهرت صعوبة مفاجئة في التنفس أو كلام غير واضح.

في هذه الحالات: أوقف التمرين فوراً، أعده إلى وضع مريح وآمن، واتصل بالطبيب أو بالطوارئ حسب شدة الأعراض.

\subsection*{ ‍♂️ تمرين: إمالة الجذع للأمام والخلف}

\textbf{الهدف:} توزيع الضغط بعيدًا عن عظمتي الجلوس وتنشيط عضلات الظهر الخفيفة.

\textbf{الخطوات:} 1. جلوس مستقيم، اليدان على مسندي الذراعين. 2. مِل للأمام 20-30 درجة بظهر مستقيم واضغط قليلًا بالذراعين 5 ثوانٍ. 3. عد ببطء للوضع المحايد، ثم استند للخلف قليلًا. 4. كرر 8-10 مرات كل ساعة جلوس.

\textbf{🛑 توقف فورًا إذا:} دوار أو ألم ظهر حاد أو صعوبة تنفس. \textbf{ثم:} أعده لوضع محايد واطلب تقييمًا طبيًا إذا استمر العرض.

\section*{4. قوائم سلامة جاهزة}

\subsection*{غرفة النوم والسرير}

\begin{longtable}[]{@{}
  >{\raggedright\arraybackslash}p{(\columnwidth - 6\tabcolsep) * \real{0.1951}}
  >{\raggedright\arraybackslash}p{(\columnwidth - 6\tabcolsep) * \real{0.2682}}
  >{\raggedright\arraybackslash}p{(\columnwidth - 6\tabcolsep) * \real{0.3903}}
  >{\raggedright\arraybackslash}p{(\columnwidth - 6\tabcolsep) * \real{0.1464}}@{}}
\toprule\noalign{}\tableheadercolor
\begin{minipage}[b]{\linewidth}\raggedright
العنصر
\end{minipage} & \begin{minipage}[b]{\linewidth}\raggedright
ما الخطر؟
\end{minipage} & \begin{minipage}[b]{\linewidth}\raggedright
الإجراء الفوري
\end{minipage} & \begin{minipage}[b]{\linewidth}\raggedright
تم \textcolor{tipborder}{$\checkmark$} \end{minipage} \\
\begin{minipage}[b]{\linewidth}\raggedright
ارتفاع السرير
\end{minipage} & \begin{minipage}[b]{\linewidth}\raggedright
صعوبة التقليب والنهوض
\end{minipage} & \begin{minipage}[b]{\linewidth}\raggedright
اجعل سطح المرتبة عند ركبة المرافق تقريبًا
\end{minipage} & \begin{minipage}[b]{\linewidth}\raggedright
⬜
\end{minipage} \\
\begin{minipage}[b]{\linewidth}\raggedright
صلابة المرتبة
\end{minipage} & \begin{minipage}[b]{\linewidth}\raggedright
نقاط ضغط على العجز والكعب
\end{minipage} & \begin{minipage}[b]{\linewidth}\raggedright
مرتبة فوم متوسطة أو هوائية متناوبة حسب توصية المختص
\end{minipage} & \begin{minipage}[b]{\linewidth}\raggedright
⬜
\end{minipage} \\
\begin{minipage}[b]{\linewidth}\raggedright
رفع الكعبين
\end{minipage} & \begin{minipage}[b]{\linewidth}\raggedright
كعبان يضغطان لساعات
\end{minipage} & \begin{minipage}[b]{\linewidth}\raggedright
وسادة تحت الساقين لرفع الكعبين عن المرتبة
\end{minipage} & \begin{minipage}[b]{\linewidth}\raggedright
⬜
\end{minipage} \\
\begin{minipage}[b]{\linewidth}\raggedright
وسادة بين الركبتين
\end{minipage} & \begin{minipage}[b]{\linewidth}\raggedright
احتكاك الركبتين والكاحلين
\end{minipage} & \begin{minipage}[b]{\linewidth}\raggedright
وسادة بين الركبتين والكاحلين عند النوم الجانبي
\end{minipage} & \begin{minipage}[b]{\linewidth}\raggedright
⬜
\end{minipage} \\
\begin{minipage}[b]{\linewidth}\raggedright
جدول التقليب
\end{minipage} & \begin{minipage}[b]{\linewidth}\raggedright
ضغط مستمر على نفس النقطة
\end{minipage} & \begin{minipage}[b]{\linewidth}\raggedright
جدول على الحائط + منبه كل 2-3 ساعات
\end{minipage} & \begin{minipage}[b]{\linewidth}\raggedright
⬜
\end{minipage} \\
\begin{minipage}[b]{\linewidth}\raggedright
تجفيف الجلد
\end{minipage} & \begin{minipage}[b]{\linewidth}\raggedright
رطوبة تضعف الجلد
\end{minipage} & \begin{minipage}[b]{\linewidth}\raggedright
تجفيف لطيف وملابس قطنية جيدة التهوية
\end{minipage} & \begin{minipage}[b]{\linewidth}\raggedright
⬜
\end{minipage} \\
\begin{minipage}[b]{\linewidth}\raggedright
جرس النداء
\end{minipage} & \begin{minipage}[b]{\linewidth}\raggedright
عدم طلب المساعدة
\end{minipage} & \begin{minipage}[b]{\linewidth}\raggedright
هاتف أو جرس في متناول اليد بعد كل تقليب
\end{minipage} & \begin{minipage}[b]{\linewidth}\raggedright
⬜
\end{minipage} \\
\midrule\noalign{}
\endhead
\bottomrule\noalign{}
\endlastfoot
\end{longtable}

\subsection*{الكرسي أو الكرسي المتحرك}

\begin{longtable}[]{@{}
  >{\raggedright\arraybackslash}p{(\columnwidth - 6\tabcolsep) * \real{0.1951}}
  >{\raggedright\arraybackslash}p{(\columnwidth - 6\tabcolsep) * \real{0.2682}}
  >{\raggedright\arraybackslash}p{(\columnwidth - 6\tabcolsep) * \real{0.3903}}
  >{\raggedright\arraybackslash}p{(\columnwidth - 6\tabcolsep) * \real{0.1464}}@{}}
\toprule\noalign{}\tableheadercolor
\begin{minipage}[b]{\linewidth}\raggedright
العنصر
\end{minipage} & \begin{minipage}[b]{\linewidth}\raggedright
ما الخطر؟
\end{minipage} & \begin{minipage}[b]{\linewidth}\raggedright
الإجراء الفوري
\end{minipage} & \begin{minipage}[b]{\linewidth}\raggedright
تم \textcolor{tipborder}{$\checkmark$} \end{minipage} \\
\begin{minipage}[b]{\linewidth}\raggedright
ارتفاع المقعد
\end{minipage} & \begin{minipage}[b]{\linewidth}\raggedright
ضغط زائد وصعوبة نهوض
\end{minipage} & \begin{minipage}[b]{\linewidth}\raggedright
ركبتان بزاوية 90 وقدمان ثابتتان
\end{minipage} & \begin{minipage}[b]{\linewidth}\raggedright
⬜
\end{minipage} \\
\begin{minipage}[b]{\linewidth}\raggedright
وسادة المقعدة
\end{minipage} & \begin{minipage}[b]{\linewidth}\raggedright
تركيز الضغط على عظمتي الجلوس
\end{minipage} & \begin{minipage}[b]{\linewidth}\raggedright
وسادة فوم متماسك أو هوائية بسيطة + فحص أسبوعي
\end{minipage} & \begin{minipage}[b]{\linewidth}\raggedright
⬜
\end{minipage} \\
\begin{minipage}[b]{\linewidth}\raggedright
مسندي الذراعين
\end{minipage} & \begin{minipage}[b]{\linewidth}\raggedright
انزلاق واحتكاك في الظهر
\end{minipage} & \begin{minipage}[b]{\linewidth}\raggedright
تأكد من وجودهما وثباتهما
\end{minipage} & \begin{minipage}[b]{\linewidth}\raggedright
⬜
\end{minipage} \\
\begin{minipage}[b]{\linewidth}\raggedright
مسند القدمين
\end{minipage} & \begin{minipage}[b]{\linewidth}\raggedright
ضغط على الكعبين
\end{minipage} & \begin{minipage}[b]{\linewidth}\raggedright
مسند مبطن أو طبقة فوم وضبط الزاوية
\end{minipage} & \begin{minipage}[b]{\linewidth}\raggedright
⬜
\end{minipage} \\
\begin{minipage}[b]{\linewidth}\raggedright
فرامل الكرسي
\end{minipage} & \begin{minipage}[b]{\linewidth}\raggedright
حركة مفاجئة أثناء التقليب
\end{minipage} & \begin{minipage}[b]{\linewidth}\raggedright
شدّ الفرامل قبل أي تمرين أو تغيير وضعية
\end{minipage} & \begin{minipage}[b]{\linewidth}\raggedright
⬜
\end{minipage} \\
\begin{minipage}[b]{\linewidth}\raggedright
وضعية الحوض
\end{minipage} & \begin{minipage}[b]{\linewidth}\raggedright
جلوس منزلق يضغط العجز
\end{minipage} & \begin{minipage}[b]{\linewidth}\raggedright
إعادة ضبط الحوض كل 30-60 دقيقة، واستخدام وسادة مانعة للانزلاق
\end{minipage} & \begin{minipage}[b]{\linewidth}\raggedright
⬜
\end{minipage} \\
\begin{minipage}[b]{\linewidth}\raggedright
تخفيف الضغط المجدول
\end{minipage} & \begin{minipage}[b]{\linewidth}\raggedright
ضغط مستمر أثناء الجلوس
\end{minipage} & \begin{minipage}[b]{\linewidth}\raggedright
تذكير بالميلان أو الرفع القصير كل 15-30 دقيقة
\end{minipage} & \begin{minipage}[b]{\linewidth}\raggedright
⬜
\end{minipage} \\
\midrule\noalign{}
\endhead
\bottomrule\noalign{}
\endlastfoot
\end{longtable}

\textbf{تذكير:} أي وسادة أو مرتبة داعمة هي أداة مساعدة، لكنها لا تغني عن الحركة والتقليب.

\section*{5. متى أطلب المساعدة فورًا؟}
\begin{itemize}
\item
    احمرار يتحول لأرجواني/أسود، أو ظهور فقاعات أو رائحة كريهة.
  
\item
    ألم شديد جديد في منطقة الضغط رغم تخفيف الوزن.
  
\item
    عدم تحسن الاحمرار بعد يومين من التقليب والوسائد.
  
\end{itemize}

في هذه الحالات، تواصل مع الممرض المنزلي أو الطبيب مباشرة. دوري كمعالج طبيعي هو ضبط خطة الحركة وتخفيف الضغط، لكن تقييم الجرح الفعلي مسؤولية الفريق الطبي المختص.

\subsection*{متى نطلب الإسعاف فوراً؟}

قرح الفراش أحياناً تبدأ بسيطة، لكن في حالات معينة تصبح حالة طارئة: - إذا ظهر جرح عميق تستطيع أن ترى فيه طبقة صفراء أو بيضاء أو العظم. - إذا خرج من الجرح إفراز ذو رائحة قوية ومزعجة. - إذا ارتفعت حرارة كبير السن، مع رجفة أو تعب شديد غير معتاد. - إذا أحسست أن المنطقة حول الجرح حارة جداً، ومتورمة مقارنة بالجهة الأخرى.

هنا لا نكتفي بالعناية المنزلية؛ يجب التواصل مع الطبيب أو التوجه للطوارئ في المستشفى، لأن الالتهاب قد ينتشر داخل الجسم.

\section*{أسئلة تتكرر من الأسر}

س: هل الأفضل أن نضع كبير السن دائماً على جهة واحدة المفضلة عنده؟ ج: لا. البقاء على نفس الجهة طوال اليوم يرفع احتمال القرحة في هذه الجهة. حاولوا تغيير الوضعية بالتدرج وبهدوء بين الجانبين والظهر، مع استخدام الوسائد لتخفيف الضغط.

س: هل كل احمرار في الجلد يعني قرحة؟ ج: ليس بالضرورة. لو زال الاحمرار بعد تغيير الوضعية بدقائق، غالباً هو إنذار مبكر يمكن السيطرة عليه. لو استمر الاحمرار أو ازداد، هنا نعامله بجدية أكبر.

س: نخاف من تحريك جدتي لأنها "عظامها هشة"، ماذا نفعل؟ ج: الحذر مطلوب، لكن عدم الحركة أخطر. اتفقوا مع أخصائي العلاج الطبيعي على طريقة آمنة لتغيير الوضعية وعدد المرات، ويمكنكم طلب أن يعلّمكم عملياً في الزيارة.

\section*{خلاصة شخصية}

الوقاية من قرح الفراش في المنزل ليست سرًا طبيًا معقدًا؛ هي مزيج من \textbf{حركة بسيطة} + \textbf{وسائد ذكية} + \textbf{جدول ثابت}. كل مرة أرى أسرة تلتزم بهذه الثلاثية، أرى جلدًا أكثر صحة ومريضًا أكثر راحة، ومرافقًا أقل إجهادًا. لنحافظ على كرامة أحبتنا بحركة دائمة وراحة مدروسة.
