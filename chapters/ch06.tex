\chapter{كيف نصمّم برنامج تمارين منزلية آمن وواقعي لكبار السن؟}
\label{ch:06}

\section*{من الميدان}

أذكر مريضة في السبعين زرتها بعد شهرين من خروجها من المستشفى إثر كسر في الورك. وجدتها جالسة على كرسيها طول اليوم، والأسرة تقول: "نخشى أن تتعب."

سألتها: "كم مرة تقومين من الكرسي في اليوم؟" قالت: "مرتين فقط، للحمام."

بعد ثلاثة أشهر من برنامج تمارين بسيط --- خمس دقائق صباحاً وخمس دقائق مساءً --- صارت تمشي في الصالة وتجلس مع الضيوف بدون مساعدة.

ليس السر في تمارين معقدة، بل في \textbf{الاستمرارية والتدرج}.

\section*{لماذا أصرّ على التمارين المنزلية في كل زيارة؟}

في الرعاية المنزلية، أرى نمطًا يتكرر بشكل مزعج:
\begin{itemize}
\item
    كبير سن يقضي أغلب اليوم بين السرير والكنبة.
  
\item
    مشي محدود جدًا داخل البيت (إلى الحمام، أو طاولة الطعام فقط).
  
\item
    الأسرة تقول بحسن نية:\\
  \textgreater{} "نخشى أن يتعب\ldots{} نخشى أن يسقط\ldots{} دعه يرتاح."
  
\end{itemize}

لكن الأدلة العلمية تقول لنا شيئًا مختلفًا تمامًا:
\begin{itemize}
\item
    قلة الحركة مع العمر تزيد من ضعف العضلات (Sarcopenia).\\

\item
    ضعف العضلات والتوازن يزيدان خطر السقوط، والاعتماد على الآخرين، ومضاعفات صحية أخرى [1][2][3][8][18].
  
\item
    برامج التمارين البسيطة، المنتظمة، ساعدت في دراسات كبيرة على:

  \begin{itemize}
  \item
        تقليل خطر السقوط،
    
  \item
        تحسين القدرة على المشي،
    
  \item
        الإبقاء على الاستقلالية لفترة أطول [2][3][13][16][18].
    
  \end{itemize}
\end{itemize}

هذا الفصل موجه لك كأسرة:

كيف تُحوِّل بيتك إلى "مركز تأهيل صغير"\\
ببرنامج تمارين بسيط، آمن، وواقعي، يناسب كبير السن في بيتكم؟

لن نضع برنامجًا "مثاليًا على الورق" يصعب تطبيقه، بل خطة عملية:
\begin{itemize}
\item
    تحترم عمر وحالة كبير السن،
  
\item
    تراعي الأمراض المزمنة والحرارة في السعودية،
  
\item
    يمكن تنفيذها في غرفة وصالة، بدون أجهزة معقدة.
  
\end{itemize}

\section*{1. لماذا التمارين مهمة لكبار السن؟}

\subsection*{1.1 ما الذي نريد أن نحافظ عليه؟}

مع تقدم العمر، أهم ما نريد الحفاظ عليه عند كبير السن:
\begin{itemize}
\item
    القدرة على القيام والجلوس،
  
\item
    المشي داخل البيت وخارجه لمسافة معقولة،
  
\item
    توازن يمنع السقوط،
  
\item
    قوة كافية لحمل أغراض خفيفة، أو استخدام الحمام بأمان.
  
\end{itemize}

الدراسات توضح أن برامج النشاط البدني المنتظمة تؤثر على هذه الأمور جميعًا:
\begin{itemize}
\item
    تقوي العضلات،
  
\item
    تحسّن التوازن،
  
\item
    ترفع القدرة على التحمل،
  
\item
    تحافظ على الاستقلالية [1][2][3][11][13][18].
  
\end{itemize}

\subsection*{1.2 ماذا يحدث لو تركنا كبير السن بدون حركة؟}

"الراحة الكاملة" لكبير السن تؤدي إلى:
\begin{itemize}
\item
    ضعف عضلي سريع (خاصة في عضلات الفخذ والساق) [3][8][15].
  
\item
    زيادة تيبّس المفاصل وآلام الظهر والركبتين.
  
\item
    ارتفاع احتمال السقوط؛ لأن أي تعثر بسيط يصبح أخطر.
  
\item
    زيادة خطر مضاعفات أخرى (تقرحات، جلطات، إمساك\ldots).
  
\end{itemize}

دراسة LIFE الشهيرة على كبار السن بيّنت أن برنامج نشاط بدني منظم قلل بشكل واضح من الإعاقة الحركية الكبرى مقارنة بمجموعة "التثقيف الصحي بدون تمرين" [18].

\section*{2. مبادئ أساسية قبل البدء}

قبل أن نرسم أي برنامج، هناك مبادئ لا نتجاوزها:

\subsection*{2.1 سلامة أولاً (Safety First)}

قبل البدء ببرنامج جديد، خصوصًا إذا كان كبير السن يعاني من:
\begin{itemize}
\item
    أمراض قلب أو صدر،
  
\item
    ضيق نفس مع أقل مجهود،
  
\item
    دوخة متكررة أو إغماءات،
  
\item
    آلام صدر غير مفسرة،
  
\end{itemize}

فمن الضروري مراجعة الطبيب للتأكد من أن التمارين المقترحة مناسبة له [4][10][11][16].

 \vspace{8pt}\noindent\textcolor{warnborder}{\rule{3pt}{12pt}}\hspace{6pt}
\textbf{تنبيه مهم:}  إذا كان كبير السن يعاني من أحد هذه الأمراض، هناك احتياطات إضافية يجب مراعاتها. \textbf{لمرضى السكري:} راقب علامات انخفاض السكر كالتعرق والرجفة والدوخة، وتجنب التمارين قبل الأكل مباشرة أو بعده مباشرة. \textbf{لمرضى القلب والضغط:} ابدأ ببطء شديد، وتوقف إذا شعر المريض بألم في الصدر أو ضيق نفس غير معتاد. \textbf{للجميع:} استشر الطبيب المعالج قبل البدء ببرنامج جديد، خاصة إذا كانت الأمراض غير مستقرة.
\vspace{8pt}

\subsection*{2.2 التدرّج (Start Low, Go Slow)}

الأبحاث والتوصيات العالمية (مثل ACSM وAGS) تؤكد مبدأ:

نبدأ بشيء بسيط ومقبول\ldots  ونزيد تدريجيًا [2][3][4][10][15][18].
\begin{itemize}
\item
    لا نبدأ مباشرةً بمدة طويلة أو تمارين صعبة.
  
\item
    نزيد عدد التكرارات أو الدقائق تدريجيًا، بناءً على تحمّل كبير السن.
  
\end{itemize}

\subsection*{2.3 "الحديث المريح" كمعيار للجهد (Talk Test)}

طريقة عملية من الدراسات لمراقبة شدة التمرين بدون أجهزة:
\begin{itemize}
\item
    إذا كان كبير السن يستطيع التحدث بجمل كاملة أثناء التمرين بدون لهث واضح → شدة معتدلة (مقبولة غالبًا).
  
\item
    إذا كان لا يستطيع إلا كلمات قليلة ويتقطع نفسه → التمرين شديد أكثر من اللازم [8][9].
  
\end{itemize}

\subsection*{2.4 مقياس الجهد (Borg Scale -- مختصر للأسرة)}

مقياس بورغ للجهد (من 6 إلى 20) يستخدم كثيرًا في الأبحاث [8].\\
يمكن تبسيطه للأسرة كالتالي:
\begin{itemize}
\item
    0--1: راحة تامة.
  
\item
    2--3: مجهود خفيف.
  
\item
    4--5: مجهود متوسط (مستوى نريده كثيرًا من الوقت).
  
\item
    6--7: مجهود عالي (نستخدمه بحذر جدًا).
  
\item
    8--10: مجهود أقصى (غير مناسب لكبار السن في البيت).
  
\end{itemize}

\section*{3. أنواع التمارين التي نحتاجها في البيت}

برنامج التمارين المنزلي الجيد ليس نوعاً واحداً فقط. فكّر فيه كوجبة متكاملة: نبدأ بـ\textbf{الإحماء} لتجهيز الجسم، ثم ننتقل إلى \textbf{تمارين القوة} للعضلات الأساسية، تليها \textbf{تمارين التوازن} لمنع السقوط، ثم \textbf{تمارين المرونة} للحفاظ على ليونة المفاصل، وأخيراً \textbf{نشاط هوائي بسيط} كالمشي. كل عنصر له دوره، والجمع بينها هو سر النتائج الحقيقية.

الأدلة العلمية تدعم الجمع بين هذه الأنواع لتقليل السقوط وتحسين الوظيفة [2][3][12][13][15][16][18].

\section*{4. الإحماء (Warm-Up) -- تجهيز الجسم قبل التمرين}

\subsection*{4.1 لماذا الإحماء ضروري؟}

الإحماء يساعد على:
\begin{itemize}
\item
    رفع حرارة العضلات تدريجيًا،
  
\item
    تحسين تروية المفاصل،
  
\item
    تقليل خطر الإصابة العضلية [5][15].
  
\end{itemize}

\subsection*{4.2 نموذج إحماء بسيط (5--10 دقائق)}

يمكن للأسرة مساعدة كبير السن على:
\begin{itemize}
\item
    المشي البطيء داخل الصالة أو الممر لمدة 2--3 دقائق.
  
\item
    حركات بسيطة للذراعين (رفع وخفض) وهو جالس أو واقف.
  
\item
    حركات دائرية خفيفة للكاحلين (وهو جالس على كرسي).
  
\end{itemize}

\textbf{توقف فورًا إذا:}
\begin{itemize}
\item
    شعر بدوخة أو دوار.
  
\item
    شعر بألم حاد في الصدر أو ضيق نفس غير معتاد.
  
\item
    شعر بألم حاد في المفاصل (ليس مجرد شد بسيط).
  
\end{itemize}

في هذه الحالات، نوقف التمرين، نجلسه، ونراقب. إذا استمر الألم أو الأعراض، يجب استشارة الطبيب.

\section*{5. تمارين القوة (Strength Training)}

\subsection*{5.1 لماذا نركز على قوة عضلات الفخذ والساق؟}

الدراسات توضح أن قوة الأطراف السفلية مرتبطة بشكل وثيق بالقدرة على:
\begin{itemize}
\item
    القيام من الكرسي،
  
\item
    صعود الدرج،
  
\item
    المشي لمسافات معقولة،
  
\item
    تجنب السقوط [3][12][13][14][18].
  
\end{itemize}

\subsection*{5.2 مبادئ عامة لتمارين القوة}
\begin{itemize}
\item
    نختار تمارين تستخدم وزن الجسم أو مقاومة خفيفة (مثل زجاجة ماء صغيرة).
  
\item
    نهدف إلى \textbf{2--3 جولات في الأسبوع}، مع يوم راحة بينهما لنفس العضلات [3][4][15].
  
\item
    نبدأ بتكرارات قليلة (مثلاً 8 مرات)، ثم نزيد تدريجيًا إلى 10--12 حسب التحمل [3][15].
  
\end{itemize}

\subsection*{التمرين الأساسي: الجلوس والقيام من الكرسي}

أعتبر هذا التمرين الأهم على الإطلاق لكبار السن. لو لم يكن لديك وقت إلا لتمرين واحد، اختر هذا.

\textbf{لماذا هو مهم؟} لأن القيام من الكرسي حركة نؤديها عشرات المرات يومياً --- للصلاة، للحمام، للأكل. إذا ضعفت هذه الحركة، تأثرت الاستقلالية بشكل مباشر.

\textbf{الطريقة:}
\begin{itemize}
\item
    الوضعية: كرسي ثابت، ارتفاع مناسب، ويفضل وجود مسند للذراعين.
  
\item
    الخطوات:

  \begin{enumerate}
  \item
        يجلس كبير السن في منتصف الكرسي، قدماه على الأرض، الذراعان على الفخذين أو على مسندي الكرسي.
    
  \item
        يميل قليلاً للأمام، ثم يحاول الوقوف حتى تمام الوقوف.
    
  \item
        يجلس مرة أخرى ببطء، ويكرر الحركة.
    
  \end{enumerate}
\item
    التدرّج:

  \begin{itemize}
  \item
        نبدأ بـ 5 تكرارات، نستهدف الوصول إلى 10--12 مع الأيام.
    
  \end{itemize}
\end{itemize}

\textbf{توقف فورًا إذا:}
\begin{itemize}
\item
    ظهر ألم حاد في الركبة أو الورك.
  
\item
    حدث دوار مفاجئ أثناء الوقوف.
  
\item
    انقطع النفس بشكل واضح بعد تكرارات قليلة جدًا.
  
\end{itemize}

\textbf{تمرين: بسط الركبة جالسًا (Knee Extension)}\hfill\break
- يجلس على كرسي، يمد ساقه للأمام حتى تصبح مستقيمة قدر الإمكان، ثم ينزلها ببطء. - يمكن إضافة ثقل خفيف (مثل كيس صغير من الرمل لاحقًا) حسب توجيه أخصائي العلاج الطبيعي [3][15].

\section*{6. تمارين التوازن (Balance Exercises)}

\subsection*{6.1 لماذا التوازن مهم؟}

ضعف التوازن أحد أهم عوامل خطر السقوط. تمارين التوازن أظهرت في عدة مراجعات منهجية أنها تقلل السقوط عندما تُدمج في برامج التمارين لكبار السن [2][12][16][17].

\subsection*{6.2 مبادئ السلامة في تمارين التوازن}
\begin{itemize}
\item
    دائمًا يكون \textbf{هناك شخص بجانبه} أو أمامه.
  
\item
    تُجرى قرب سطح ثابت (طاولة، ظهر كرسي ثقيل، حافة مطبخ).
  
\item
    لا نجرب تمارين صعبة فجأة؛ نبدأ من الأسهل.
  
\end{itemize}

\subsection*{6.3 أمثلة لتمارين توازن بسيطة}

\textbf{تمرين: الوقوف مع مسك الكرسي بيد واحدة}\hfill\break
- يمسك بيد واحدة ظهر كرسي ثابت.\\
- يقلل الضغط على اليد تدريجيًا مع الوقت، لكن دون تركها بالكامل إذا كان التوازن ضعيفًا.

\textbf{تمرين: الوقوف مع تقريب القدمين}\hfill\break
- يقف والقدمان متقاربتان أكثر من المعتاد.\\
- يحاول الثبات 10--20 ثانية، مع وجود مرافق قريب.\\
- يمكن زيادة الصعوبة لاحقًا بتقليل الاعتماد على اليدين.

\textbf{توقف فورًا إذا:}
\begin{itemize}
\item
    بدأ يتمايل بشكل واضح.
  
\item
    قال: "أحس أني بطيح".
  
\item
    ظهرت دوخة أو غباش في النظر.
  
\end{itemize}

\section*{7. تمارين المرونة والإطاله (Stretching)}

\subsection*{7.1 الهدف من التمطية}
\begin{itemize}
\item
    الحفاظ على مدى حركة جيد في المفاصل،
  
\item
    تقليل التيبّس،
  
\item
    تحسين الشعور بالراحة بعد الجلوس الطويل [5][15].
  
\end{itemize}

\subsection*{7.2 مبادئ عامة للإطاله عند كبار السن}
\begin{itemize}
\item
    يكون التمدد لطيفًا، بلا "نط" أو حركات سريعة [5][15].
  
\item
    يمسك وضعية التمدد لمدة 10--20 ثانية، حسب التحمل.
  
\item
    لا نبحث عن "ألم قوي"، بل إحساس شد خفيف مقبول.
  
\end{itemize}

مثال بسيط:

\textbf{تمرين: إطاله} \textbf{خلف الفخذ (Hamstring Stretch)}\hfill\break
- يجلس على حافة الكرسي، يمد ساقًا واحدة للأمام، الكعب على الأرض، الركبة شبه مستقيمة.\\
- ينحني برفق للأمام حتى يشعر بشد لطيف خلف الفخذ، ثم يمسك الوضعية 10--15 ثانية.\\
- يكرر على الساق الأخرى.

\textbf{توقف فورًا إذا:}
\begin{itemize}
\item
    ظهر ألم حاد في الفخذ أو الركبة أو الظهر.
  
\item
    شعر بتنميل شديد أو خدر مفاجئ في الطرف.
  
\end{itemize}

\section*{8. النشاط الهوائي (Aerobic Activity)}

\subsection*{8.1 ما المقصود بالنشاط الهوائي؟}

هو أي نشاط يحرك مجموعات عضلية كبيرة لفترة متواصلة نسبيًا، مثل:
\begin{itemize}
\item
    المشي داخل البيت أو حول المنزل،
  
\item
    المشي في الممر أو الحوش،
  
\item
    أحيانًا باستخدام جهاز بسيط (مثل دراجة ثابتة) إذا كان متاحًا ومناسبًا [2][4][18].
  
\end{itemize}

\subsection*{8.2 كم نحتاج من النشاط الهوائي؟}

التوصيات العالمية لكبار السن القادرين تشير إلى:
\begin{itemize}
\item
    150 دقيقة أسبوعيًا من نشاط هوائي معتدل (يمكن تقسيمها على الأيام)،\\
  لكن هذا \textbf{هدف طويل المدى}، وليس نقطة البداية [2][4][18].
  
\end{itemize}

في البداية، قد نكتفي بـ:
\begin{itemize}
\item
    5--10 دقائق من المشي البطيء،\\

\item
    نكررها مرة أو مرتين في اليوم،\\

\item
    ثم نزيد المدة تدريجيًا حسب التحمل.
  
\end{itemize}

\subsection*{8.3 الانتباه للحرارة والجفاف في السعودية}

في بلد حار مثل السعودية، خصوصًا في الصيف أو مع المرضى أصحاب الأمراض المزمنة، يجب الحذر [6][7]:
\begin{itemize}
\item
    نفّذ أكثر التمارين داخل البيت في أوقات معتدلة (الصباح الباكر أو بعد المغرب).
  
\item
    تجنّب التمارين في أماكن غير مكيّفة خلال فترات الحر الشديد.
  
\item
    شجع كبير السن على شرب سوائل مناسبة (حسب تعليمات الطبيب، خاصة في مرضى القلب والكلى) [6][7].
  
\end{itemize}

\textbf{توقف فورًا إذا:}
\begin{itemize}
\item
    ظهر تعرّق شديد مفاجئ مع دوخة أو غثيان.
  
\item
    شعر بخفقان قوي في القلب أو ألم في الصدر.
  
\item
    ظهر صداع قوي مع شعور بحر شديد.
  
\end{itemize}

في هذه الحالات، أوقف التمرين، ابحث عن مكان بارد، وأبلغ الطبيب أو الطوارئ إذا لم تتحسن الأعراض.

\section*{9. كيف نرتّب برنامج أسبوعي بسيط؟}

\subsection*{9.1 نموذج جدول أسبوعي مرن}

يمكن للأسرة استخدام نموذج مبسّط مثل:
\begin{itemize}
\item
    \textbf{3 أيام في الأسبوع}: تمارين قوة + توازن + مرونة (مثلاً: السبت، الاثنين، الأربعاء).
  
\item
    \textbf{4--5 أيام في الأسبوع}: نشاط هوائي خفيف (مشي) في أيام متفرقة.
  
\end{itemize}

كل يوم تمارين قوة/توازن يمكن أن يشمل:

\begin{enumerate}
\item
    إحماء (5 دقائق مشي خفيف).\\

\item
    تمرين جلوس وقيام من الكرسي (1--2 مجموعة × 5--10 تكرارات).\\

\item
    تمرين توازن بسيط قرب الكرسي.\\

\item
    إطاله بسيط للعضلات المستخدمة.
  
\end{enumerate}

مع الوقت، يمكن زيادة:
\begin{itemize}
\item
    عدد التكرارات،
  
\item
    أو عدد المجموعات،
  
\item
    أو مدة المشي.
  
\end{itemize}

\subsection*{9.2 مبدأ "خطوة صغيرة، لكن ثابتة"}

دراسات التغيير السلوكي (مثل نموذج المراحل المتدرجة -- Transtheoretical Model) تشير إلى أن التغيير التدريجي، مع دعم الأسرة، أكثر استدامة من البدايات القوية القصيرة [19].
\begin{itemize}
\item
    لا نطالب كبير السن ببرنامج مثالي من اليوم الأول.
  
\item
    نحتفل بالإنجازات الصغيرة ("اليوم مشيت زيادة عن أمس"، "اليوم سويت جلوس وقيام مرتين بدل مرة").
  
\end{itemize}

\section*{10. من يساعد من؟ دور الأسرة في نجاح البرنامج}

من واقع التجربة:
\begin{itemize}
\item
    \textbf{البرنامج الذي تشارك فيه الأسرة} (تشجيع، متابعة، مشاركة في المشي) ينجح أكثر بكثير من برنامج يُترك للمريض وحده [12][13][18].
  
\item
    يمكن لأحد الأبناء أو الأحفاد أن:

  \begin{itemize}
  \item
        يمشي معه داخل البيت.
    
  \item
        يحسب التكرارات.
    
  \item
        يسجل الإنجازات في ورقة أو دفتر صغير (نوع من التحفيز).
    
  \end{itemize}
\end{itemize}

أيضًا:
\begin{itemize}
\item
    التزام الأسرة بجدول بسيط (مثلاً: "بعد العصر وقت تمارين جدي/جدتي") يساعد على تحويل التمارين إلى عادة يومية.
  
\end{itemize}

\subsection{\texorpdfstring{11. متى يجب \textbf{إيقاف التمرين فورًا} والاتصال بالطبيب؟}{11. متى يجب إيقاف التمرين فورًا والاتصال بالطبيب؟}}

هذه قائمة حمراء واضحة، يجب أن يعرفها كل من يشرف على التمارين:

اتصل بالطبيب أو الطوارئ إذا ظهر أثناء التمرين أو بعده مباشرة:
\begin{itemize}
\item
    ألم حاد في الصدر أو ضغط على الصدر.
  
\item
    ضيق نفس غير معتاد أو صعوبة في الكلام.
  
\item
    دوخة شديدة أو فقدان توازن مفاجئ مع سقوط.
  
\item
    خدر مفاجئ في وجه أو ذراع أو ساق، أو ضعف واضح في جهة واحدة.
  
\item
    صداع مفاجئ قوي جدًا غير معتاد.
  
\item
    خفقان قلب غير طبيعي أو إحساس قوي بعدم انتظام ضربات القلب.
  
\end{itemize}

وفي الحالات الأقل حدة:
\begin{itemize}
\item
    إذا استمر ألم المفاصل بشكل واضح بعد التمرين لفترة طويلة،\\
  أو ظهر انتفاخ واحمرار في المفصل،\\
  نخفف شدة التمرين ونتواصل مع الطبيب أو أخصائي العلاج الطبيعي.
  
\end{itemize}

\section*{12. خلاصة: كيف تجعل التمارين "جزء من الحياة" وليس "فصل مؤقت"؟}

من تجربة العمل مع كبار السن، ومن الأدلة العلمية [1][2][3][12][13][15][18][19]:

\begin{enumerate}
\item
    \textbf{الاستمرارية أهم من الكمال.}\hfill\break
  10 دقائق يوميًا، تستمر أشهر، أفضل من أسبوع مثالي ثم توقف.
  
\item
    \textbf{التمارين ليست عقابًا، بل استثمارًا} في القدرة على:

  \begin{itemize}
  \item
        الذهاب للحمام بدون مساعدة،
    
  \item
        الوضوء والصلاة براحة،
    
  \item
        زيارة الأقارب والحضور في المناسبات.
    
  \end{itemize}
\item
    \textbf{برنامج التمارين الجيد بسيط وواضح للأسرة}:

  \begin{itemize}
  \item
        يعرفون متى يتم إيقاف التمرين،
    
  \item
        يعرفون متى يتصلون بالطبيب،
    
  \item
        يعرفون كيف يزدون الجهد تدريجيًا بأمان.
    
  \end{itemize}
\item
    \textbf{بيئة البيت جزء من البرنامج:}

  \begin{itemize}
  \item
        إزالة عوائق السقوط،
    
  \item
        إضاءة جيدة،
    
  \item
        مكان مخصص وآمن لممارسة التمارين.
    
  \end{itemize}
\end{enumerate}

في النهاية، التمارين المنزلية لكبير السن ليست هدفًا لوحدها؛\\
هي وسيلة ليعيش:
\begin{itemize}
\item
    مستقلاً قدر الإمكان،
  
\item
    مشاركًا في حياة الأسرة،
  
\item
    متحركًا بأمان داخل بيته وخارجه.
  
\end{itemize}

\section*{المراجع}

[1] World Health Organization. (2015). \emph{World report on ageing and health}. WHO Press.

[2] Sherrington, C., Fairhall, N. J., Wallbank, G. K., et al.~(2019). Exercise for preventing falls in older people living in the community. \emph{Cochrane Database of Systematic Reviews}, 2019(1).

[3] Liu, C. J., \& Latham, N. K. (2009). Progressive resistance strength training for improving physical function in older adults. \emph{Cochrane Database of Systematic Reviews}, 2009(3).

[4] American College of Sports Medicine. (2018). \emph{ACSM's Guidelines for Exercise Testing and Prescription} (10th ed.). Wolters Kluwer.

[5] Woods, K., Bishop, P., \& Jones, E. (2007). Warm-up and stretching in the prevention of muscular injury. \emph{Sports Medicine}, 37(12), 1089--1099.

[6] Kenny, G. P., Yardley, J., Brown, C., Sigal, R. J., \& Jay, O. (2010). Heat stress in older individuals and patients with common chronic diseases. \emph{CMAJ}, 182(10), 1053--1060.

[7] Maughan, R. J., \& Shirreffs, S. M. (2010). Dehydration and rehydration in competitive sport. \emph{Scandinavian Journal of Medicine \& Science in Sports}, 20(Suppl 3), 40--47.

[8] Borg, G. A. (1982). Psychophysical bases of perceived exertion. \emph{Medicine and Science in Sports and Exercise}, 14(5), 377--381.

[9] Persinger, R., Foster, C., Gibson, M., Fater, D. C., \& Porcari, J. P. (2004). Consistency of the talk test for exercise prescription. \emph{Medicine and Science in Sports and Exercise}, 36(9), 1632--1636.

[10] American Geriatrics Society. (2001). Guideline for the prevention of falls in older persons. \emph{Journal of the American Geriatrics Society}, 49(5), 664--672.

[11] Liu, H. H. (2009). Assessment and management of falls and gait disorders in geriatric patients. \emph{Medical Clinics of North America}, 93(2), 355--369.

[12] Granacher, U., Muehlbauer, T., Gollhofer, A., Kressig, R. W., \& Zahner, L. (2011). An intergenerational approach in the promotion of balance and strength for fall prevention. \emph{Journal of the American Geriatrics Society}, 59(12), 2219--2228.

[13] de Vries, N. M., van Ravensberg, C. D., Hobbelen, J. S., et al.~(2012). Effects of physical exercise therapy on mobility, physical functioning, physical activity and quality of life in community-dwelling older adults with impaired mobility. \emph{BMC Geriatrics}, 12, 64.

[14] Bohannon, R. W. (2019). Grip strength: An indispensable biomarker for older adults. \emph{Clinical Interventions in Aging}, 14, 1681--1691.

[15] Kisner, C., \& Colby, L. A. (2020). \emph{Therapeutic Exercise: Foundations and Techniques} (7th ed.). F.A. Davis Company.

[16] Gillespie, L. D., et al.~(2012). Interventions for preventing falls in older people living in the community. \emph{Cochrane Database of Systematic Reviews}, 2012(9).

[17] Vellas, B. J., Wayne, S. J., Romero, L. J., Baumgartner, R. N., \& Garry, P. J. (1997). One-leg balance is an important predictor of injurious falls in older persons. \emph{Journal of the American Geriatrics Society}, 45(6), 735--738.

[18] Pahor, M., Guralnik, J. M., Ambrosius, W. T., et al.~(2014). Effect of structured physical activity on prevention of major mobility disability in older adults: the LIFE study randomized clinical trial. \emph{JAMA}, 311(23), 2387--2396.

[19] Prochaska, J. O., \& Velicer, W. F. (1997). The transtheoretical model of health behavior change. \emph{American Journal of Health Promotion}, 12(1), 38--48.
