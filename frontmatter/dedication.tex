\thispagestyle{empty}

\vspace*{3cm}

\begin{center}
{\Large\bfseries\textcolor{chaptercolor}{الإهداء}}

\vspace{0.8cm}
\textcolor{rulecolor}{\rule{0.3\textwidth}{0.6pt}}
\vspace{1cm}
\end{center}

\textbf{إلى والدي ووالدتي…}

اللذين علّماني معنى الصبر والعطاء، وأدركت من خلالهما أن رعاية الوالدين ليست واجباً فحسب، بل شرفٌ وفرصةٌ عظيمة للبر. هذا العمل ثمرةٌ من غرسكما؛ فكل خطوةٍ في مهنتي تقف خلفها أيادٍ ربّت، وقلوبٌ دعت، وعيونٌ سهرت. وأنا حين أتعامل مع كبار السن… أراهم بعين الابن الذي يتمنى لوالديه الأمان والكرامة؛ فأجتهد قدر استطاعتي كأن كل كبير سنٍّ هو أحدكما.

\vspace{1.2cm}

\textbf{إلى إخوتي وأخواتي…}

سندي بعد الله. أحمد الله على برّكم ورعايتكم لوالديّ رعايةً أفتخر بها، وأشهد أنكم في هذا الباب من المتفوقين الذين يُحتذى بهم. جزاكم الله عني خير الجزاء، وأدامكم الله لي ذخراً.

\vspace{1.2cm}

\textbf{إلى ابنتي نورة…}

أسأل الله أن يحفظكِ لي، وأتمنى — بعد عمرٍ طويل — أن أجد فيكِ اليد الحانية إذا احتجتُ للمساعدة يوماً.

\vspace{1.2cm}

\textbf{إلى زملائي وزميلاتي في الطب الطبيعي والتأهيل…}

شكراً لأنكم كنتم الرفقة، والنقاش، والدافع. ومنكم تعلمت أن الفريق القوي يخفف الألم قبل أن يصف العلاج.

\vspace{1.2cm}

\textbf{إلى زملائي وزميلاتي في منظومة التأهيل الطبي والرعاية الصحية المنزلية...}

شكراً لأنكم كنتم الرفقة التي أجد معها راحتي النفسية وسط ضغوط الميدان، ولأن لزمالتكم فضلاً كبيراً في مسيرتي المهنية. لقد وفرتم لي بيئة عملٍ ملهمة تشجع على التطوير والتحسين المستمر.

\vspace*{\fill}
\clearpage
