\chapter*{المقدمة}
\addcontentsline{toc}{chapter}{المقدمة}

\vspace{0.5cm}

\textbf{بسم الله الرحمن الرحيم}

\vspace{0.8cm}

إلى كل أسرة تحمل على كاهلها رعاية أحبائها من كبار السن… هذا الكتاب لكم.

\vspace{1cm}

\section*{لماذا كتبت هذا الكتاب؟}

لأنني رأيت نفس السؤال يتكرر في بيوت كثيرة: "وش نسوي الآن؟" أسرة محبة تريد الخير، لكنها قد تمنع كبير السن من الحركة خوفاً عليه، أو تنقله بطريقة خاطئة تتعبه وتتعب المرافق. المشكلة غالباً ليست نقص حب، بل نقص معرفة عملية بلغة البيت.

\vspace{1cm}

\section*{ماذا ستجد هنا؟}

هذا الكتاب يحوّل العلاج الطبيعي من مصطلحات معقدة إلى خطوات بسيطة وواضحة: كيف تلاحظ حالة كبير السن في البيت، كيف تنقل بأمان، كيف تقلل خطر السقوط، كيف تختار جهاز المساعدة، كيف تصمم برنامج تمارين منزلي واقعي، وكيف تتعامل مع قرح الفراش والشرقة وغيرها من المشاكل الشائعة.

\vspace{1cm}

\section*{ملاحظة}

قد تلاحظ أن الأسلوب يتراوح بين الفصحى ولهجة بيضاء قريبة من كلام الناس. هذا مقصود لتقريب المعنى في القصص والأمثلة، بينما التعليمات والتنبيهات تُكتب بوضوح وضبط أعلى. أنا لست أديباً ولا نحويّاً، وحديث تجربة في إخراج الكتب التعليمية، لكن المقصد هو الفائدة وسهولة التطبيق.

\vspace{1cm}

\section*{تنبيه}

هذا الكتاب للتثقيف ولا يغني عن الطبيب أو المختص. وفي حالات الخطر أو التدهور المفاجئ، الأولوية دائماً للسلامة وطلب المساعدة الطبية فوراً.

\vspace{1.5cm}

أسأل الله أن يجعل هذا العمل نافعاً، وأن يعين كل مقدم رعاية على هذا الشرف العظيم.

\vspace{1cm}

\begin{flushleft}
\textbf{عبدالكريم بن محمد الدوسري}
\end{flushleft}

\clearpage
