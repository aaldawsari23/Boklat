\chapter*{مقدمة الكتاب}
\addcontentsline{toc}{chapter}{مقدمة الكتاب}

\section*{لماذا كتبت هذا الكتاب؟}

في كل زيارة منزلية، أرى نفس المشهد يتكرر: ابن أو ابنة يحاولون مساعدة والدهم أو والدتهم، لكنهم لا يعرفون الطريقة الصحيحة. يرفعونهم بشكل يؤلم ظهورهم، يخافون من تحريكهم خشية إيذائهم، أو يتركونهم طريحي الفراش ظناً منهم أن الراحة هي الأفضل.

هذا الكتاب ولد من تلك اللحظات — من الأسئلة التي تتكرر في كل بيت، ومن الحاجة الحقيقية لدليل عملي يتحدث بلغة الأسرة السعودية ويفهم واقعها.

\section*{لمن هذا الكتاب؟}

\begin{itemize}
\item \textbf{الأبناء والبنات} الذين يرعون والديهم المسنين في المنزل
\item \textbf{المرافقون والمرافقات} الذين يعملون في رعاية كبار السن
\item \textbf{أخصائيو العلاج الطبيعي} الجدد في مجال الرعاية المنزلية
\item \textbf{كل من يريد أن يفهم} كيف يتغير الجسم مع التقدم بالعمر
\end{itemize}

\section*{ماذا ستجد في هذا الكتاب؟}

ستجد في هذا الكتاب ثلاثة عشر فصلاً تغطي كل ما تحتاجه لرعاية كبير السن حركياً في المنزل:

\begin{itemize}
\item فهم التغيرات الجسدية المصاحبة للشيخوخة
\item تقييم حالة المريض وقدراته الحركية
\item النقل الآمن وحماية ظهر المرافق
\item الوقاية من السقوط وتحسين التوازن
\item استخدام الأجهزة المساعدة بشكل صحيح
\item تصميم برنامج تمارين منزلية آمن
\item التعامل مع الألم والمشاكل الشائعة
\item الوقاية من قرح الفراش
\item الرعاية بعد الجلطات والكسور والعمليات
\item التعامل مع الرفض والخوف من الحركة
\item توثيق التقدم ومتابعته
\item نصائح عملية للتغذية الآمنة
\end{itemize}

\section*{رسالتي لك}

هذا الكتاب ليس بديلاً عن المتخصصين، لكنه رفيق يساعدك على فهم ما يحدث، ويمنحك الثقة للتعامل مع المواقف اليومية، ويُعلّمك متى تطلب المساعدة المتخصصة.

أتمنى أن يكون هذا الكتاب عوناً لك في رحلة رعاية أحبابك، وأن يُخفف عنك بعض العبء الذي تحمله.

\vspace{1cm}
\begin{flushleft}
\textbf{عبدالكريم الدوسري}\\
بيشة، 1446هـ
\end{flushleft}

\clearpage
