\chapter*{اللجنة العلمية للكتاب}
\addcontentsline{toc}{chapter}{اللجنة العلمية للكتاب}

\section*{سالم بن زقر}
\textbf{أخصائي علاج وظيفي} — مستشفى الملك عبدالله ببيشة

ركّزت مراجعته على الفصول ذات الطابع التأهيلي الوظيفي واستخدام الأجهزة المساعدة وتكييف بيئة المنزل، وهي: الفصل الثاني (قراءة حالة المريض)، الفصل الثالث (النقل الآمن)، الفصل الرابع (التوازن والسقوط)، الفصل الخامس (الأجهزة المساعدة)، الفصل السادس (برنامج التمارين المنزلية)، والفصل التاسع (ما بعد الجلطة والكسور)، مع التأكد من أن التوصيات تراعي أنشطة الحياة اليومية (ADLs) وقابلية تطبيقها في البيت.

\vspace{0.5cm}

\section*{خالد شبلي}
\textbf{ماجستير علاج طبيعي — أخصائي أول علاج طبيعي} — مستشفى الملك عبدالله ببيشة

ركّزت مراجعته على الفصول التي تجمع بين البعد العلمي المتقدم والتطبيق العملي: الفصل الأول (التغيرات الحركية)، الفصل الرابع (التوازن)، الفصل الخامس (المشي)، الفصل السادس (برنامج التمارين)، الفصل السابع (ألم التمارين)، والفصل التاسع (الجلطة والكسور)، مع مراجعة المراجع العلمية والتأكد من مواكبة التوصيات للإرشادات الحديثة.

\vspace{0.5cm}

\section*{ناصر محمد القحطاني}
\textbf{ممرض رعاية صحية منزلية} — برنامج الرعاية الصحية المنزلية، مستشفى الملك عبدالله ببيشة

راجع الفصول المرتبطة بعمل فرق الرعاية المنزلية: الفصل الثاني (ملاحظة الحالة)، الفصل الثالث (النقل الآمن)، الفصل الثامن (الجلوس الطويل وقرح الفراش)، الفصل الحادي عشر (دفتر المتابعة)، والفصل الثاني عشر (رسائل عملية)، مع التركيز على سلامة المريض ودور الأسرة والتوثيق الواقعي.

\vspace{0.5cm}

\section*{عبدالمؤمن أحمد صميلي}
\textbf{أخصائي علاج طبيعي — الرعاية المديدة} — مستشفى النقاهة والرعاية المديدة

راجع المحتوى المرتبط بطريحي الفراش والرعاية طويلة الأمد: الفصل الثالث (النقل الآمن لضعيفي الحركة)، الفصل الثامن (الجلوس الطويل وقرح الفراش)، الفصل الثالث عشر (وضعيات الأكل الآمنة)، والفصل الرابع عشر (طريح الفراش)، مع فقرات من الفصل التاسع المرتبطة بالمرضى في الرعاية المديدة.

\vspace{0.5cm}

\section*{د. سعد عبدالله الحويزي}
\textbf{طبيب رعاية صحية منزلية} — برنامج الرعاية الصحية المنزلية، مستشفى الملك عبدالله ببيشة

راجع المقدمة وتنبيهات السلامة الطبية، إضافة إلى: الفصل الثالث (المخاطر الطبية الحادة)، الفصل السابع (الألم التحذيري)، الفصل التاسع (مضاعفات الجلطات والكسور)، الفصل العاشر (التعامل مع الرفض والخوف)، والفصل الثالث عشر (الوقاية من الشرقة)، مع التأكد من وضوح متى يجب التواصل مع الطبيب أو الإسعاف.

\vspace{0.5cm}

\section*{خالد بدير المطيري}
\textbf{أخصائي علاج طبيعي} — مستشفى الملك عبدالله ببيشة

راجع الفصول التأهيلية الأساسية: الفصل الثاني (قراءة الحالة)، الفصل الثالث (النقل الآمن)، الفصل الرابع (التوازن)، الفصل الخامس (الأجهزة المساعدة)، الفصل السادس (برنامج التمارين)، والفصل التاسع (ما بعد الجلطة والكسور)، مع التأكد من مواكبة التوصيات للإرشادات الحديثة.

\vspace{1cm}

\begin{center}
\rule{0.4\textwidth}{0.5pt}
\end{center}

\vspace{0.5cm}

\textbf{ملاحظة مهمة:} يتحمّل المؤلف كامل المسؤولية عن الصياغة النهائية للكتاب، بينما اقتصرت مساهمة المراجعين على مراجعة المحتوى العلمي والعملي ضمن نطاق خبرتهم السريرية، دون أي مسؤولية قانونية عن طريقة تطبيقه الفردية.

\clearpage
