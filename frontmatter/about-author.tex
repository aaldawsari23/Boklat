\thispagestyle{empty}

\vspace*{1.5cm}

\begin{center}
{\Large\bfseries\textcolor{chaptercolor}{نبذة عن المؤلف}}

\vspace{0.8cm}
\textcolor{rulecolor}{\rule{0.3\textwidth}{0.6pt}}
\vspace{1cm}
\end{center}

\textbf{\large عبدالكريم بن محمد الدوسري}

\textbf{أخصائي علاج طبيعي مرخّص}

\vspace{0.8cm}

يمارس عمله السريري في المنطقة الجنوبية بالمملكة العربية السعودية، ويتنقّل بين مراحل رحلة المريض كاملة: من التنويم داخل المستشفى، إلى العيادات الخارجية، وصولاً إلى زيارات الرعاية المنزلية. تمنحه هذه الخبرة فهماً عملياً لما يحدث بعد الخروج من المستشفى؛ حيث تظهر التحديات داخل البيت: الخوف من السقوط، إجهاد المرافق، والعوائق اليومية التي لا تُرى في العيادة.

\vspace{0.8cm}

عمل المؤلف مع كبار السن في حالات شائعة ومتقدمة تشمل خشونة المفاصل، اضطرابات التوازن، ما بعد الجلطات والكسور واستبدال المفاصل، طريحي الفراش، ومشاكل البلع ومخاطر الشرقة. وخلال جائحة كورونا، عايش عن قرب تداخل الجانب البدني مع النفسي؛ لذلك يهتم في هذا الكتاب ليس فقط بما نفعله، بل أيضاً بكيف نقولها ونقنع كبير السن بالحركة بأمان وكرامة، وبطريقة تناسب بيئته وأسرته.

\vspace*{\fill}
\clearpage
